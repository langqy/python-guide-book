%!Mode:: "TeX:UTF-8"%確保文檔utf-8編碼
%新加入的命令如下:addchtoc addsectoc reduline showendnotes hlabel
%新加入的环境如下:common-format  fig  xverbatim

\documentclass[12pt,oneside]{book}
\newlength{\textpt}
\setlength{\textpt}{12pt}


\usepackage{myconfig}
\usepackage{mytitle}



\begin{document}
\frontmatter

\titlea{python3指南}
\titleb{用python3玩转电脑}
\author{万泽}
\authorinfo{作者:}
\editor{德山书生}
\email{a358003542@gmail.com}
\editorinfo{编者:}
\version{0.02}
\titleLB

\addchtoc{前言}
\chapter*{前言}
\begin{common-format}
需要提醒的是在本文档中不管是xverbatim环境生成的code文件还是cverbatim环境,最前面都多了一个空行,因为最前面有一个newlinechar符号最后转变成换行了,我还不知道如何避免,所以,如果你需要将代码文件以可执行模式执行(以脚本文件模式载入的没有问题),你需要进入文件按一下Backspace键将第一行消去即可。

主要参考资料:

1.\href{https://drive.google.com/open?id=0ByWxOeitx54PSW40bU5zNVhuMlU&authuser=0}{python入门教程}  作者:Guido van Rossum  Fred L. Drake

2.learning python v5  主要python语言参考   python学习手册(第四版) 老鼠版

3.programming python v4 蟒蛇版

4.python官网的\href{https://docs.python.org/3/}{参考文档}。

5. numpy, scipy, matplotlib等第三方模块官网发布的pdf手册。

6.pyqt4 tutorial \href{http://zetcode.com/gui/pyqt4/}{英文网站} 也参考了jimmykuu的中文翻译,\href{http://blog.cx125.com/books/PyQt4_Tutorial/}{中文翻译网站}

7.dive into python3 \href{http://www.diveintopython3.net/index.html}{dive into python3}


正则表达式部分:
1.\href{http://wiki.ubuntu.org.cn/Python%E6%AD%A3%E5%88%99%E8%A1%A8%E8%BE%BE%E5%BC%8F%E6%93%8D%E4%BD%9C%E6%8C%87%E5%8D%97}{Python正则表达式操作指南}

2.\href{https://docs.python.org/3.4/howto/regex.html#regex-howto}{regex-howto}

3.\href{http://my.oschina.net/o0Kira0o/blog/138516#OSC_h4_2}{Python下的正则表达式原理和优化笔记}


%这里空一行。

\end{common-format}


\addchtoc{目录}
\setcounter{tocdepth}{2}
\tableofcontents

\begin{common-format}
\mainmatter

\part{python3基础}

\chapter{beginning}
\section{python简介}
Python是个成功的脚本语言。它最初由Guido van Rossum开发,在1991年第一次发布。Python由ABC和Haskell语言所启发。Python是一个高级的、通用的、跨平台、解释型的语言。一些人更倾向于称之为动态语言。它很易学,Python是一种简约的语言。它的最明显的一个特征是,不使用分号或括号,Python使用缩进。现在,Python由来自世界各地的庞大的志愿者维护。

python现在主要有两个版本区别,python2和python3。作为新学者推荐完全使用python3编程,本文档完全基于python3。

完全没有编程经验的人推荐简单学一下c语言和scheme语言(就简单学习一下这个语言的基本概念即可)。相信我学习这两门语言不会浪费你任何时间,其中scheme语言如果你学得深入的话甚至编译器的基本原理你都能够学到。了解了这两门语言的核心理念,基本上任何语言在你看来都大同小异了。

\section{进入python的REPL环境}
在ubuntu13.10下终端中输入python即进入python语言的REPL环境,目前默认的是python2。你可以运行:\\
\verb+python  --version+\\
来查看。要进入python3在终端中输入python3即可。


\section{python3命令行用法}
命令行的一般格式就是:\\
\verb+python3  [可选项]  test.py  [可选参数1 可选参数2]+

同样类似的运行\verb+python3  --help+即可以查看python3命令的一些可选项。比如加入\textbf{-i}选项之后,python执行完脚本之后会进入REPL环境继续等待下一个命令,这个在最后结果一闪而过的时候有用。后面的-c,-m选项还看不明白。

\subsection{python执行脚本参数的传递}
上面的命令行接受多个参数都没有问题的,不会报错,哪怕你在py文件并没有用到他们。在py文件中要使用他们,首先导入sys模块,然后sys.argv[0]是现在这个py文件在系统中的文件名,接下来的sys.argv[1]就是之前命令行接受的第一个参数,后面的就依次类推了。


\section{geany的相关配置}
geany的其他配置这里不做过多说明,就自动执行命令默认的应该是python2,修改成为:\\
\verb+python3  -i  %f  +\\
即可。


\section{代码注释}
python语言的注释符号和bash语言(linux终端的编程语言)一样用的是\#{}符号来注释代码。然后py文件开头一般如下面代码所示:
\begin{Verbatim}
#!/usr/bin/env python3
#-*-coding:utf-8-*-
\end{Verbatim}
其中代码第一行表示等下如果py文件可执行模式执行那么将用python3来编译\footnote{也就是用chmod加上可执行权限那么可以直接执行了。第一行完整的解释是什么通过\textit{env}程序来搜索python的路径,这样代码更具可移植性。},第二行的意思是py文件编码是utf-8编码的,python3直接支持utf-8各个符号,这是很强大的一个更新。


多行注释可以利用编辑器快速每行前面加上\#{}符号。

\section{Unicode码支持}
前面谈及python3是可以直接支持Unicode码的,如果以可执行模式加载,那么第二行需要写上:\\
\verb+#-*-coding:utf-8-*-+\\
这么一句。

\begin{Verbatim}
#!/usr/bin/env python3
#-*-coding:utf-8-*-
print('\u2460')
\end{Verbatim}
上面的数字是具体这个Unicode符号的十六进制。

\section{代码多行表示一行}
这个技巧防止代码越界所以经常会用到。用反斜线\textbackslash 即可。不过通常更常用的是将表达式用圆括号( )括起来,这样内部可以直接换行并继续。在python中任何表达式都可以包围在圆括号中。

\subsection{一行表示多行}
python中一般不用分号,但是分号的意义大致和bash或者c语言中的意义类似,表示一行结束的意思。其中c语言我们知道是必须使用分号的。


\section{输入和输出}
\subsection{最基本的input和print命令}
input函数请求用户输入,并将这个值赋值给某个变量。注意赋值之后类型是字符串,但后面你可以用强制类型转换——int函数(变成整数),float函数(变成实数),str函数(变成字符串)——将其转变过来。print函数就是一般的输出函数。

\begin{Verbatim}
x=input('请输入一个实数:')
string001='你输入的这个实数乘以2等于:'+ str(float(x)*2)
print(string001)
\end{Verbatim}





\chapter{程序中的逻辑}
\section{布尔值}
\label{sec:布尔值}
boolean类型,和大多数语言一样,就两个值:\textbf{True},\textbf{False}。然后强制类型转换使用函数\textbf{bool}。

\subsection{其他逻辑小知识}
在python中,有些关于逻辑真假上的小知识,需要简单了解下。
\begin{itemize}
\item 数0、空对象或者其他特殊对象None值都认为是假\sidenote{比如列表都是真,但空列表是假。}
\item 其他非零的数字或非空的对象都认为是真
\item 前面两条用bool函数可以进行强制类型转换
\item 比较和相等测试会递归作用在数据结构中
\item 比较和相等测试会返回True或False(1和0的custom version(翻译为定制版?))
\end{itemize}

\section{条件判断}
python中的条件语句基本格式如下:
\begin{Verbatim}
if  test:
    条件判断执行区块
\end{Verbatim}
也就是if命令后面跟个条件判断语句,然后记住加个冒号,然后后面缩进的区块都是条件判断为真的时候要执行的语句。

\begin{Verbatim}
if  test:
    do something001
else :
    do something002
\end{Verbatim}
这里的逻辑是条件判断,如果真,do something001;如果假,do something002。

\begin{Verbatim}
if  test001:
    do something001
elif test002:
    do something002
\end{Verbatim}
显然你一看就明白了,elif是else和if的结合。






\subsection{逻辑与或否}
and表示逻辑与,or表示逻辑或,not表示逻辑否。

下面编写一个逻辑,判断一个字符串,这个字符串开头必须是a或者b,结尾必须是s,倒数第二个字符不能是单引号'。在这里就演示一下逻辑。。
\begin{xverbatim}[129]{py}
x='agais'
if ((x[0] == 'a' or x[0] == 'b')
    and x[-1] =='s'
    and (not x[-2] =="'")):
    print('yes it is..')
\end{xverbatim}



\subsection{稍复杂的条件判断}
现在我们了解了if,elif和else语句,然后还了解了逻辑与或非的组合判断。那么在实际编程中如何处理复杂的条件逻辑呢?

首先能够用逻辑语句与或非组合起来的就将其组合起来,而不要过分使用嵌套。如下面代码所示,如果一个情况分成两部分,那么就用if...eles...语句,
\begin{tcbpython}[]
x=-2
if x>0:
    print('x大于0')
else:
    print('x小于0')
\end{tcbpython}

而如果一个情况分成三部分,那么就用if...elif...else语句。同一深度的这些平行语句对应的是“或”逻辑,或者说类似其他编程语言的switch语句。
\begin{tcbpython}
x=2
if x>0:
    print('x大于0')
elif x<0:
    print('x小于0')
else:
    print('x等于0')
\end{tcbpython}

我们再看一看下面的代码,这个代码是\emph{错误的},两个if语句彼此并不构成逻辑分析关系。\footnote{四个甚至更多的平行或逻辑就用更多的elif,读者请自己实验一下。}
\begin{tcbpython}
x=2
if x>0:
    print('x大于0')
if x<0:
    print('x小于0')
else:
    print('x等于0')
\end{tcbpython}


然后我们看到下面的代码,这个例子演示的是在加深一个深度的条件判断语句它当时处于的逻辑判断情况,这个语句的条件判断逻辑是本语句的判断逻辑再和左边(也就是前面)的深度的判断逻辑的“与”逻辑,或者说成是“交集”。比如说\textit{print('0<x<2')}这个语句就是本语句的判断逻辑\textit{x<2}和上一层判断逻辑\textit{x>0}的“交集”,也就是\textit{0<x<2}。

\begin{tcbpython}
x=-2
if x>0:
    print('x大于0')
    if x>2:
        print('x>2')
    elif x<2:
        print('0<x<2')
    else:
        print('x=2')
elif x<0:
    print('x小于0')
else:
    print('x等于0')
\end{tcbpython}


整个过程的情况如下图所示:
\begin{linefig}{复杂条件判断}
\caption{复杂条件判断}
\label{fig:复杂条件判断}
\end{linefig}
为了在编程的时候对处于何种判断逻辑之下有一个清晰的认识,强烈建议读者好好思考一下。毕竟磨刀不误砍柴功。





\subsection{try语句捕捉错误}
try语句是编程中用来处理可能出现的错误或者已经出现但并不打算应付的错误最通用的方式。比如一个变量你预先想的是接受一个数值,但是用户却输入了一个字符,这个时候你就可以将这段语句包围在try里面;或者有时你在编程的时候就发现了这种情况,只是懒得理会他们,那么简单的把这块出错的语句包围在try里面,然后后面跟个except语句,打印出一个信息“出错了”,即可。用法如下所示:
\begin{tcbpython}
while True:
    x=input('请输入一个数,将返回它除以2之后的数值\n输入"quit"退出\n')
    if x=='quit':
        break
    try :
        num=float(x)
        print(num/2)
    except:
        print('出错了')
\end{tcbpython}






\subsection{in语句}
in语句对于可迭代对象都可以做出是否某个元素包含在某个对象之中的判断。
\begin{Verbatim}
>>> 'a' in ['a',1,2]
True
>>> dict
{'a': 1, 'c': 2, 'b': 3, 'd': 4}
>>> 'e' in dict
False
>>> '2' in dict
False
\end{Verbatim}
从上面例子可以看到,一般的列表判断元素是否存在和我们之前预料的一致,关于字典需要说的就是in语句\uwave{只判断键},不判断值。






\section{迭代}
一般有内部重复操作的程序可以先考虑for迭代结构实现,实在不行才考虑while循环结构,毕竟简单更美更安全。

python的for迭代语句有点类似lisp语言的dolist和dotimes函数,具体例子如下:
\begin{xverbatim}[129]{py}
for x in 'abc':
    print(x)
\end{xverbatim}
in后面跟的是\textbf{序列}类型,也就是字符串,列表,数组都是可以的。这个语句可以看作先执行x='a'或者类似的匹配赋值操作,然后执行缩进的区块,后面依次类推。(所以for语句也支持序列解包赋值,请参看:\ref{sec:序列解包赋值})


\subsection{range函数}
range函数是为for迭代语句准备的,有点类似于lisp的dotimes函数,但是功能更全更接近common-lisp的loop宏了。

\verb+range(1,10,2)+\\
range函数的用法如上,表示从1开始到10,步长为2,如果用list函数将其包裹,将会输出[1,3,5,7,9]。如果不考虑步长的话,这个range函数就有点类似于在序列调出多个值那一小节\ref{sec:调出多个值}谈论的区间的情况。所以range(10)就可以看作[0,10),range(1,10)就可以看作[1,10)。但是在这里再加上步长的概念和区间的概念又有所不同了,range函数产生的是一个什么迭代器对象,目前我只知道这个对象和之前谈论的序列对象是不同的。

\begin{xverbatim}[129]{py}
for x in range(-10,-20,-3):
    print(x)
\end{xverbatim}
上面例子还演示了range的负数概念,这里如果用区间概念来考察的话,是不能理解的,之所以行得通,是因为它的步长是负数,如果不是负数,那么情况就会和之前讨论的结果类似,将是一个空值。


\subsection{迭代加上操作}
迭代产生信息流并经过某些操作之后生成目标序列,更多内容请参见列表解析一节\ref{sec:列表解析}。
\begin{Verbatim}
>>> squares=[x**2 for x in [1,2,3,4,5]]
>>> squares
[1, 4, 9, 16, 25]
\end{Verbatim}




\section{循环}
while语句用法和大多数编程语言类似,就是条件控制,循环结构。
\begin{Verbatim}
while test:
    do something
else :
    do something
\end{Verbatim}

值得一提的是else语句和while语句属于一个整体,通常情况下while执行完了然后执行下面的语句似乎不需要加上else来控制(最后一下while是False所以会跳转到执行else的语句那里。)。不过else语句的一个功用就是如果while循环的时候遇到break那么else语句也不会执行而是直接跳过去了,见下面。

\subsection{break命令}
break跳出最近的while或者for循环结构。前面谈到了else和while语句构成一个整体的时候,break可以跳过else语句。

\subsection{continue命令}
continue命令接下来的循环结构的执行区块将不执行了,跳到条件判断那里看看是不是继续循环。如果是,那么继续循环。

值得一提的是for语句也有else分句用法,虽然不常用。同样for语句也支持break命令,也一样跳出else分句。同样for语句也支持continue命令。

\subsection{pass命令}
pass命令就是什么都不做。pass命令即可用于循环语句也可用于条件语句。

pass命令什么都不做似乎没有什么意义,不过作为一个空占位符还是很有用的。比如你编写一个大型的GUI程序,信号-槽机制都构思好了,只是对应的函数暂时还没写好,这个时候你可以将对应的函数,只是空的函数名加上pass语句写上,这样整个程序就可以继续边编写边调试了。




\chapter{程序中的操作对象}
python和c语言不同,c 是什么\verb+int x = 3+ ,也就是这个变量是整数啊,字符啊什么的都要明确指定,python不需要这样做,只需要声明\verb+x = 3+即可。但是我们知道任何程序语言它到最后必然要明确某一个变量(这里也包括后面的更加复杂的各个结构对象)的内存分配,只是python语言帮我们将这些工作做了,所以就让我们省下这份心吧。

\begin{Verbatim}
''' 这是一个多行注释
    你可以在这里写上很多废话
    '''
x = 10
print(x,type(x))
\end{Verbatim}

python程序由各个模块(modules)组成,模块就是各个文件。模块由声明(statements)组成,声明由表达式(expressions)组成,表达式负责创造和操作对象(objects)。在python中一切皆对象。python语言内置对象(数值、字符串、列表、数组、字典、文件、集合、其他内置对象。)后面会详细说明之。


\section{赋值}
python中的赋值语法非常的简单,x=1,就是一个赋值语句了。和c语言不同,c是必须先声明int x之类,开辟一个内存空间,然后才能给这个x赋值。而python的x=1语句实际上至少完成了三个工作:一,判断1的类型(动态类型语言必须要这步);二,把这个类型的对象存储在内存里面;三,创建x这个名字和这个名字指向这个内存,x似乎可以称之为对应c语言的指针对象。

\subsection{序列赋值}
\begin{xverbatim}[129]{py}
x,y=1,'a'
[z,w]=['b',10]
print(x,y,z,w)
\end{xverbatim}

我们记得python中表达式可以加上圆括号,所以这里\verb+x,y+产生的是一个数组\verb+(x,y)+,然后是对应的数组平行赋值,第二行是列表的平行赋值。这是一个很有用的技巧。

在其他语言里面常常会介绍swap函数,就是接受两个参数然后将这两个参数的值交换一下,交换过程通常要用到临时变量。而在python中不需要再创建一个临时变量了,因为序列赋值会自动生成一个临时的右边的序列(其中的变量都对应原来的原始值),然后再\uwave{一一对应}赋值(这里强调一一对应是指两边的序列长度要一致。)

\subsubsection{交换两个元素}
在python中交换两个元素用序列赋值形式是很便捷的:
\begin{Verbatim}
>>> x = 1
>>> y = 2
>>> x,y = y,x
>>> print(x,y)
2 1
\end{Verbatim}
这个过程显然不是先执行x=y然后执行y=x,如上所述的,程序首先右边创建一个临时的序列,其中的变量都对应原来的值,即\verb+x,y=(2,1)+,然后再进行序列赋值。



\subsection{同时赋相同的值}
\begin{xverbatim}[129]{py}
x=y='a'
z=w=2
print(x,y,z,w)
\end{xverbatim}

这种语句形式c语言里面也有,不过内部实现机制就非常的不一样了。python当声明x=y的时候,x和y是相同的指针值,然后相同的指针值都指向了'a'这个字符串对象,也可以说x和y就是一个东西,只是取的名字不同罢了。

我们用is语句\footnote{is语句用来测试对象的同一性,就是真正是内存里的同一个东西,而不仅仅是值相同而已。==只是确保值相同。}来测试,显示x和y就是一个东西。
\begin{Verbatim}
>>> x=y='a'
>>> x is y
True
>>> x == y
True
>>> 
\end{Verbatim}


但如果写成这种形式:
\begin{Verbatim}
>>> x = 'a'
>>> y = 'a'
>>> x is y
True
\end{Verbatim}
x和y还是指向的同一个对象,关于这点python内部是如何实现的我还不太清楚(似乎有点神奇)。为了说明is语句功能正常这里再举个例子吧:
\begin{Verbatim}
>>> x = [1,2,3]
>>> y = [1,2,3]
>>> x == y
True
>>> x is y
False
\end{Verbatim}
我们看到这里就有了两个列表对象,我的一个推测是可变的对象会多次生成,而不可变的对象多个变量是共用的。那么我们看一下元组的情况:
\begin{Verbatim}
>>> x = (1,2,3)
>>> y = (1,2,3)
>>> x is y
False
\end{Verbatim}
元组不可变,不过他们也不是两个共用的,打住了,这个问题到此吧,有点偏题了。



\subsection{增强赋值语句}
x=x+y可以写作x += y。类似的还有:
\begin{tabular}{|c|c|c|}
\hline 
+= & \&{}= & >>= \\ 
\hline 
-= & |= & <<= \\ 
\hline 
*= & \^{}= & **= \\ 
\hline 
/= & \%{}= & //= \\ 
\hline 
\end{tabular} 

\subsection{序列解包赋值}
具体内容请参看后面的序列解包赋值这一小节\ref{sec:序列解包赋值}。

\subsection{可迭代对象的迭代赋值}
在我们对python语言有了深入的了解之后,我们发现python中迭代思想是深入骨髓的。我们在前面接触了序列的赋值模式之后,发现似乎这种赋值除了临时创建右边的序列之外,还似乎与迭代操作有关,于是我们推测python的这种平行赋值模式可以扩展到可迭代对象,然后我们发现确实如此!
\begin{Verbatim}
>>> x,y,z= map(lambda x : x+2,[-1,0,1])
>>> print(x,y,z)
1 2 3
\end{Verbatim}

最后要强调一点的是确保变量名和后面的可迭代对象的输出元素数目是一致的,当然进一步扩展的序列解包赋值也是支持的:
\begin{Verbatim}
>>> x,y,*z= map(lambda x : x+2,[-1,0,1,2])
>>> print(x,y,z)
1 2 [3, 4]
\end{Verbatim}
通配赋值,我喜欢这样称呼了,通配之后收集的元素在列表里面;而函数参数的婆娘通配传递,收集的元素在元组里面。

最后我们总结到,可迭代对象的赋值就是迭代操作加上各个元素的一对一的赋值操作。


\section{数值}
python的数值的内置类型有:int,float,complex等\footnote{这些int、float等命令都是强制类型转换命令}。\\python的基本算术运算操作有加减乘除(+ - * /)。然后‘=’表示赋值,类似数学书上的中缀表达式和优先级和括号法则等,这些都是一般编程语言说到烂的东西了。

\begin{Verbatim}
print((1+2)*(10-5)/2)
print(2**100)
\end{Verbatim}

\subsection{二进制八进制十六进制}
二进制的数字以0b(零比)开头,八进制的数字以0o(零哦)开头,十六进制的数字以0x(零艾克斯)开头。
\begin{Verbatim}
0b101010, 0o177, 0x9ff
\end{Verbatim}

以二进制格式查看数字使用bin命令,以十六进制查看数字使用hex命令。
\begin{Verbatim}
>>> bin(42)
'0b101010'
>>> hex(42)
'0x2a'
\end{Verbatim}

\subsubsection{进制转换小程序}

\begin{tcbpython}[]
number=input("请输入一个数字:")
number= eval(number)
#
radix= input('''请输入你想转换的进制系统
2   表示  二进制
8   表示  八进制
16  表示  十六进制
''')
radix =eval(radix)

while True:
    if radix == 2:
        print(bin(number))
        break
    elif radix == 8:
        print(oct(number))
        break
    elif radix == 16:
        print(hex(number))
        break
    else:
        print("sorry you input the wrong radix")
\end{tcbpython}
程序运行的情况如下所示:
\begin{Verbatim}
请输入一个数字:20
请输入你想转换的进制系统
2   表示  二进制
8   表示  八进制
16  表示  十六进制
8
0o24
\end{Verbatim}





\subsection{数学幂方运算}
$ x^y $,x的y次方如上面第二行所述就是用\verb+x**y+这样的形式即可。此外pow函数作用是一样的,\verb+pow(x,y)+。


\subsection{数值比较}
数值比较除了之前提及的>,<,==之外,>=,<=,!=也是有的(大于等于,小于等于,不等于)。此外python还支持连续比较,就是数学格式$a<x<b$,x在区间$(a,b)$的判断。在python中可以直接写成如下形式:\verb+a<x<b+。这实际实现的过程就是两个比较操作的进一步与操作。

\subsection{相除取商或余}
就作为正整数相除使用\verb+x//y+得到的值意义还是很明显的就是\textbf{商}。带上负号感觉有点怪了,这里先略过。相关的还有\textbf{取余}数,就是\verb+x%y+,这样就得到x除以y之后的余数了,同样带上负号情况有变,这里先略过。



\subsection{复数}
python直接支持复数, 复数的写法是类似\verb|1+2j|这样的形式,然后如果z被赋值了一个复数,这样它就是一个复数类型,那么这个类具有两个属性量,\textbf{real}和\textbf{imag}。也就是使用\verb+z.real+就给出这个复数的实数部。imag是imaginary number的缩写,虚数,想像出来的数。

\subsection{abs函数}
大家都知道abs函数是绝对值函数,这个python自带的,不需要加载什么模块。作用于复数也是可以的:
\begin{Verbatim}
z=3+4j
print(z.real,z.imag)
print(abs(z))
\end{Verbatim}

这个和数学中复数绝对值的定义完全一致,也就是复数的模:\\
$ \left| z \right| =\sqrt { a^{ 2 }+b^{ 2 } }  $

\subsection{round函数}
简单的理解就是这个函数实现了对数值的\uwave{四舍五入}功能。
\begin{Verbatim}
>>> round(3.1415926)
3
>>> round(3.1415926,0)
3.0
>>> round(3.1415926,1)
3.1
>>> round(3.1415926,2)
3.14
>>> round(3.1415926,4)
3.1416
\end{Verbatim}

这里第二个参数接受0或者负数多少有点没意义了,一般使用还是取1或大于1的数吧,意思就是保留几位小数。


\subsection{min,max和sum函数}
\label{sec:sum函数}
min,max函数的用法和sum的用法稍微有点差异,简单起见可以认为min,max,sum都接受一个元组或者列表(还有其他?),然后返回这个元组或者列表其中的最小值,最大值或者相加总和。此外min和max还支持min(1,2,3)这样的形式,而sum不支持。
\begin{Verbatim}
>>> min((1,6,8,3,4))
1
>>> max([1,6,8,3,4])
8
>>> sum([1,6,8,3,4])
22
>>> min(1,6,8,3,4)
1
\end{Verbatim}



\subsection{位操作}
python支持位操作的,这里简单说一下:位左移操作<<,位与操作\&{},位或操作|,位异或操作\^{}。
\begin{Verbatim}
>>> x=0b0001
>>> bin(x << 2)
'0b100'
>>> bin(x | 0b010)
'0b11'
>>> bin(x & 0b1)
'0b1'
>>> bin(x ^ 0b101)
'0b100'
\end{Verbatim}




\subsection{math模块}
在\verb+from math import *+之后,可以直接用符号pi和e来引用圆周率和自然常数。此外math模块还提供了很多数学函数,比如:
\begin{description}
\item[sqrt] 开平方根函数,sqrt(x)。
\item[sin] 正弦函数,类似的还有cos,tan等,sin(x)。
\item[degrees] 将弧度转化为角度,三角函数默认输入的是弧度值。
\item[radians] 将角度转化位弧度,radians(30)。 
\item[log] 开对数,log(x,y),即$\log_y x$,y默认是e。
\item[exp] 指数函数,exp(x)。
\item[pow] 扩展了内置方法,现在支持float了。pow(x,y)
\end{description}

这里简单写个例子:
\begin{Verbatim}
>>> from math import *
>>> print(pi)
3.141592653589793
>>> print(sqrt(85))
9.219544457292887
>>> print(round(sin(radians(30)),1))#sin(30°)
0.5
\end{Verbatim}


\begin{Large}
更多内容请参见\href{http://docs.python.org/3.4/library/math.html}{官方文档}。
\end{Large}



\subsection{random模块}
random模块提供了一些函数来解决随机数问题。
\begin{description}
\item[random] random函数产生0到1之间的随机实数(包括0)。\\ random()->[0.0, 1.0)。
\item[uniform] uniform函数产生从a到b之间的随机实数(a,b的值指定,包括a。)。\\ uniform(a,b)->[a.0, b.0)。
\item[randint] randint函数产生从a到b之间的随机整数,包含a和b。\\ randint(a,b)->[a,b]
\item[choice] choice随机从一个列表或者字符串中取出一个元素。
\item[randrange] randrange函数产生从a到b之间的随机整数,步长为c(a,b,c的值指定,相当于choice(range(a,b,c))。整数之间就用randint函数吧,这里函数主要是针对range函数按照步长从而生成一些整数序列的情况。
\end{description}

下面是一个简单的例子:
\begin{xverbatim}[129]{py}
from random import *
print(random())
print(uniform(1,10))
print(randrange(1,6))
print(randint(1,10))
print(choice('abcdefghij'))
print(choice(['①','②','③']))
\end{xverbatim}


作为随机实数,所谓开始包含的那个临界值可能数学意义大于实际价值,你可以写一个类似下面的小脚本看一下,随机实数是很难随机到某个具体的数的。
\begin{tcbpython}[]
from random import *
i = 0
while True:
    x = uniform(0,2)
    if x == 0:
        print(i)
        break
    else:
        print(x)
        i += 1
\end{tcbpython}

从上一个例子我们看到,虽然我不确定随具体随机到某个实数的概率是不是永远也没有可能,但肯定很小很小。所以如果我们要解决某个问题,需要某个确定的概率的话还是用随机整数好一些。


\begin{Large}
更多内容请参见\href{http://docs.python.org/3.4/library/random.html}{官方文档}。
\end{Large}


\subsection{statistics模块}
这个模块python3.4才加入进来。

上面的那个例子这里稍作修改,使之成为一个骰子模拟器。其中\verb+i_list+这个列表收集多次实验中掷多少次骰子才遇到6的次数。
\begin{tcbpython}[]
from random import *
i_list = []
while len(i_list) < 100:
    i = 1
    while True:#一次实验
        x = randint(1,6)
        if x == 6:
            print('times:' , i)
            break
        else:
            print(x)
            i += 1
    i_list.append(i)

print(i_list)
from statistics import *
print(mean(i_list))#平均值
print(median(i_list))#中位数,去掉最高最低...
\end{tcbpython}

statistics模块中的mean函数接受一组数值列表,然后返回这组数值的平均值。而median函数返回的是统计学上所谓的中位数,你可以简单看作一组数字不断的去掉一个最高和最低,然后剩下来的一个或者两个(两个要取平均值)的数值的值。

\begin{Large}
更多内容请参见\href{https://docs.python.org/3/library/statistics.html}{官方文档}。
\end{Large}







\section{序列}
字符串,列表,元组(tuple,这里最好翻译成元组,因为里面的内容不一定是数值。)都是序列(sequence)的子类,所以序列的一些性质他们都具有,最好在这里一起讲方便理解记忆。

\subsection{len函数}
len函数返回序列所含元素的个数:
\begin{xverbatim}[129]{py}
string001='string'
list001=['a','b','c']
tuple001=(1,2,3,4)

for x in [string001,list001,tuple001]:
    print(len(x))
\end{xverbatim}


\subsection{调出某个值}
对于序列来说后面跟个方括号,然后加上序号(程序界的老规矩,从0开始计数。),那么调出对应位置的那个值。还以上面那个例子来说明。
\begin{xverbatim}[129]{py}
string001='string'
list001=['a','b','c']
tuple001=(1,2,3,4)

for x in [string001,list001,tuple001]:
    print(x[2])
\end{xverbatim}

\subsubsection{倒着来}
倒着来计数-1表示倒数第一个,-2表示倒数第二个。依次类推。
\begin{xverbatim}[129]{py}
string001='string'
list001=['a','b','c']
tuple001=(1,2,3,4)

for x in [string001,list001,tuple001]:
    print(x[-1],x[-2])
\end{xverbatim}

\subsection{调出多个值}
\label{sec:调出多个值}
前面不写表示从头开始,后面不写表示到达尾部。中间加个冒号的形式表示从那里到那里。这里\textbf{注意}后面那个元素是\uwave{不包括}进来,看来python区间的默认含义都是包头不包尾。这样如果你想要最后一个元素也进去,只有使用默认的不写形式了。
\begin{xverbatim}[129]{py}
string001='string'
list001=['a','b','c']
tuple001=(1,2,3,4)

for x in [string001,list001,tuple001]:
    print(x[1:3],x[-2:-1],x[:-1],x[1:],x[1:-1])
\end{xverbatim}
用数学半开半闭区间的定义来理解这里的包含关系还是很便捷的。
\begin{enumerate}
\item 首先是数学半开半闭区间,左元素和右元素都是之前叙述的对应的定位点。左元素包含右元素不包含。
\item 其次方向应该是从左到右,如果定义的区间是从右到左,那么将产生空值。
\item 如果区间超过,那么从左到右包含的所有元素就是结果,即\uwave{不会返回错误}。
\item 最后如果左右元素定位点相同,那么将产生空值,比如:\\
\verb+string001[2:-4]+,其中2和-4实际上是定位在同一个元素之上的。额外值得一提的列表插入操作,请参看列表的插入操作这一小节。\ref{sec:列表插入操作}
\end{enumerate}


\subsection{序列反转}
这是python最令人叹为观止的地方了,其他的语言可能对列表啊什么的反转要编写一个复杂的函数,我们python有一种令人感动的方法。

\begin{xverbatim}[129]{py}
string001='string'
list001=['a','b','c']
tuple001=(1,2,3,4)

for x in [string001,list001,tuple001]:
    print(x[::-1])
\end{xverbatim}

之前在range函数的介绍时提及序列的索引和range函数的参数设置很是类似,这是我们可以参考理解之,序列(列表,字符串等)的索引参数[start:end:step]和range函数的参数设置一样,第一个参数是起步值,第二个参数是结束值,第三个参数是步长。

然后range函数生成的迭代器对象同样接受这种索引参数语法,看上去更加的怪异了:
\begin{Verbatim}
>>> range(1,10,2)
range(1, 10, 2)
>>> range(1,10,2)[::-2]
range(9, -1, -4)

>>> list(range(1,10,2))
[1, 3, 5, 7, 9]
>>> list(range(1,10,2)[::-2])
[9, 5, 1]
\end{Verbatim}
我们可以看到对range函数进行切片操作之后返回的仍然是一个range对象,经过了一些修正。似乎这种切片操作和类的某个特殊方法有关,和python的slice对象有关。更进一步的讨论参见......



\subsection{序列的可更改性}
字符串不可以直接更改,但可以组合成为新的字符串;列表可以直接更改;元组不可以直接更改。



\subsection{序列的加法和减法}
两个字符串相加就是字符串拼接了。乘法就是加法的重复,所以一个字符串乘以一个数字就是自己和自己拼接了几次。列表还有元组和字符串一样大致情况类似。

\begin{xverbatim}[129]{py}
print('abc'+'def')
print('abc'*3)
print([1,2,3]+[4,5,6])
print((0,'a')*2)
\end{xverbatim}

序列的交集



\section{字符串}
python语言不像c语言字符和字符串是不分的,用单引号或者双引号包起来就表示一个字符串了。单引号和双引号的区别是一般用单引号,如果字符串里面有单引号,那么就使用双引号,这样单引号直接作为字符处理而不需要而外的转义处理——所谓转义处理和其他很多编程语言一样用\textbackslash 符号。比如要显示\verb+'+就输入\verb+\'+。

\subsection{三单引号和三双引号}
在单引号或者双引号的情况下,你可以使用\verb+\n+来换行,其中\textbackslash n表示换行。此外还可以使用三单引号'''或者三双引号"""来包围横跨多行的字符串,其中换行的意义就是换行,不需要似前面那样的处理。

\begin{Verbatim}
print('''\
这是一段测试文字
  this is a test line
      其中空白和    换行都所见所得式的保留。''')
\end{Verbatim}


\subsection{find方法}
字符串的find方法可用来查找某个子字符串,没有找到返回-1,找到了返回字符串的偏移量。用法就是:\verb+s.find('d')+。


\subsection{replace方法}
字符串的replace方法进行替换操作,接受两个参数:第一个参数是待匹配的子字符串,第二个参数是要替换成为的样子。
\begin{Verbatim}
>>> print('a b 11 de'.replace('de','ding'))
a b 11 ding
>>> print('1,1,5,4,1,6'.replace('1','replaced'))
replaced,replaced,5,4,replaced,6
\end{Verbatim}




\subsection{upper方法}
将字符串转换成大写形式。
\begin{Verbatim}
>>> str='str'
>>> str.upper()
'STR'
\end{Verbatim}


\subsection{isdigit方法}
类似的还有isalpha方法,测试是不是数字或字母。值得注意的是就算是字母组成的语句,中间有空间也会返回False。

\subsection{split方法}
字符串的split方法可以将字符串比如有空格或者逗号等分隔符分割而成,可以将其分割成子字符串列表。默认是空格是分隔符。
\begin{Verbatim}
>>> string='a=1,b=2,c=3'
>>> string.split(',')
['a=1', 'b=2', 'c=3']
\end{Verbatim}

\subsubsection{splitline方法}
把一个字符串按照行分开。这个可以用上面的split方法然后接受\verb+\n+参数来实现,所不同的是splitline方法不需要接受参数:
\begin{Verbatim}
>>> string
'this is line one\nthis is line two\nthis is line three'
>>> string.splitlines()
['this is line one', 'this is line two', 'this is line three']
>>> string.split('\n')
['this is line one', 'this is line two', 'this is line three']
\end{Verbatim}




\subsection{join方法}
字符串的join方法非常有用,严格来说它接受一个迭代器参数,不过最常见的是列表。将列表中的多个字符串连接起来,我们看到他采用了一种非常优雅的方式,就是只有两个字符串之间才插入某个字符,这正是我们所需要的。具体例子如下所示:
\begin{Verbatim}
>>> list001=['a','b','c']
>>> "".join(list001)
'abc'
>>> ','.join(list001)
'a,b,c'
\end{Verbatim}


\subsection{strip方法}

\subsubsection{rstrip方法}
字符串右边的空格都删除。换行符也会被删除掉。

\subsubsection{lstrip方法}
类似rstrip方法,字符串左边的空格都删除。换行符也会被删除掉。


\subsection{format方法}
字符串的format方法方便对字符串内的一些变量进行替换操作,其中花括号不带数字跟format方法里面所有的替换量,带数字0表示第一个替换量,后面类推。此外还可以直接用确定的名字引用。
\begin{Verbatim}
>>> print('1+1={0},2+2={1}'.format(1+1,2+2))
1+1=2,2+2=4
>>> print('my name is {name}'.format(name='Jim T Kirk'))
my name is Jim T Kirk
\end{Verbatim}

\subsection{转义和不转义}
\verb+\n    \t  +这是一般常用的转义字符,换行和制表。此外还有\verb+\\+输出\textbackslash 符号。

如果输出字符串不想转义那么使用如下格式:
\begin{Verbatim}
>>> print(r'\t \n \test')
\t \n \test
\end{Verbatim}

\subsection{count方法}
统计字符串中某个字符或某一连续的子字符串出现的次数。
\begin{Verbatim}
>>> string = 'this is a test line.'
>>> string.count('this')
1
>>> string.count('t')
3
\end{Verbatim}






\section{列表}
方括号包含几个元素就是列表。


\subsection{列表的插入操作}
\label{sec:列表插入操作}
字符串和数组都不可以直接更改所以不存在这个问题,列表可以。其中列表还可以以一种定位在相同元素的区间的方法来实现插入操作,这个和之前理解的区间多少有点违和,不过考虑到定位在相同元素的区间本来就概念模糊,所以在这里就看作特例,视作在这个\uwave{定位点相同元素之前}插入吧。
\begin{xverbatim}[129]{py}
list001=['one','two','three']
list001[1:-2]=['four','five']
print(list001)
\end{xverbatim}

除了序列中的一些继承的操作之外,列表还有很多方法,实际上这还算少的(如果你见识了lisp中各种列表操作)。因为列表这个数据结构可以直接修改相当灵活,下面我打算将我学lisp语言中接触到的一些列表操作对应过来一一说明之:

extend方法似乎和列表之间的加法重合了,比如\\list001.extend([4,5,6])就和list001=list001+[4,5,6]是一致的,而且用加法表示还可以自由选择是不是覆盖原定义,这实际上更加自由。所以extend方法略过。

insert方法也就是列表的插入操作,这个前面关于列表的插入实现方法说过一种了,所以insert方法也略过。

\subsection{append方法}
python的append方法和lisp中的append还是有点差异的,python的append就是在最后面加\textbf{一个元素},如果你append一个列表那么这一个列表整体作为一个元素。lisp的append函数和python的extend方法类似,接受一个列表。

其次append是list类中的一个方法,也就是list001.append这样的形式,也就是永久的改变了某个列表实例的值了。

记住,append方法是原处修改列表,是没有返回值的。

\subsection{reverse方法}
reverse方法不接受任何参数,直接将一个列表\uwave{永久性地}翻转过来。

\subsection{sort方法}
也就是排序,\uwave{永久性}改变列表。默认是递增排序,可以用\textbf{reverse=True}来调成递减排序。

默认的递增排序顺序如果是数字那么意思是数字越来越大,如果是字符那么(似乎)是按照ACSII码编号递增来排序的。如果列表一些是数字一些是字符会报错。
\begin{Verbatim}
>>> list = ['a','ab','A','123','124','5']
>>> list.sort()
>>> list
['123', '124', '5', 'A', 'a', 'ab']
\end{Verbatim}

sort方法很重要的一个可选参数\textbf{key=function},这个function函数就是你定义的函数(或者在这里直接使用lambda语句。),这个函数只接受一个参数,就是排序方法(在迭代列表时)接受的当前的那个元素。下面给出一段代码,其中tostr函数将接受的对象返回为字符,这样就不会出错了。
\begin{xverbatim}[129]{py}
def tostr(item):
    return str(item)

list001 = ['a','ab','A',123,124,5]

list001.sort(key=tostr)

print(list001)
\end{xverbatim}

\subsubsection{sort函数}
sorted函数在这里和列表的sort方法最大的区别是它返回的是\uwave{一个新的列表}而不是原处修改。其次sorted函数的第一个参数严格来说是所谓的可迭代对象,也就是说它还可以接受除了列表之外的比如元组字典等的待排序对象。至于用法他们两个差别不大。

\begin{Verbatim}
>>> sorted((1,156,7,5))
[1, 5, 7, 156]
>>> sorted({'andy':5,'Andy':1,'black':9,'Black':55},key=str.lower)
['Andy', 'andy', 'black', 'Black']
\end{Verbatim}

上面第二个例子调用了\textbf{str.lower}函数,从而将接受的item,这里比如说'Andy',转化为andy,然后参与排序。也就成了对英文字母大小写不敏感的排序方式了。

\paragraph{字典按值排序}
同样类似的有字典按值排序的方法\footnote{\href{http://www.cnpythoner.com/post/266.html}{参考网站}}:

\begin{Verbatim}
>>> sorted({'andy':5,'Andy':1,'black':9,'Black':55}.items(),key=lambda i: i[1])
[('Andy', 1), ('andy', 5), ('black', 9), ('Black', 55)]
\end{Verbatim}

这个例子先用字典的items方法处理反馈(key,value)对的可迭代对象,然后用后面的lambda方法返回具体接受item的值,从而根据值来排序。


\paragraph{中文排序}

\begin{tcbpython}[]
list001=['张三','李四','王二','麻子','李二','李一']
def zhsort(lst):
    from pypinyin import  lazy_pinyin
    pinyin=[lazy_pinyin(lst[i]) for i in range(len(lst))]
    lst0=[(a,b) for (a,b) in zip(lst,pinyin)]
    lst1= sorted(lst0, key=lambda d:d[1])
#    print(lst1)
    return [x[0] for x in lst1]
print(zhsort(list001))
\end{tcbpython}
\begin{Verbatim}
['李二', '李四', '李一', '麻子', '王二', '张三']
\end{Verbatim}



这是一个简单的中文姓名排序函数,整个函数的思路就是用\href{https://github.com/mozillazg/python-pinyin}{pypinyin}(一个第三方模块),将中文姓名的拼音对应出来,然后组成一个列表,然后根据拼音对这个组合列表排序,然后生成目标列表。


\subsection{删除某个元素}
\begin{itemize}
\item 赋空列表值,相当于所有元素都删除了。 
\item pop方法:接受一个参数,就是列表元素的定位值,然后那个元素就删除了,方法并返回那个元素的值。如果不接受参数默认是删除最后一个元素。
\item remove方法:移除第一个相同的元素,如果没有返回相同的元素,返回错误。
\item del函数:删除列表中的某个元素。
\end{itemize}

\begin{Verbatim}
>>> list001=['a','b','c','d','e']
>>> list001.pop(2)
'c'
>>> list001
['a', 'b', 'd', 'e']
>>> list001.pop()
'e'
>>> list001
['a', 'b', 'd']
>>> list001.remove('a')
>>> list001
['b', 'd']
>>> del list001[1]
>>> list001
['b']
\end{Verbatim}


\subsection{count方法}
统计某个元素出现的次数。
\begin{Verbatim}
>>> list001=[1,'a',100,1,1,1]
>>> list001.count(1)
4
\end{Verbatim}


\subsection{index方法}
index方法返回某个相同元素的偏移值。
\begin{Verbatim}
>>> list001=[1,'a',100]
>>> list001.index('a')
1
\end{Verbatim}

\subsection{列表元素去重}
我写了一个列表元素去重的函数,读者可以体会一下,其中涉及到递归思想\ref{sec:递归函式},就是很直白的我很懒惰,我把一个重复元素删除了,我就把作业上交给主管了,我说我的那部分工作干完了,其他的你接着干吧。
\begin{xverbatim}[129]{py}
def removeduplicate(list):
    newlist = list.copy()
    for j in newlist:
        for index in range(newlist.index(j)+1,len(newlist)-1):
            if j == newlist[index]:
                del newlist[index]
                return removeduplicate(newlist)
    return newlist

list001=[1,2,3,1,2,4,4,5,5,5,7]
print(removeduplicate(list001))
\end{xverbatim}
  


\subsection{列表元素的替换}
lisp语言中有subst函数,是substitute的缩写。作用于整个列表,列表中所有出现的某个元素都要被另一个元素替换掉。

由于我现在对如何修改python语言内置类还毫无头绪,只好简单写这么一个函数了。
\begin{xverbatim}[129]{py}
def subst(list001,element001,element002):
    try:
        list001.index(element001)
    except ValueError:
        return list001
    else:
        n=list001.index(element001)
        del list001[n]
        list001[n:n]=[element002]
        return subst(list001,element001,element002)

print(subst([1,'a',3,[4,5]],[4,5],'b'))
print(subst([1,1,5,4,1,6],1,'replaced'))
\end{xverbatim}
这个subst函数接受三个参数,表示接受的列表,要替换的元素和替换成为的元素。这里使用的程序结构是try...except...else...语句。其中try来侦测是不是有错误,其中index方法是看那个要替换的元素存不存在,由于不存在这个函数将产生一个\textbf{ValueError}错误,所以用except来接著。既然没有要替换的元素了,那么返回原列表即可,程序中止。

else语句接著没有错误的时候你要执行的操作,先index再删掉这个元素,再在之前插入那个元素,然后使用了递归算法,调用函数自身。

\subsection{列表解析}
\label{sec:列表解析}
lisp中的mapcar函数有这个功用,python中的map函数基本上和它情况类似。

我们先来看lisp中的情况:
\begin{xverbatim}[129]{lisp}
(defun square (n) (* n n))
(format t "~&~s" (mapcar #'square '(1 2 3 4 5)) )
\end{xverbatim}

再来看python中的情况:
\begin{xverbatim}[129]{py}
def square(n):
    return n*n
    
print(list(map(square,[1,2,3,4,5])))
print([square(x) for x in [1,2,3,4,5]])
\end{xverbatim}
map函数将某个应数应用于某个列表的元素中并生成一个map对象,需要外面加上list函数才能生成列表形式。第二种方式更有python风格,是推荐使用的列表解析方法。

在python中推荐多使用迭代操作和如上的列表解析风格,因为python中的迭代操作是直接用c语言实现的。

\subsubsection{列表解析加上过滤条件}
for语句后面可以跟一个if子句表示过滤条件,看下面的例子来理解吧:
\begin{Verbatim}
>>> [s*2 for s in ['hello','abc','final','help'] if s[0] == 'h']
['hellohello', 'helphelp']
\end{Verbatim}

这个例子的意思是列表解析,找到的元素进行乘以2的操作,其中过滤条件为字符是h字母开头的,也就是后面if表达式不为真的元素都被过滤掉了。

\subsubsection{排列组合赋值}
我想到列表解释的时候如果解析加赋值会如何呢?当然如果同一个变量逐个赋值这毫无意义,但如果我们将其变成一个字符串,从而形成一种延迟赋值语句这又会如何呢?
\begin{Verbatim}
>>> ["x,y={0},{1};".format(a,b) for a in [1,2,3] for b in [4,5]]
['x,y=1,4;', 'x,y=1,5;', 'x,y=2,4;', 'x,y=2,5;', 'x,y=3,4;', 'x,y=3,5;']
>>> exec(["x,y={0},{1};".format(a,b) for a in [1,2,3] for b in [4,5]][0])
>>> print(x,y)
1 4
>>> exec(["x,y={0},{1};".format(a,b) for a in [[1,2],2,3] for b in [4,5]][0])
>>> x
[1, 2]
\end{Verbatim}
我们看到通过exec来执行之后x和y都赋给了相应的值,这似乎有点意思,因为我们可以通过itertools(排列组合等等技术)来形成一种批量赋值和进行某些操作的机制。

然后我们看到将赋值改为列表也是可行的,这样我们可以猜测一般对象赋值都是可行的。

但是字符串赋值就出现问题了:
\begin{Verbatim}
>>> ["x,y={0},{1};".format(a,b) for a in ['a',2,3] for b in [4,5]]
['x,y=a,4;', 'x,y=a,5;', 'x,y=2,4;', 'x,y=2,5;', 'x,y=3,4;', 'x,y=3,5;']
\end{Verbatim}
我们可以看到a进入之后就成了变量名了,然后exec执行就会出错。如果你给a赋一个值,这样就不会出错了。(可能会有某种手段处理来实现字符串的赋值?)

不过这里我们看到另外一种有趣的现象,那就是我们可以通过字符串组合来生成新的变量名。
\begin{Verbatim}
 exec(["x={0};".format(str(a)+str(b)) for a in ['x'] for b in [0,1,2]][0])
\end{Verbatim}
比如如下面这样就将执行\verb+x=x0+这样的表达式,因为x0没有定义将会出错。


\subsubsection{批量赋值}
下面是一个批量赋值函数的粗糙例子,作为自动生成一些变量名和批量赋值这些变量名似乎没有什么用处,不过也许以后可能对于某些问题会有用处的,先暂时放在这里。
\begin{Verbatim}
def multiassign(n):
    for i in ["global {0};{0} = {1};".format(str(a)+str(b),i) 
    for a in ['x'] for (b,i) in [(i,i) for i in list(range(n))]]:
        exec(i)

if __name__ == '__main__':
    multiassign(100)
\end{Verbatim}


\subsubsection{元组的生成}
>>> [(x,x**2) for x in range(5)]
[(0, 0), (1, 1), (2, 4), (3, 9), (4, 16)]



\subsection{for语句中列表可变的影响}
\begin{tcbpython}[]
lst = [1,2,3,4,5]
index = 0
for x in lst:
    lst.pop(index)
    print(x)
\end{tcbpython}

\begin{Verbatim}
1
3
5
>>> 
\end{Verbatim}










\section{字典}
与列表一样字典是可变的,可以像列表一样引用然后原处修改,del语句也适用。

\subsection{创建字典}
字典是一种映射,并没有从左到右的顺序,只是简单地将键映射到值。字典的声明格式如下:
\begin{Verbatim}
dict001={'name':'tom','height':'180','color':'red'}
dict001['name']
\end{Verbatim}

或者创建一个空字典,然后一边赋值一边创建对应的键:
\begin{Verbatim}
dict002={}
dict002['name']='bob'
dict002['height']=195
\end{Verbatim}

所以对字典内不存在的键赋值是可行的。

\subsubsection{根据列表创建字典}
如果是[['a',1],['b',2],['c',3]]这样的形式,那么直接用dict函数处理就变成字典了,如果是['a','b','c']和[1,2,3]这样的形式那么需要用zip函数处理一下,然后用dict函数处理一次就变成字典了:
\begin{Verbatim}
>>> lst
[['a', 1], ['b', 2], ['c', 3]]
>>> dict001=dict(lst)
>>> dict001
{'a': 1, 'b': 2, 'c': 3}
\end{Verbatim}

zip函数的例子本文多有涉及,这里不会赘述了。



\subsection{字典里面有字典}
和列表的不同就在于字典的索引方式是根据“键”来的。
\begin{Verbatim}
dict003={'name':{'first':'bob','second':'smith'}}
dict003['name']['first']
\end{Verbatim}

\subsection{字典遍历操作}
字典特定顺序的遍历操作的通用做法就是通过字典的keys方法收集键的列表,然后用列表的sort方法处理之后用for语句遍历,如下所示:
\begin{Verbatim}
dict={'a':1,'c':2,'b':3}
dictkeys=list(dict.keys())
dictkeys.sort()
for key in dictkeys:
    print(key,'->',dict[key])
\end{Verbatim}

\emph{警告}:上面的例子可能对python早期版本并不使用,因为python中一大规则是对对象的原处修改的函数并没有返回值。上面的语句只是到了python3后期才能适用,保险起见,推荐使用sorted函数,sorted函数是默认对字典的键进行排序并返回键的值组成的列表。
\begin{Verbatim}
dict={'a':1,'c':3,'b':2}
>>> for key in sorted(dict):
...   print(key,'->',dict[key])
... 
a -> 1
b -> 2
c -> 3
\end{Verbatim}


如果你对字典遍历的顺序没有要求,那么就可以简单的这样处理:
\begin{Verbatim}
>>> for key in dict:
...     print(key,'->',dict[key])
... 
c -> 2
a -> 1
b -> 3
\end{Verbatim}


\subsubsection{keys方法}
收集键值,返回\uwave{可迭代对象}。

\subsubsection{values方法}
和keys方法类似,收集的值,返回\uwave{可迭代对象}。
\begin{Verbatim}
>>> dict001.values()
dict_values([3, 1, 2])
>>> list(dict001.values())
[3, 1, 2]
\end{Verbatim}

\subsubsection{items方法}
和keys和values方法类似,不同的是返回的是(key,value)对的\uwave{可迭代对象}。
\begin{Verbatim}
>>> dict001.items()
dict_items([('c', 3), ('a', 1), ('b', 2)])
>>> list(dict001.items())
[('c', 3), ('a', 1), ('b', 2)]
\end{Verbatim}



\subsection{字典的in语句}
可以看到in语句只针对字典的键,不针对字典的值。
\begin{Verbatim}
>>> dict001={'a':1,'b':2,'c':3}
>>> 2 in dict001
False
>>> 'b' in dict001
True
\end{Verbatim}

\subsection{字典对象的get方法}
get方法是去找某个键的值,为什么不直接引用呢,get方法的好处就是某个键不存在也不会出错。
\begin{Verbatim}
>>> dict001={'a':1,'b':2,'c':3}
>>> dict001.get('b')
2
>>> dict001.get('e')
\end{Verbatim}

\subsection{update方法}
感觉字典就是一个小型数据库,update方法将另外一个字典里面的键和值覆盖进之前的字典中去,称之为更新,没有的加上,有的覆盖。
\begin{Verbatim}
>>> dict001={'a':1,'b':2,'c':3}
>>> dict002={'e':4,'a':5}
>>> dict001.update(dict002)
>>> dict001
{'c': 3, 'a': 5, 'e': 4, 'b': 2}
\end{Verbatim}

\subsection{pop方法}
pop方法类似列表的pop方法,不同引用的是键,而不是偏移地址,这个就不多说了。



\subsection{字典解析}
这种字典解析方式还是很好理解的。
\begin{Verbatim}
>>> dict001={x:x**2 for x in [1,2,3,4]}
>>> dict001
{1: 1, 2: 4, 3: 9, 4: 16}
\end{Verbatim}

\subsubsection{zip函数创建字典}
可以利用zip函数来通过两个可迭代对象平行合成一个配对元素的可迭代对象,然后用dict函数将其变成字典对象。具体的理解请参看深入理解python3的迭代这一章\ref{sec:深入理解python3的迭代}。
\begin{Verbatim}
>>> dict001=zip(['a','b','c'],[1,2,3])
>>> dict001
<zip object at 0xb7055eac>
>>> dict001=dict(dict001)
>>> dict001
{'c': 3, 'b': 2, 'a': 1}
\end{Verbatim}




\section{集合}
python实现了数学上的无序不重复元素的集合概念,还记得前面列表中我们定义了一个removeduplicate函数来移除列表中的重复元素,well,现在我们可以直接将列表(更确切的表述是可迭代对象)进行集合解析即可:

\begin{Verbatim}
>>> list001=[1,2,3,1,2,4,4,5,5,5,7]
>>> {x for x in list001}
{1, 2, 3, 4, 5, 7}
>>> set(list001)
{1, 2, 3, 4, 5, 7}
\end{Verbatim}
用集合解析的形式表示出来就是强调set命令可以将任何可迭代对象都变成集合类型。当然如果我们希望继续使用列表的话使用list命令强制类型转换为列表类型即可,不过如果我们在应用中确实一致需要元素不重复这一特性,就可以考虑直接使用集合作为主数据操作类型。

集合也是可迭代对象。关于可迭代对象可以进行的列表解析操作等等就不啰嗦了。下面介绍集合的一些操作。

\subsection{集合添加元素}
值得一提的是如果想创建一个空的集合, 需要用set命令,用花括号{}系统会认为你创建的是空字典。然后我们看到用集合的add方法添加,那些重复的元素是添加不进来的。

\emph{警告}:值得一提的是集合只能包括不可变类型,因此列表和字典不能作为集合内部的元素。元组不可变,所以可以加进去。还有\uwave{集合也是不可以包括进去的},觉得这点好逊啊,数学里面的集合概念能够包含集合那是基本的特性啊,感觉这点不修正好还是用列表方便些。

\begin{Verbatim}
>>> set001=set()
>>> set001.add(1)
>>> set001
{1}
>>> set001.add(2)
>>> set001
{1, 2}
>>> set001.add(1)
>>> set001
{1, 2}
\end{Verbatim}

或者使用update方法一次更新多个元素:
\begin{Verbatim}
>>> set001=set('a')
>>> set001.update('a','b','c')
>>> set001
{'b', 'a', 'c'}
\end{Verbatim}


\subsection{集合去掉某个元素}
有两个集合对象的方法可以用于去掉集合中的某个元素,discard方法和remove方法,其中discard方法如果删除集合中没有的元素那么什么都不会发生,而remove方法如果删除某个不存在的元素那么会产生KeyError。

\begin{Verbatim}
>>> set001=set('hello')
>>> set001.discard('h')
>>> set001
{'e', 'o', 'l'}
>>> set001.discard('l')
>>> set001
{'e', 'o'}
\end{Verbatim}

remove方法与之类似就不做演示了。

\subsection{两个集合之间的关系}

\subsubsection{子集判断}
集合对象有一个issubset方法用于判断这个集合是不是那个集合的子集。
\begin{Verbatim}
>>> set001=set(['a','b'])
>>> set002=set(['a','b','c'])
>>> set001.issubset(set002)
True
\end{Verbatim}

还有更加简便的方式比较两个集合之间的关系,那就是>,<,>=,<=,==这样的判断都是适用的。也就是set001是set002的子集,它的元素set002都包含,那么set001<=set002,然后真子集的概念就是set001<set002即不等于即可。


\subsection{两个集合之间的操作}
下面的例子演示的是两个集合之间的交集:\verb+ & +;并集:\verb+ | +;差集:\verb+ - +
\begin{Verbatim}
>>> set001=set('hello')
>>> set002=set('hao')
>>> set001 & set002 #交集
{'o', 'h'}
>>> set001 | set002 #并集
{'h', 'l', 'a', 'e', 'o'}
>>> set001 - set002 #差集
{'e', 'l'}
\end{Verbatim}


类似的集合对象还有intersection方法,union方法,difference方法:
\begin{Verbatim}
>>> set001=set('hello')
>>> set002=set('hao')
>>> set001.intersection(set002) #交集
{'h', 'o'}
>>> set001.union(set002) #并集
{'e', 'a', 'h', 'o', 'l'}
>>> set001.difference(set002) #差集
{'e', 'l'}
\end{Verbatim}


\subsection{clear方法}
将一个集合清空。

\subsection{copy方法}
类似列表的copy方法,制作一个集合copy备份然后赋值给其他变量。

\subsection{pop方法}
无序弹出集合中的一个元素,直到没有然后返回KeyError错误。



\section{元组}
圆括号包含几个元素就是元组(tuple)。元组和列表的不同在于元组是不可改变。元组也是从属于序列对象的,元组的很多方法之前都讲了。而且元组在使用上和列表极其接近,有很多内容这里也略过了。

值得一提的是如果输入的时候写的是\textit{x,y}这样的形式,实际上表达式就加上括号了,也就是一个元组了\textit{(x,y)}。





\section{文件}

\subsection{写文件}
对文件的操作首先需要用open函数创建一个文件对象,简单的理解就是把相应的接口搭接好。文件对象的write方法进行对某个文件的写操作,最后需要调用close方法写的内容才真的写进去了。

\begin{Verbatim}
file001 = open('test.txt','w')
file001.write('hello world1\n')
file001.write('hello world2\n')
file001.close()
\end{Verbatim}

如果你们了解C语言的文件操作,在这里会为python语言的简单便捷赞叹不已。就是这样三句话:创建一个文件对象,然后调用这个文件对象的wirte方法写入一些内容,然后用close方法关闭这个文件即可。


\subsection{读文件}
一般的用法就是用open函数创建一个文件对象,然后用read方法调用文件的内容。最后记得用close关闭文件。
\begin{Verbatim}
file001 = open('test.txt')
filetext=file001.read()
print(filetext)
file001.close()
\end{Verbatim}

此外还有readline方法是一行一行的读取某文件的内容。


\subsection{open函数的处理模式}
open函数的处理模式如下:
\begin{description}
\item['r'] 默认值,read,读文件。
\item['w'] wirte,写文件,如果文件不存在会创建文件,如果文件已存在,文件原内容会清空。
\item['a'] append,附加内容,也就是后面用write方法内容会附加在原文件之后。
\item['b'] 处理模式设置的\uwave{附加}选项,'b'不能单独存在,要和上面三个基本模式进行组合,比如'rb'等,意思是二进制数据格式读。
\item['+'] 处理模式设置的\uwave{附加}选项,同样'+'不能单独存在,要和上面三个基本模式进行组合,比如'r+'等,+是updating更新的意思,也就是既可以读也可以写,那么'r+','w+','a+'还有什么区别呢?区别就是'r+'不具有文件创建功能,如果文件不存在会报错,然后'r+'不会清空文件,如果'r+'不清空文件用write方法情况会有点复杂;而'w+'具有文件创建功能,然后'w+'的write方法内容都是重新开始的;而'a+'的write方法内容是附加在原文件上的,然后'a+'也有文件创建功能。
\end{description}



\subsection{用with语句打开文件}
类似之前的例子我们可以用with语句来打开文件,这样就不用close方法来关闭文件了。然后with语句来提供了类似try语句的功能可以自动应对打开文件时的一些异常情况。
\begin{Verbatim}
with open('test.txt','w') as file001:
    file001.write('hello world1\n')
    file001.write('hello world2\n')

with open('test.txt','r') as file001:
    filetext=file001.read()
    print(filetext)
\end{Verbatim}




\subsection{除字符串外其他类型的读取}
文本里面存放的都是字符串类型,也就是写入文件需要用str函数强行将其他类型转变成字符串类型,而读取进来想要进行一些操作则需要将字符串类型转变回去。比如用int或者float等,不过列表和字典的转变则需要eval函数。

eval这个函数严格来讲作用倒不是为了进行上面说的类型转换的,它就是一个内置函数,一个字符串类型python代码用eval函数处理了之后就能转变为可执行代码。
\begin{Verbatim}
>>> eval('1+1')
2
>>> eval('[1,2,3]')
[1, 2, 3]
>>> eval("{'a':1,'b':2,'c':3}")
{'c': 3, 'b': 2, 'a': 1}
\end{Verbatim}

推荐使用pickle模块来处理其他类型的文件读写问题,相对来说更简单更安全。请参看pickle模块这一小节\ref{sec:pickle模块}。







\chapter{类}
在python中一切皆对象。前面学的那些操作对象都是python程序语言自己内部定义的对象(Object),而接下来介绍的类的语法除了更好的理解之前的那些对象之外,再就是可以创造自己的操作对象。一般面向对象(OOP)编程的基本概念这里不重复说明了,如有不明请读者自己随便搜索一篇网页阅读下即可。

\section{python中类的结构}
python中的类就好像树叶,所有的类就构成了一棵树,而python中超类,子类,实例的重载或继承关系等就是由一种搜索机制实现的:
\begin{fig}{类搜索结构图}
\label{fig:类搜索结构图}
\end{fig}
python首先搜索self有没有这个属性或者方法,如果没有,就向上搜索。比如说实例l1没有,就向上搜索C1,C1没有就向上搜索C2或C3等。

实例继承了创造他的类的属性,创造他的类上面可能还有更上层的超类,类似的概念还有子类,表示这个 类在树形层次中比较低。

well,简单来说类的结构和搜索机制就是这样的,很好地模拟了真实世界知识的树形层次结构。

上面那副图实际编写的代码如下:
\begin{Verbatim}
class C2: ...
class C3: ...
class C1(C2,C3): ...
l1=C1()
l2=C1()
\end{Verbatim}
其中class语句是创造类,而C1继承自C2和C3,这是多重继承,从左到右是内部的搜索顺序(会影响重载)。l1和l2是根据类C1创造的两个实例。

对于初次接触类这个概念的读者并不指望他们马上就弄懂类这个概念,这个概念倒并一定要涉及很多哲学的纯思考的东西,也可以看作一种编程经验或技术的总结。多接触也许对类的学习更重要,而不是纯哲学抽象概念的讨论,毕竟类这个东西创造出来就是为了更好地描述现实世界的。

最后别人编写的很多模块就是一堆类,你就是要根据这些类来根据自己的情况情况编写自己的子类,为了更好地利用前人的成果,或者你的成果更好地让别人快速使用和上手,那么你需要好好掌握类这个工具。

\section{类的最基础知识}
\subsection{类的创建}
\begin{Verbatim}
class MyClass:
    something
\end{Verbatim}
类的创建语法如上所示,然后你需要想一个好一点的类名。类名规范的写法是首字母大写,这样好和其他变量有所区分。

\subsection{根据类创建实例}
按照如下语句格式就根据MyClass类创建了一个实例myclass001。
\begin{Verbatim}
myclass001=MyClass()
\end{Verbatim}

\subsection{类的属性}
\begin{Verbatim}
>>> class MyClass:
...  name='myclass'
... 
>>> myclass001=MyClass()
>>> myclass001.name
'myclass'
>>> MyClass.name
'myclass'
>>> myclass001.name='myclass001'
>>> myclass001.name
'myclass001'
>>> MyClass.name
'myclass'
\end{Verbatim}
如上代码所示,我们首先创建了一个类,这个类加上了一个name属性,然后创建了一个实例myclass001,然后这个实例和这个类都有了name属性。然后我们通过实例加上点加上name的这种格式引用了这个实例的name属性,并将其值做了修改。

这个例子简单演示了类的创建,属性添加,实例创建,多态等核心概念。后面类的继承等概念都和这些大同小异了。


\subsection{类的方法}
类的方法就是类似上面类的属性一样加上def语句来定义一个函数,只是函数在类里面我们一般称之为方法。这里演示一个例子,读者看一下就明白了。
\begin{Verbatim}
>>> class MyClass:
...  name='myclass'
...  def double(self):
...   self.name=self.name*2
...   print(self.name)
... 
>>> myclass001=MyClass()
>>> myclass001.name
'myclass'
>>> myclass001.double()
myclassmyclass
>>> myclass001.name
'myclassmyclass'
\end{Verbatim}

这里需要说明的是在类的定义结构里面,self代表着类自身,self.name代表着对自身name属性的引用。然后实例在调用自身的这个方法时用的是myclass001.double()这样的结构,这里double函数实际上接受的第一个参数就是自身,也就是myclass001,而不是无参数函数。所以类里面的方法(被外部引用的话)至少有一个参数self。





\section{类的继承}
实例虽然说是根据类创建出来的,但实际上实例和类也是一种继承关系,实例继承自类,而类和类的继承关系也与之类似,只是语法稍有不同。下面我们来看这个例子:
\begin{xverbatim}[129]{py}
class Hero():
    def addlevel(self):
        self.level=self.level+1
        self.hp=self.hp+self.addhp

class Garen(Hero):
    level=1
    hp=455
    addhp=96

garen001=Garen()
for i in range(6):
    print('级别:',garen001.level,'生命值:' ,garen001.hp)
    garen001.addlevel()
\end{xverbatim}

\begin{fig}[0.5]{类的继承示例}
\caption{类的继承示例}
\label{fig:类的继承示例}
\end{fig}

这里就简单的两个类,盖伦Garen类是继承自Hero类的,实例garen001是继承自Garen类的,这样garen001也有了addlevel方法,就是将自己的level属性加一,同时hp生命值也加上一定的值,整个过程还是很直观的。

\subsection{super}
\begin{flushright}
\begin{notecard}[red!30]{16em}
警报:这一小节super涉及到的知识稍微有点高深,请类大体基本知识都熟悉之后再来看这个。 
\end{notecard}
\end{flushright}

super是python3新加入的特性,按照官方文档,有两种用法:

第一种是如果是单继承的类的系统,super()这种形式就直接表示父类的意思。然后用super.什么什么的来引用父类的某个变量或方法,值得一提的是原父类的self参量会默认加进去了,详细请看下面的调试例子。

第二种是多重继承的,搜索顺序和多重继承的搜索顺序相同,也就是从左到右。请注意调试下面的例子,如果调用c.d就会返回错误,说明调用的是类A的构造函数。

\begin{xverbatim}[129]{py}

class A():
    def __init__(self,a):
        self.a=a

    @staticmethod
    def fun():
        print('fun')

    def fun2(self,what):
        print('fun',what)

class B():
    def __init__(self):
        self.d=5
    b=2
    def fun3(self):
        print('fun3')

class C(A,B):
    def __init__(self):
        super().__init__(3)
        super().fun()
        super().fun2('what')
        super().fun3()
        print(super().b)

c=C()
print(c.a,c.b)
\end{xverbatim}
fun的输出没有产生错误是因为让其成了静态函数,所以可以以fun这样的空参数形式调用。其中的self参数不会干扰它。但是通常情况下不是静态函数的话,写的函数通常就要加上self,而super()这种调用形式默认第一个参数就是self。

fun3也能被调用是因为多重继承的机制在这里,所以它会逐个找父类。然后c.d会出错,因为这里初始化是用的A类的构造函数。


\section{类的内置方法}
如果构建一个类,就使用pass语句,什么都不做,python还是会为这个类自动创建一些属性或者方法。
\begin{Verbatim}
>>> class TestClass:
...  pass
... 
>>> dir(TestClass)
['__class__', '__delattr__', '__dict__', '__dir__', '__doc__',
 '__eq__', '__format__', '__ge__', '__getattribute__',
  '__gt__',  '__hash__', '__init__', '__le__', '__lt__',
   '__module__', '__ne__', '__new__', '__reduce__', 
   '__reduce_ex__', '__repr__', '__setattr__', 
   '__sizeof__', '__str__', '__subclasshook__',
    '__weakref__']
\end{Verbatim}

这些变量名字前后都加上双下划线是给python这个语言的设计者用的,一般应用程序开发者还是不要这么做。

这些内置方法用户同样也是可以重定义他们从来覆盖掉原来的定义,其中特别值得一讲的就是\verb+__init__+方法或者称之为构造函数。

\subsection{\_\_init\_\_{}方法}
\verb+__init__+方法对应的就是该类创建实例的时候的构造函数。比如:
\begin{Verbatim}
>>> class Point:
...  def __init__(self,x,y):
...   self.x=x
...   self.y=y
... 
>>> point001=Point(5,4)
>>> point001.x
5
>>> point001.y
4
\end{Verbatim}
这个例子重载了\verb+__init__+函数,然后让他接受三个参数,self等下要创建的实例,x,还有y通过下面的语句给这个待创建的实例的属性x和y赋了值。


\subsection{self意味着什么}
self在类中是一个很重要的概念,当类的结构层次较简单时还容易看出来, 当类的层次结构很复杂之后,你可能会弄糊涂。\uwave{self就是指现在引用的这个实例}。比如你现在通过调用某个实例的某个方法,这个方法可能是一个远在天边的某个类给出的定义,就算如此,那个定义里面的self还是指调用这个方法的那个实例,这一点要牢记于心。



\subsection{类的操作第二版}
现在我们可以写出和之前那个版本相比更加专业的类的使用版本了。
\begin{xverbatim}[129]{py}
class Hero():
    def addlevel(self):
        self.level=self.level+1
        self.hp=self.hp+self.addhp

class Garen(Hero):
    def __init__(self):
        self.level=1
        self.hp=455
        self.addhp=96
        self.skill=['不屈','致命打击','勇气','审判','德玛西亚正义']

garen001=Garen()
for i in range(6):
    print('级别:',garen001.level,'生命值:' ,garen001.hp)
    garen001.addlevel()
print('盖伦的技能有:',"".join([x + '  ' for x in garen001.skill]))
\end{xverbatim}

似乎专业的做法类里面多放点方法,最好不要放属性,不太清楚是什么。但确实这样写给人感觉更干净点,方法是方法,如果没有调用代码就放在那里我们不用管它,后面用了构造函数我们就去查看相关类的构造方法,这样很省精力。


\section{类的操作第三版}
\label{sec:类的操作第三版}
\begin{xverbatim}[129]{py}
class Unit():
    def __init__(self,hp,atk,color):
        self.hp=hp
        self.atk=atk
        self.color=color
    def __str__(self):
        return '生命值:{0},攻击力:{1},颜色:\
        {2}'.format(self.hp,self.atk,self.color)

class Hero(Unit):
    def __init__(self,level,hp,atk,color):
        Unit.__init__(self,hp,atk,color)
        self.level=level
    def __str__(self):
        return '级别:{0},生命值:{1},攻击力:{2},\
        颜色:{3}'.format(self.level,self.hp,self.atk,self.color)

    def addlevel(self):
        self.level=self.level+1
        self.hp=self.hp+self.addhp
        self.atk=self.atk+self.addatk

class Garen(Hero):
    def __init__(self,color='blue'):
        Hero.__init__(self,1,455,56,color)
        self.name='盖伦'
        self.addhp=96
        self.addatk=3.5
        self.skill=['不屈','致命打击','勇气','审判','德玛西亚正义']

if __name__ == '__main__':
    garen001=Garen('red')
    garen002=Garen()
    print(garen001)
    unit001=Unit(1000,1000,'gray')
    print(unit001)
    for i in range(6):
        print(garen001)
        garen001.addlevel()
    print('盖伦的技能有:',"".join([x + '  ' for x in garen001.skill]))
\end{xverbatim}
现在就这个例子相对于第二版所作的改动,也就是核心知识点说明之。其中函数参量列表中这样表述\verb+color='blue'+表示blue是color变量的备选值,也就是color成了可选参量了。


\subsection{构造函数的继承和重载}
上面例子很核心的一个概念就是\verb+__init__+构造函数的继承和重载。比如我们看到garen001实例的创建,其中就引用了Hero的构造函数,特别强调的是\uwave{只有创造实例的时候比如这样的形式Garen()才叫做调用了Garen类的构造方法}, 比如这里\\
\verb+Hero.__init__(self,1,455,56,color)+就是调用了Hero类的构造函数,这个时候需要把self写上,因为self就是最终创建的实例garen001,而不是Hero,而且调用Hero类的构造函数就必须按照它的参量列表形式来。这个概念需要弄清楚!

理解了这一点,在类的继承关系中的构造函数的继承和重载就好看了。比如这里Hero类的构造函数又是继承自Unit类的构造函数,Hero类额外有一个参量level接下来也要开辟存储空间配置好。

\subsection{\_\_str\_\_{}函数的继承和重载}
第二个修改是这里重定义了一些类的\verb+__str__+函数,通过重新定义它可以改变默认print某个类对象是的输出。默认只是一段什么什么类并无具体内容信息。具体就是return一段你想要的字符串样式即可。




\subsection{类的其他内置方法}
类还有其他的一些内置方法,比如\verb+__add__+就控制这对象面对加号时候的行为。这些我们暂时先略过。

更多高级类的内置方法的讨论参见类的内置高级方法这一章\ref{sec:类的高级内置方法}。




\chapter{操作或者函数}
函数也是一个对象,叫函数对象。函数名和变量名一样都是引用,函数名后面带个括号才真正实际执行。比如下面不带括号就只是返回了对这个函数对象的引用地址。
\begin{Verbatim}
>>> print
<built-in function print>
\end{Verbatim}



\section{自定义函数}
定义函数用def命令,语句基本结构如下:
\begin{Verbatim}
def yourfunctionname(para001,para002...):
    do something001
    do something002
\end{Verbatim}



\section{参数传递问题}
函数具体参数的值是通过赋值形式\footnote{整个过程有点类似前面讨论的一般赋值语句,但又有点区别的。}来传递的,这有助于理解后面的不定变量函数。而函数的参数名是没有意义的,这个可以用lambda函式来理解之,def定义的为有名函数,有具体的引用地址,但内部作用原理还是跟lambda无名函式一样,形式参数名是x啊y啊都无所谓。为了说明这点,下面给出一个古怪的例子:

\begin{xverbatim}[129]{py}
y=1
def test(x,y=y):
    return x+y
print(test(4))
\end{xverbatim}

输出结果是5。我们看到似乎函数的形式参数y和外面的y不是一个东西,同时参数的传递是通过赋值形式进行的,那么具体是怎样的呢?具体的解释就是函数的形式参数y是这个函数自己内部的\textbf{本地变量}y,和外面的y不一样,更加深入的理解请看下面的变量作用域问题。

然后还有:
\begin{Verbatim}
>>> x=[1,2,3]
>>> for x in x:
...  print(x)
... 
1
2
3
\end{Verbatim}
我们知道for语句每进行一次迭代之前也进行了一次赋值操作,所以for语句里面刚开始定义的这个x和外面的x也不是一个东西,刚开始定义的x也是for语句内部的\textbf{本地变量}。更加深入的理解请看下面的变量作用域问题。

想到这里我又想起之前编写removeduplicate函数遇到的一个问题,那就是for语句针对列表这个可变的可迭代对象的工作原理是如何的?具体请看下面的例子:
\begin{Verbatim}
>>> lst=[1,2,3,4]
>>> for x in lst:
...  print(x,lst)
...  del lst[-1]
... 
1 [1, 2, 3, 4]
2 [1, 2, 3]
\end{Verbatim}
可迭代对象的惰性求值内部机制在我看来很神奇,目前还不太清楚,但从这个例子看来列表的惰性求值并没有记忆内部的数值,只是记忆了\uwave{返回返回值的次数}(合情合理),然后如果迭代产生了StopIteration异常就终止。



\section{变量作用域问题}
python的变量作用域和大部分语言比如c语言或lisp语言的概念都类似,就是函数里面是局部变量,一层套一层,里面可以引用外面,外面不可以引用里面。

具体实现机制是每个函数都有自己的命名空间,(和模块类似)就好像一个盒子一样封装着内部的变量。所谓的本地变量和函数有关,或者其他类似的比如for语句;所谓的全局变量和模块有关,更确切的表述是和文件有关,比如说在现在这个文件里,你可以通过导入其他模块的变量名,但实际上模块导入之后那些变量名都引入到这个文件里面来了。

具体实现和类的继承类似也是一种搜索机制,先搜索本地作用域\sidenote{函数就是有函数作用域的情况也是盒子里面有盒子},然后是上一层(def,lambda,for)的本地作用域,然后是全局作用域,然后是内置作用域。更加的直观的说明如下图所示:
\begin{fig}{python的变量作用域}
\caption{python的变量作用域}
\label{fig:python的变量作用域}
\end{fig}

简单来说python的变量作用域问题就是:盒子套盒子,搜索是从盒子最里面然后往外面寻找,里面可以用外面的变量,外面的不可以用里面的。


\subsection{内置作用域?}
内置作用域就是由一个\verb+__builtin+模块来实现的,python的作用机制最后会自动搜索这个内置模块的变量。这个内置模块里面就是我们前面学习的那些可以直接使用的函数名,比如print,range等等之类的,然后还有一些内置的异常名。

所以我们想到即使对于这些python的内置函数我们也是可以覆盖定义的,事实确实如此:
\begin{Verbatim}
>>> abs(-3)
3
>>> def abs(x):
...  print(x)
... 
>>> abs(3)
3
>>> abs(-3)
-3
\end{Verbatim}


\subsection{global命令}
如果希望函数里面定义的变量就是全局变量,在变量声明的时候前面加上\textbf{global}命令即可。

通常不建议这么做,除非你确定需要这么做,然后你需要写两行代码才能实现,意思也是不推荐你这么做。
\begin{tcbpython}
def test():
    global var
    var= 'hello'
test()
print(var)
\end{tcbpython}

而且就算你这样做了,这个变量也只能在本py文件中被引用,其他文件用不了。推荐的做法是另外写一个专门用于配置参数的config.py文件,然后那些全局变量都放在里面,如果某个文件要用,就import进来。而对与这个config.py文件的修改会影响所有的py文件配置,这样让全局变量可见可管可控更加通用,才是正确的编程方式。


\subsection{nonlocal命令}
nonlocal命令python3之后才出现,这里实现的概念有点类似于lisp语言的闭包(closure技术),就是如果你有某个需要,需要函数记忆一点自己的状态,同时又不想这个状态信息是全局变量,也不希望用类的方式来实现,那么就可以用nonlocal命令来简单地完成这个任务。

global意味着命名只存在于一个嵌套的模块中,而nonlocal的查找只限于嵌套的def中。要理解nonlocal首先需要理解函数里面嵌套函数的情况——也就是所谓的工厂函数,一个函数返回一个函数对象。比如说
\begin{tcbpython}
def add(x):
    x=x
    def action(y):
        return x+y
    return action
\end{tcbpython}
\begin{Verbatim}
>>> add1=add(1)
>>> add1(5)
6
>>> add2=add(2)
>>> add2(5)
7
\end{Verbatim}
这里的return action是返回一个函数对象,这样add1的实际接口是def action那里。熟悉lisp语言的明白,action外面的那个函数的变量叫做自由变量,不过嵌套函数在这里可以引用自由变量\footnote{如果自己定义那么就是自己的本地变量了,这里的自由变量的意思是嵌套函数自己没有定义,引用母函数的变量。}但\uwave{不能直接修改}自由变量。如果我们声明nonlocal x,那么就可以修改嵌套函数外面声明的变量了。

\begin{tcbpython}
def add(x):
    x=x
    def action(y):
        nonlocal x
        x=x+1
        return x+y
    return action
\end{tcbpython}
\begin{Verbatim}
>>> add2=add(2)
>>> add2(5)
8
>>> add2(5)
9
>>> add2(5)
10
\end{Verbatim}
然后我们看到这个生产出来的函数具有了运行上的状态性,实际上通过类也能构建出类似的效果,不过对于某些问题可能闭包方式处理显得更适合一些。

下面给出一个稍微合理点的例子:
\begin{tcbpython}
def myrange(n):
    i=n
    def action():
        nonlocal i
        while i>0:
            i=i-1
            return i
    return action
\end{tcbpython}

\begin{Verbatim}
>>> myrange5=myrange(5)
>>> myrange5()
4
>>> myrange5()
3
>>> myrange5()
2
>>> myrange5()
1
>>> myrange5()
0
>>> myrange5()
>>> 
\end{Verbatim}

下面给出类似的类的实现方法:
\begin{tcbpython}
class myrange:
    def __init__(self,n):
        self.i=n
    def action(self):
        while self.i > 0:
            self.i -= 1
            return self.i
\end{tcbpython}
\begin{Verbatim}
>>> myrange5=myrange(5)
>>> 
>>> myrange5.action()
4
>>> myrange5.action()
3
>>> myrange5.action()
2
>>> myrange5.action()
1
>>> myrange5.action()
0
>>> myrange5.action()
>>> 
\end{Verbatim}
我们看到从编码思路上基本上没什么差异,可以说稍作修改就可以换成类的实现版本。推荐一般使用类的实现方法。但有的时候可能用类来实现有点不伦不类和大材小用了。这里就不做进一步讨论了,闭包思想是函数编程中很重要的一个思想,学习了解一下也好。



\section{参数和默认参数}
定义的函数圆括号那里就是接受的参数,如果参数后面跟个等号,来个赋值语句,那个这个赋的值就是这个参数的默认值。比如下面随便写个演示程序:
\begin{xverbatim}[129]{py}
def test(x='hello'):
    print(x)
test()
test('world')
\end{xverbatim}


\section{不定参量函数}
我们在前面谈到sum函数\ref{sec:sum函数}只接受一个列表,而不支持这样的形式:sum(1,2,3,4,5)。现在我们设计这样一个可以接受不定任意数目参量的函数。首先让我们看看一种奇怪的赋值方式。

\subsection{序列解包赋值}
\label{sec:序列解包赋值}
\begin{Verbatim}
>>> a,b,*c=1,2,3,4,5,6,7,8,9
>>> print(a,b,c,sep=' | ')
1 | 2 | [3, 4, 5, 6, 7, 8, 9]
>>> a,*b,c=1,2,3,4,5,6,7,8,9
>>> print(a,b,c,sep=' | ')
1 | [2, 3, 4, 5, 6, 7, 8] | 9
>>> *a,b,c=1,2,3,4,5,6,7,8,9
>>> print(a,b,c,sep=' | ')
[1, 2, 3, 4, 5, 6, 7] | 8 | 9
\end{Verbatim}
带上一个星号*的变量变得有点类似通配符的味道了,针对后面的序列\footnote{似乎序列赋值内置迭代操作}(数组,列表,字符串),它都会将遇到的元素收集在一个列表里面,然后说是它的。

for语句也支持序列解包赋值,也是将通配到的的元素收集到了一个列表里面,如:
\begin{xverbatim}[129]{py}
for (a,*b,c) in [(1,2,3,4,5,6),(1,2,3,4,5),(1,2,3,4)]:
    print(b)
\end{xverbatim}


\subsection{函数中的通配符}
\begin{Verbatim}
>>> def test(*args):
...  print(args)
... 
>>> test(1,2,3,'a')
(1, 2, 3, 'a')
\end{Verbatim}
我们看到类似上面序列解包赋值中的带星号表通配的概念,在定义函数的时候写上一个带星号的参量(我们可以想象在函数传递参数的时候有一个类似的序列解包赋值过程),在函数定义里面,这个args就是接受到的参量组成的\emph{元组或称之为数组}。


\subsection{mysum函数}
\begin{xverbatim}[129]{py}
def mysum(*args):
    return sum(args[:])

print(mysum(1,2,3,4,5,6))
\end{xverbatim}
这样我们定义的可以接受任意参数的mysum函数,如上所示。具体过程就是将接受到的args(已成一个元组了),然后用sum函数处理了一下即可。


\subsection{任意数目的可选参数}
在函数定义的写上带上两个星号的变量**args,那么args在函数里面的意思就是接受到的可选参数组成的一个字典值。
\begin{Verbatim}
>>> def test(**args):
...  return args
... 
>>> test(a=1,b=2)
{'b': 2, 'a': 1}
\end{Verbatim}

我们看到利用这个可以构建出一个简单的词典对象生成器。

\subsection{解包可迭代对象传递参数}
之前*args是在函数定义中,然后通配一些参数放入元组中。这里是在函数调用中,针对可迭代对象,可以用一个*星号将其所包含的元素迭代出来,然后和参数一一对应赋值。
\begin{Verbatim}
>>> def f(a,b,c=3):
...  print(a,b,c)
>>> map=map(lambda x : x+2 ,[1,2,3])
>>> f(*map)
3 4 5
>>> f(*[1,2,3])
1 2 3
\end{Verbatim}

\subsection{解包字典成为关键字参数}
和上面的类似,通过**args语法可以将某个字典对象解包成为某个函数的关键字参数。还是以上面那个函数f为例子:
\begin{Verbatim}
>>> f(**{'c':6,'b':4,'a':2})
2 4 6
\end{Verbatim}



\section{参数的顺序}
老实说一般参数,可选参数(关键字参数),任意(通配)参数,任意(通配)关键字参数所有这些概念混在一起非常的让人困惑。就一般的顺序是:
\begin{enumerate}
\item 一般参数,这个如果有一定要在第一位,然后通过位置一一对应分配参数。
\item 关键字参数,关键字参数跟在一般参数后面,通过匹配变量名来分配。
\item 通配一般参数,其他额外的非关键字的参数分配到*args元组里面。
\item 通配关键字参数,其他额外的关键字参数分配到**args字典里面,这个必须在最后面。
\end{enumerate}
一般的情况就是这些吧,可能你会遇到更加困难的情况,到时候再说吧。



\section{递归函式}
\label{sec:递归函式}
虽然递归函式能够在某种程度上取代前面的一些循环或者迭代程序结构,不过不推荐这么做。这里谈及递归函式是把某些问题归结为数学函数问题,而这些问题常常用递归算法更加直观(不一定高效)。比如下面的菲波那奇函数:
\begin{xverbatim}[129]{py}
def fib(n):
    if n==0:
        return 1
    if n==1:
        return 1
    else:
        return fib(n-1)+fib(n-2)
        
for x in range(5):
    print(fib(x))
\end{xverbatim}
我们可以看到,对于这样专门的数学问题来说,用这样的递归算法来表述是非常简洁易懂的。至于其内部细节,我们可以将上面定义的fib称之为函式,函式是一种操作的模式,然后具体操作就是复制出这个函式(函数或者操作都是数据),然后按照这个函式来扩展生成具体的函数或者操作。

下面看通过递归函式来写阶乘函数,非常的简洁,我以为这就是最好最美的方法了。
\begin{xverbatim}[129]{py}
def fact(n):
    if n == 0:
        return 1
    else:
        return n*fact(n-1)
        
print(fact(0),fact(10))
\end{xverbatim}

\subsection{什么时候用递归?}
最推荐使用递归的情况是这样的情况,那就是一份工作(或函数)执行一遍之后你能够感觉到虽然所有的工作没有做完,但是已经做了一小部分了,有了一定的进展了,就好比是蚂蚁吞大象一样,那么这个时候你就可以使用递归思想了。其次有的时候有那么一种情况虽然表面上看似乎并没有什么进展,但事情在发展,你能感受到有一个条件最终将会终止程序从而得到一个输出,那么这个时候就可以用递归。

递归思想最核心的两个概念就是一做了一小部分工作,你能感觉到做着做着事情就会做完了;二有一个终止判断最终将会起作用。

其实通过递归函式也可以实现类似for的迭代结构\sidenote{这种情况不推荐使用递归},不过我觉得递归函式还是不应该滥用。比如下面通过递归函式生成一种执行某个操作n次的结构:

\begin{xverbatim}[129]{py}
def dosomething(n):
    if n==0:
        pass
    elif n==1:
        print('do!')
    else:
        print('do!')
        return dosomething(n-1)

print(dosomething(5))
\end{xverbatim}
可以看到,如果把上面的print语句换成其他的某个操作,比如机器人向前走一步,那么这里dosomething换个名字向前走(5)就成了向前走5步了。

\subsection{lisp的car-cdr递归技术}
在lisp语言中, car-cdr递归技术是很重要的一门技术,它的特长就是遍历随意嵌套的列表结构可以同一对列表中的每一个元素执行某种操作。

首先我们来看下面的例子,一个把任意嵌套列表所有元素放入一个列表中的函数:
\begin{xverbatim}[129]{py}
lst = [[1,2,[3]],[4,[5,[[[[10],11]]]],(1,2,3)],[{'a','b','c'},8,9]]

def is_list(thing):
    return isinstance(thing, list)

def flatten(iter):
    templst = []
    for x in iter:
        if not is_list(x):
            templst.append(x)
        else:
            templst += flatten(x)
    return templst

print(flatten(lst))
\end{xverbatim}

这个函数的逻辑是如果是最小元素对象不是列表,那么收集进列表;如果不是,那么把它展开,这里就是调用的原函数继续展开函式。

上面的例子严格意义上来讲还不算lisp的经典car-cdr递归技术,下面给出一个典型的例子,就是复制任意嵌套结构的列表。当然列表的copy方法就可以做这个工作,这里主要通过这个例子来进一步深入car-cdr技术。

\begin{tcbpython}[]
def is_list(thing):
    return isinstance(thing, list)

def copy_list(lst):
    if  not  is_list(lst):
        return lst
    elif lst == []:
        return []
    else:
        return [copy_list(lst[0])] + copy_list(lst[1:])

print(copy_list([1,[2,6],3]))
\end{tcbpython}


这种嵌套列表的复制已经后面的修改等等操作,最合适的就是lisp的car-cdr技术了,但我不得不承认,这种递归写法是递归函式里面最难懂的了。

不管怎么严格,在这个基础之上,因为第一个if not的语句中传递下来的lst实际上已经是非列表的其他元素了,然后我们可以进行一些其他修改操作,这样在保持原列表的复杂嵌套的基础上,等于遍历的对列表中的所有元素进行了某种操作。

比如所有元素都平方:
\begin{tcbpython}[]
def square(x):
    return x**2

def square_list(lst):
    if  not  is_list(lst):
        return square(lst)
    elif lst == []:
        return []
    else:
        return [square_list(lst[0])] + square_list(lst[1:])

print(square_list([1,[2,6],3]))
\end{tcbpython}

我们可以想像更加复杂功能的函数作用于列表中所有的元素同时又不失去原列表复杂的嵌套结构,lisp的car-cdr这种技术了解一下吧,但是不是一定要使用复杂的嵌套结构呢?也许没有必要吧。。


\section{lambda函式}
lambda
λ表达式这个在刚开始介绍lisp语言的时候已有所说明,简单来说就是函数只是一个映射规则,变量名,函数名都无所谓的。这里就是没有名字的函数的意思。

具体的样子如下面所示:
\begin{xverbatim}[129]{py}
f=lambda x,y,z:x+y+z
print(f(1,2,3))
\end{xverbatim}

lanmbda函式在有些情况下要用到,比如pyqt里面的信号-槽机制用connect方法的时候,槽比如是函数名或者无参函数,如果用户想加入参量的话,可以使用lamba函式引入,具体这里我还不够清晰。

读者如果对lambda函式表达不太熟悉强烈建议先简单学一学\textbf{scheme}语言。

\section{print函数}
print函数因为很常用和基础,就放在这里了。

print函数接受任意的参量,逐个打印出来。然后它还有一些关键字参数,\textbf{sep}:默认值是' ',也就是一个空格,如果修改为空字符串,那么逐个打印出来的字符之间就没有间隔了。\textbf{end}:默认值是'\textbackslash{}n',\textbf{file}默认值是sys.stdout,也就是在终端显示,你可以修改为某个文件变量,这样直接往某个文件里面输出内容。



\chapter{模块}
现在让我们进入模块基础知识的学习吧,建立编写自己的模块,这样不断积累自己的知识,不断变得更强。

实际上之前我们已经接触过很多python自身的标准模块或者其他作者写的第三方模块,而import和from语句就是加载模块用的。这里我们主要讨论如何自己编写自己的模块。

from语句和import语句内部作用机制很类似,只是在变量名的处理方式上有点差异(from会把变量名复制过来)。这里重点就import的工作方式说明如下:
\begin{enumerate}
\item 首先需要找到模块文件。
\item 然后将模块文件编译成位码(需要时,根据文件的时间戳。),你会看到新多出来一个\verb+__pycache__+文件夹。
\item 执行编译出来的位码,创建该py文件定义的对象。
\end{enumerate}
这三个步骤是第一次import的时候会执行的,第二次import的时候会跳过去,而直接引用内存中已加载的对象。



\section{找到模块文件}
python模块的搜索路径会搜索几个地方,这些地方最后都会放在sys.path这个列表里面,所以在你的py文件刚开始修改这个sys.path,append上你想要的地址也是可以的。我在这里选择了这种简单的方法,除此之外还有很多方法这里先不涉及。

比如主文件一般如下:
\begin{Verbatim}
import os,sys
sys.path.append(os.environ['HOME']+'/pymf')
from pyconfig import *
\end{Verbatim}
这里为什么使用\verb+from pyconfig import *+这样的语句而不是import语句呢?因为我决定整个项目的主py文件除了这一个from语句之外不会再import或者from其他模块了,其他所有模块的引用都放在pyconfig.py这个主配置文件里面。

pyconfig.py任务就是加载最常用最通用的一些模块,如果你实际编写的另外一个项目通用pyconfig文件满足不了你的要求了,那么你可以把那个pyconfig文件复制过来,然后放在你的项目文件夹里面,然后继续衍化修改。这个经验是我从\LaTeX 文档的编写中总结出来的,既满足了共性又满足了个性。

那么为什么要用from语句,很简单。如果用import语句,那么pyconfig.py文件里面import math模块,在主py文件里面引用就要使用这样的格式\verb+pyconfig.math.pi+,这既不方便而且违背大家平时惯用的那种math.pi格式。


现在我们让可以开始编写自己的模块吧。

\section{编写模块}
well,编写模块就是一些py文件,然后模块的名字和py文件里面的内容编写好就是了。

我现在编写了一个pyconfig.py文件,放在主文件夹(ubuntu系统)的pymf文件里面的。里面定义了一个斐波拉契函数,如下所示:
\begin{Verbatim}
#菲波那奇数列
def fib(n):
    if n==0:
        return 1
    if n==1:
        return 1
    else:
        return fib(n-1)+fib(n-2)
\end{Verbatim}

然后我们的测试小脚本如下:
\begin{xverbatim}[129]{py}
import os,sys
sys.path.append(os.environ['HOME']+'/pymf')
from pyconfig import *

print([fib(n) for n in range(10)])
\end{xverbatim}


\section{import语句}
import语句的一般使用方法之前已有接触,比如import math,然后要使用math模块里面的函数或者类等需要使用这样的带点的变量名结构:math.pi。

此外import语句还有一个常见的缩写名使用技巧,比如import numpy as np,那么后面就可以这样写了, np.array,而不是numpy.array。


\section{from语句}
from语句的使用有以下两种情况:
\begin{Verbatim}
from this import this
from what import *
\end{Verbatim}
第一种形式是点名只导入某个变量,第二种形式是都导入进来。我想读者肯定知道这点,使用第二种导入形式的时候要小心变量名覆盖问题,这个自己心里有数即可。



\section{reload函数}
reload函数可以重新载入某个模块,reload函数的优点就是不需要重新启动应用程序,更加合理的动态重载一些模块。reload只能用于python编写的模块,在python3中,reload函数被移到imp模块里面去了,因此首先需要import imp才能使用了。比如说:
\begin{Verbatim}
from imp import reload
reload(somemodule)
\end{Verbatim}




\chapter{异常处理}





\part{python3高级篇}
\chapter{字符串操作进阶}
\section{bytes类型介绍}
一般就使用str字符串类型然后使用默认的utf-8编码,然后大部分情况下不会有任何问题。而且按照《Unix编程艺术》作者的观点,一般数据的存储也推荐都文本化,就直接以字符串的形式存储,至少在编程开发早期不会有任何问题。不过到后面可能要考虑速度问题还有保密问题还有某些特殊的领域是需要直接和二进制打交道的,比如处理图像声音文件等。

python3为了真正支持二进制数据的处理,引入了一个新的类型bytes。bytes字节类型的定义是8位整数的不可变序列,表示绝对的字节值。bytes类型支持之前谈论的str也就是常用的字符串类型的几乎所有的操作:包括类型内部的方法,序列操作,re模块匹配等,但不包括字符串的格式化(也就是format方法)。

虽然bytes类型也可以打印,因为8位整数正好对应ASCⅡ的编码,但这只是使用方便罢了,最好还是把bytes看作纯数值没有任何字符含义的二进制数据。我们在读取文件的时候也分为常规读取和以二进制形式读取(需要加个b字符),这些都涉及到编码问题。

bytes类型有一个decode方法,可以把自己按照某种编码方式(比如'big5'或'utf-8'等)进行解码\footnote{二进制数据难读,相当于密码。}。字符串类型有一个encode方法,就是把这个字符串按照某种编码方式进行编码。
\begin{Verbatim}
>>> string001='你好'
>>> bytes001=string001.encode('utf-8')
>>> bytes001
b'\xe4\xbd\xa0\xe5\xa5\xbd'
>>> bytes002=string001.encode('big5')
>>> bytes002
b'\xa7A\xa6n'
>>> string002=bytes002.decode('utf-8')
Traceback (most recent call last):
  File "<stdin>", line 1, in <module>
UnicodeDecodeError: 'utf-8' codec can't decode byte 0xa7 in position 0: invalid start byte
>>> string002=bytes002.decode('big5')
>>> string002
'你好'
\end{Verbatim}
我们看到不同编码解码弄混了会出问题,本来想找个例子然后出现字符混乱的情况的。






\section{字符串中的赋值语句变字典值}
现在我们打算定义一个

\chapter{类的高级知识}
在造成程序探索过程中,大部分功能都可以通过一些函数的形式来实现,只有在软件开发中期才开始考虑优化和类的包装问题。面向对象编程思想并不是一定要如此不可,在某些领域简单的函数编程也许更加简单和直接,这个倒不是我个人的观点,很多极客比如《Unix编程艺术》的作者就如此认为。



\section{静态方法}
\begin{Verbatim}
class Test:
#    @staticmethod
    def hello():
        print('aaa')

test=Test()
test.hello()
\end{Verbatim}
在上面的例子中,我们希望创造一个函数,这个函数和self或者其他类都没有关系(这里的其他类一般指继承来的)。如上所示,hello函数只是希望简单打印一小段字符,如上面这样的代码是错误的,如果我们在这个函数上面加上\textbf{@staticmethod},那么上面这段代码就不会出错了,

\begin{Verbatim}
class Test:
    @staticmethod
    def hello():
        print('aaa')

test=Test()
test.hello()
\end{Verbatim}
这样在类里面定义出来的函数叫做这个类的静态方法,静态方法同样可以继承等等,而静态方法通常使用最大的特色就是不需要建立实例,即可以直接从类来调用,如下所示:
\begin{Verbatim}
class Test:
    @staticmethod
    def hello():
        print('aaa')

Test.hello()
\end{Verbatim}

\chapter{类的高级内置方法}
\label{sec:类的高级内置方法}
这里所谓的高级内置方法是指除了构造方法\verb+__init__+和打印方法\verb+__str__+之外其他方法。

这一块的内容如果单纯罗列式的介绍会很繁琐和无趣,这里主要通过一个讲解一个融合了我的一点哲学理念的有趣的例子来讲解。整个例子讲完之后一些常用的类的内置方法都会有所涉及,没有涉及的请读者自己Google搜索之。



\section{和迭代操作有关}
比如文件对象本身就是可迭代的,调用\verb+__next__+方法就返回文件中下一行的内容,到达文件尾也就是迭代越界了返回:\textbf{StopIteration}异常。

\subsection{next函数}
next函数比如next(f)等价于\verb+f.__next__()+,其中f是一个文件对象。

\begin{Verbatim}
>>> for line in open('removeduplicate.py'):
...  print(line,end='')
... 
#!/usr/bin/env python3
#-*-coding:utf-8-*-
#此处一些内容省略。
    
>>> f=open('removeduplicate.py')
>>> next(f)
'#!/usr/bin/env python3\n'
\end{Verbatim}

\subsection{iter函数}
iter函数接受一个不可迭代对象(比如列表)然后生成一个可迭代对象。这样的可迭代对象就可以使用next函数和\verb+__next__+方法了。

\begin{Verbatim}
>>> list=[1,2,3]
>>> next(list)
Traceback (most recent call last):
  File "<stdin>", line 1, in <module>
TypeError: 'list' object is not an iterator
>>> i=iter(list)
>>> next(i)
1
>>> i.__next__()
2
\end{Verbatim}

下面是for语句的while实现版本:
\begin{Verbatim}
>>> list=[1,2,3]
>>> iter=iter(list)
>>> while True:
...    try:
...        x=next(iter)
...    except StopIteration:
...        break
...    print(x)
... 
1
2
3
\end{Verbatim}

range对象也可以通过iter函数来生成一个可迭代对象。

\subsection{enumerate函数}
enumerate函数返回一个enumerate对象,这个对象将偏移值和元素组合起来,成为一个可迭代对象了,也就是next函数也可以使用了。

\begin{Verbatim}
>>> for (offset,item) in enumerate('abcdefg'):
...  print(offset,item)
... 
0 a
1 b
2 c
3 d
4 e
5 f
6 g
\end{Verbatim}

\chapter{深入理解python3的迭代}
\label{sec:深入理解python3的迭代}
在python中一般复杂的代码运算效率就会低一点,如果完成类似的工作但你可以更简单的语句那么运算效率就会高一点。当然这只是python的一个设计理念,并不尽然,但确实很有意思。

程序结构中最有用的就是多个操作的重复,其中有迭代和递归还有一般的循环语句。递归函式感觉对于某些特殊的问题很有用,然后一般基于数据结构的不是特别复杂的操作重复用迭代语句即可,最后才考虑一般循环语句。

迭代语句中for语句运算效率最低,然后是map函数(不尽然),然后是列表解析。所以我们在处理问题的时候最pythonic的风格,运算效率最高的就是列表解析了,如果一个问题能够用列表解析解决那么就用列表解析解决,因为python的设计者的很多优化工作都是针对迭代操作进行的,然后python3进一步深化了迭代思想,最后python中的迭代是用c语言来实现的(你懂的)。

可是让我们反思一下为什么列表解析在问题处理的时候如此通用?比如说range函数或者文件对象或者列表字符串等等,他们都可以称之为可迭代对象。可迭代对象有内置方法\verb+__next__+这个我们之前有所谈及,可迭代对象最大的特色就是有一系列的元素,然后这一系列的元素可以通过上面的内置方法逐个调出来,而列表解析就是对这些调出来的元素进行了某个表达式操作,然后将其收集起来。这是什么?我们看下面这张图片:
\begin{fig}{列表解析}
\caption{列表解析}
\label{fig:列表解析}
\end{fig}

这张图片告诉我们列表解析和数学上所谓的集合还有函数的定义非常的类似,可迭代对象就好像是一个集合(有顺序或者没顺序都行),然后这些集合中的所有元素经过了某个操作,这个操作似乎就是我们数学中定义的函数,然后加上过滤条件,某些元素不参加运算,这样就生成了第二个可迭代对象(一般是列表也可以是字典什么的。)

有一个哲学上的假定,那就是我们的世界一切问题都可以用数学来描述,而一些数学问题都可以用函数即如上的信息操作过滤流来描述之。当然这不尽然,但我们可以看到列表解析在一般问题处理上是很通用的思想。

不过我们看到有限的元素的集合问题适合用迭代,但无限元素的集合问题也许用递归或者循环更适合一些。然后我们又想到集合的描述分为列举描述(有限个元素的列举)和定义描述。比如说1<x<10,x属于整数,这就定义了一个集合。那么我们就想到python存在这样的通过描述而不是列举(如列表一样)的集合吗?range函数似乎就是为了这样的目的而生的,比如说range(10)就定义了[0,10)这一系列的整数集合,range函数生成一个range对象,range对象是一个可迭代对象,我们可以把它看作可迭代对象中的描述集合类型吧。这时我们就问了,既然0<=x<10这样的整数集合可以通过描述来实现,那么更加复杂的函数描述可不可以实现呢?我们可不可以建立更加复杂的类似range对象的描述性可迭代对象呢?

\section{yield语句}
一般函数的定义使用return语句,如果使用yield语句,我们可以构建出一个生成器函数,
\begin{Verbatim}
>>> def test(x):
...    for i in range(x):
...        yield 2*i+ 1
... 
>>> test(3)
<generator object test at 0xb704348c>
>>> [x for x in test(3)]
[1, 3, 5]
>>> [x for x in test(5)]
[1, 3, 5, 7, 9]
\end{Verbatim}

这个test函数叫做什么生成器函数,返回的是什么generator object,生成器对象?anyway,通过yield这样的形式定义出来的生成器函数返回了一个生成器对象和range对象类似,都是描述性可迭代对象,里面的元素并不立即展开,而是请求一次运算一次,所以这种编程风格对内存压力很小,主要适合那些迭代元素特别多的时候的情况吧。

上面的test函数我们就可以简单理解为2x+1,其中0<=x<n(赋的值)。

下面给出一个问题作为练习:描述素数的生成器函数。
这是网上流行的素数检验函数,效率还是比较高的了。
\begin{tcbpython}
def isprime(n):
    if n ==2:
        return True
    #按位与1,前面一定都是0个位数如果是1则
    #是奇数则返回1则真则假,如果是偶数则返回
    #0则假则真
    elif n<2 or not n & 1:
        return False
    #埃拉托斯特尼筛法...
#查一个正整数N是否为素数,最简单的方法就是试除法,
#将该数N用小于等于N**0.5的所有素数去试除,
#若均无法整除,则N为素数
    for x in range(3,int(n**0.5)+1,2):
        if n % x == 0:
            return False
    return True
\end{tcbpython}

然后我们给出两种形式的素数生成器函数,其中prime2的意思是范围到(to)那里。而prime(n)的意思是到第几个素数。我们知道生成器函数是一种惰性求值运算,然后yield每迭代一次函数运算一次(即产生一次yield),但这种机制还是让我觉得好神奇。

\begin{tcbpython}
def prime2(n):
    for x in range(n):
        if isprime(x):
            yield x

def prime(n):
    i=0
    x=1
    while i<n:
        if isprime(x):
            i +=1
            yield x
        x +=1
\end{tcbpython}

在加载这些函数之后我们可以做一些检验:
\begin{Verbatim}
>>> isprime(479)
True
>>> [x for x in prime2(100)]
[2, 3, 5, 7, 11, 13, 17, 19, 23, 29, 31, 37, 41, 43, 47, 53, 59, 61, 67, 71, 73, 79, 83, 89, 97]
>>> [x for x in prime2(1000) if 100< x < 200]
[101, 103, 107, 109, 113, 127, 131, 137, 139, 149, 151, 157, 163, 167, 173, 179, 181, 191, 193, 197, 199]
>>> len([x for x in prime2(10000) if -1 < x < 3572])
500
>>> [x for x in prime(1)]
[2]
>>> [x for x in prime(2)]
[2, 3]
\end{Verbatim}


\section{实数集合表示}
range函数只能近似表示某部分的整数集合,那么实数集合的表示呢?我估计目前有那个python模块已经实现这个功能呢?是什么呢?

\section{map和filter函数}
按照之前的迭代模式的描述,虽然使用常见的列表解析格式(for 语句)就可以完成对某个集合中各个元素的操作或者过滤,不过python中还有另外两个函数来实现类似的功能,map对应对集合中各个元素进行某个函数操作(可以接受lambda函式),而filter则实现如上所述的过滤功能。然后值得一提的是python3之后map函数和filter函数返回都是一个可迭代对象而不是列表,和range函数等其他可迭代对象一样可用于列表解析结构。

\subsection{map函数}
这里列出一些例子,具体编程还是先考虑列表解析模式,可能会在某些情况下需要用到map函数?

\begin{Verbatim}
>>> map(abs, [-2,-1,0,1,2])
<map object at 0xb707dccc>
>>> [x for x in map(abs, [-2,-1,0,1,2])]
[2, 1, 0, 1, 2]
>>> [x for x in map(lambda x : x+2, [-2,-1,0,1,2])]
[0, 1, 2, 3, 4]
\end{Verbatim}


map函数还可以接受两个可迭代对象的协作参数模式,这个学过lisp语言的会觉得很眼熟,不过这里按照我们的理解也是很便捷的。具体就是第一个可迭代对象取出一个元素作为map的函数的第一个参数,然后第二个可迭代对象取出第二个参数,然后经过函数运算,得到一个结果,这个结果如果不列表解析的话就是一个map对象(可迭代对象),然后展开以此类推。值得一提的是两个可迭代对象的深度由\uwave{最短}的那个决定,请看下面的例子:
\begin{Verbatim}
>>> [x for x in map(lambda x,y : x+y, [-2,-1,0,1,2],[-2,-1,0,1,2])]
[-4, -2, 0, 2, 4]
>>> [x for x in map(lambda x,y : x+y, [-2,-1,0,1,2],[-2,-1,0,1])]
[-4, -2, 0, 2]
\end{Verbatim}

\subsection{filter函数}
同样和上面的谈及的类似,filter函数过滤一个可迭代对象然后产生一个可迭代对象。类似的功能可以用列表解析的后的if语句来实现。前面谈到map函数的时候提及一般还是优先使用列表解析模式,但filter函数这里有点不同,因为列表解析后面跟个if可能有时会让人困惑,所有推荐还是用filter函数来进行可迭代对象的过滤操作,当然你喜欢用if也没什么问题喽。

filter函数的基本逻辑是只有return True(用lambda表达式就是这个表达式的值为真,具体请参看python的逻辑小知识和布尔值的一些规则\ref{sec:布尔值})的时候元素才被收集起来,或者说是过滤出来。这里强调True是因为如果你的函数没有return值那么默认的是return None,这个时候元素也是不会过滤出来的。

请参看下面的例子来理解:\sidenote{这里位运算与就是控制个位数是1那么就是奇数,这种方式更加的节省计算。}
\begin{Verbatim}
>>> [x for x in filter(lambda x:x&1,[1,2,3,5,9,10,155,-20,-25])]
[1, 3, 5, 9, 155, -25]
>>> [x for x in filter(lambda x:not x&1,[1,2,3,5,9,10,155,-20,-25])]
[2, 10, -20]
\end{Verbatim}


当然你也可以传统的编写函数:
\begin{Verbatim}
>>> def even(n):
...    if n % 2 ==0:
...         return True

>>> [x for x in filter(even,[1,2,3,5,9,10,155,-25])]
[2, 10]
\end{Verbatim}

\subsection{zip函数}
这里就顺便把zip函数也一起提了,zip函数同样返回一个可迭代对象,他接受两个可迭代对象,然后将其拼接起来。和map函数类似迭代深度由\uwave{最短}的那个可迭代对象决定。
\begin{Verbatim}
>>> zip(['a','b','c'],[1,2,3,4])
<zip object at 0xb7055e6c>
>>> [x for x in zip(['a','b','c'],[1,2,3,4])]
[('a', 1), ('b', 2), ('c', 3)]
>>> list(zip(['a','b','c'],[1,2,3,4]))
[('a', 1), ('b', 2), ('c', 3)]
>>> dict(zip(['a','b','c'],[1,2,3,4]))
{'c': 3, 'b': 2, 'a': 1}
\end{Verbatim}


\section{完整的列表解析结构}
下面给出一个完整的列表解析结构,最常见的情况一般就一两个for语句吧,这里if外加个括号是可选项的意思。
\begin{Verbatim}
[ expression for var1 in iterable1 [if condition1 ]
                    for var2 in iterable2 [if condition2 ]
                    ........
                            ]
\end{Verbatim}

这里的iterable1是指某个可迭代对象,也就是说那些能够返回可迭代对象的函数比如map,filter,zip,range等函数都可以放进去。不过我们要克制自己在这里别写出太过于晦涩的程序了。还有for循环语句也别嵌套太多了,这样就极容易出错的。

下面这个程序大家看看:
\begin{Verbatim}
>>> [x+str(y) for x in ['a','b','c'] for y in [1,2,3,4,5,6] if y & 1]
['a1', 'a3', 'a5', 'b1', 'b3', 'b5', 'c1', 'c3', 'c5']
>>> [x+str(y) for x in ['a','b','c'] for y in [1,2,3,4,5,6] if not  y & 1]
['a2', 'a4', 'a6', 'b2', 'b4', 'b6', 'c2', 'c4', 'c6']
\end{Verbatim}

\section{列表解析的好处}
在熟悉列表解析的语句结构之后,一两个for语句不太复杂的情况下,还是很简单明了的。同时语法也更加精炼,同时运行速度较for循环要至少快上一倍。最后python的迭代思想深入骨髓,以后python的优化工作很多都围绕迭代展开,也就是多用列表解析会让你的代码以后可能运行的更快。

有这么多的好处,加上这么cool的pythonic风格,推荐大家多用列表解析风格解决问题。





\chapter{模块包}
多个模块py文件组成一个多文件夹目录的整体就是一个模块包。

模块包这部分知识是我们理解前人编写的各个有用的模块包的基础,同时以后我们自己要编写大型的项目也是一个人编写一个模块,一个模块对应一个任务或者一个功能的形式展开的,然后多个模块合并成一个大型的模块包。以前我们都是编写的不超过一百行的小python代码,不过就是对于大型的项目也不意味着我们要找一个大型的编辑器,然后一写就是上万行。在模块包的合理布局下,我们完全还可以轻松的一次编写也就那么一两百行的小代码,最后各个模块组合起来,就是一个宏大的系统了。

模块包和模块在使用上的区别如下所示:
\begin{tcbpython}
import  dir1.dir2.mod
form dir1.dir2.mod import *
\end{tcbpython}
这里模块文件在系统中的具体路径是dir0/dir1/dir2/mod.py,其中dir0是python模块的搜索路径,然后在这个搜索路径下有一个dir1文件夹,然后下面有一个dir2文件夹,然后dir2文件夹里面有个mod.py文件,这样python程序就找到这个py文件了,然后按照前面介绍的如何处理加载模块文件的方式继续进行下一步工作。


\section{\_{}\_{}init\_{}\_{}文件}
well,模块包和简单地模块在管理上多出来的唯一的一个内容就是你需要在每个文件夹里面加一个\verb+__init__.py+文件,文件的内容就是空白都没关系,但必须要有。然后前面说的dir0,也就是默认的各个python模块的搜索路径,这个文件夹里面不需要加\verb+__init__.py+文件。


\subsection{我的第一个模块包例子}
还记得在前面模块那一章我们通过修改sys.path把主文件夹下的pymf文件夹加入python模块的搜索路径了。现在我们在这个pymf文件夹里面新建一个文件夹,\verb+mymodule+。然后进入mymodule文件夹,新建一个空白文件\verb+__init__.py+,然后新建一个文件\verb+mymod.py+,然后里面定义了一个简单的myfun函数,没什么意义,就是打印了一段信息。

接下来我们看看运行效果如何:
\begin{xverbatim}[129]{py}
import os,sys
sys.path.append(os.environ['HOME']+'/pymf')
from pyconfig import *

import mymodule.mymod

mymodule.mymod.myfun()
\end{xverbatim}



\subsection{如果不是空白?}
\verb+__init__.py+文件如果是空白我们不用关心它到底起了什么作用,但如果它不是空白,里面有一些代码呢?那么我们就需要了解如下知识了。

每个模块包下的\verb+__init__.py+在python首次导入该目录时都会执行一次这个文件内部的代码。

比如上面的例子的那个\verb+__init__.py+文件我们随便写上一行打印信息命令,这样输出结果和上面一样的\footnote{上面代码的运行结果是即时编译生成的。},然后我们做了一些其他测试。
\begin{Verbatim}
mymodule already import
myfun is found
>>> mymodule.mymod.myfun()
myfun is found
>>> from imp import reload
>>> reload(mymodule)
mymodule already import
<module 'mymodule' from '/home/wanze/pymf/mymodule/__init__.py'>
>>> import mymodule.mymod
>>> 
\end{Verbatim}

\subsection{一般有什么}
\verb+__init__.py+文件里面一般有些什么内容?我们可以看看别人编写的模块包。一般是空白的,然后有的时候情况会很复杂,定义了一些类或者函数。然后有的时候只是简单的import语句或者from语句。

比如我现在在前面的例子的基础上修改\verb+__init__.py+文件,加上一句话:
\begin{tcbpython}
from mymodule import  mymod
\end{tcbpython}

那么上面的测试代码就可以写成这个样子了:
\begin{tcbpython}
import os,sys
sys.path.append(os.environ['HOME']+'/pymf')
from pyconfig import *

import mymodule
\end{tcbpython}

只需要简单的import mymodule即可,然后\verb+__init__.py+会自动from mymodule import mymod,也就是你的那个mymod.py文件,这样mymodule文件夹里面的一些子模块py文件你都可以引用了。如下所示:
\begin{Verbatim}
mymodule already import
>>> mymodule
<module 'mymodule' from '/home/wanze/pymf/mymodule/__init__.py'>
>>> mymodule.mymod
<module 'mymodule.mymod' from '/home/wanze/pymf/mymodule/mymod.py'>
>>> mymodule.mymod.myfun
<function myfun at 0xb70cde84>
>>> mymodule.mymod.myfun()
myfun is found
\end{Verbatim}

不得不承认我喜欢这种管理模式,这样的管理模式更加层次结构清晰。同时我们看到\verb+__init__.py+文件似乎就是对应的mymodule这个模块。

\subsection{from*全部导入}
\verb+__init__.py+文件还有一个高级功能,不过一般不建议使用这样的全部导入语句,但这种from*语句有一个好处,那就是主模块的名字你可以不用写了,然后\verb+__init__.py+文件里面你需要写上\verb+__all__+这个变量,后面跟上一个列表,列表里面的字符串就是你想要引入的子模块。

\begin{tcbpython}[]
#mymod.py
def myfun():
    print('myfun is found')
\end{tcbpython}

\begin{tcbpython}[]
#__init__.py
print('mymodule already import')

from mymodule import mymod
__all__ = ['mymod']
\end{tcbpython}
然后测试代码如下:
\begin{tcbpython}
import os,sys
sys.path.append(os.environ['HOME']+'/pymf')
from pyconfig import *

from  mymodule import *
\end{tcbpython}

这种写法测试结果中,mymodule是不能用了,而需要直接引用子模块的名字了:
\begin{Verbatim}
mymodule already import
>>> mymod
<module 'mymodule.mymod' from '/home/wanze/pymf/mymodule/mymod.py'>
>>> mymodule
Traceback (most recent call last):
  File "<stdin>", line 1, in <module>
NameError: name 'mymodule' is not defined
\end{Verbatim}

然后如果测试代码我们改成import语句,我们发现之前的那种逐步引用格式照样可行:
\begin{tcbpython}
import os,sys
sys.path.append(os.environ['HOME']+'/pymf')
from pyconfig import *

import  mymodule 
\end{tcbpython}
\begin{Verbatim}
mymodule already import
>>> mymodule
<module 'mymodule' from '/home/wanze/pymf/mymodule/__init__.py'>
>>> mymodule.mymod
<module 'mymodule.mymod' from '/home/wanze/pymf/mymodule/mymod.py'>
\end{Verbatim}


\section{模块中的帮助信息}
well,大家都知道,我们编写的模块主要是给别人看得,给别人用的,所以多写点帮助信息吧,这个没人嫌你写得多的。其他\#{}下的注释就不用说了,这里主要讲一下其他的帮助信息。

还是跟着上面的例子来:
\begin{tcbpython}
"""mymod.py
这是一个测试模块"""

def myfun():
    """myfun函数
    用于打印测试"""

    print('myfun is found')
\end{tcbpython}

\begin{tcbpython}
"""mymodule
我在mymodule文件夹的__init__.py文件里面"""

print('mymodule already import')

from mymodule import mymod
__all__ = ['mymod']
\end{tcbpython}

然后是测试代码:
\begin{tcbpython}
import os,sys
sys.path.append(os.environ['HOME']+'/pymf')
from pyconfig import *

import  mymodule
\end{tcbpython}


具体测试查看情况如下所示,其中help命令显示的内容就不粘贴在这里了,请读者自己查看之,还是很有意思的。
\begin{Verbatim}
mymodule already import
>>> dir(mymodule)
['__all__', '__builtins__', '__cached__', '__doc__', '__file__', '__loader__', '__name__', '__package__', '__path__', '__spec__', 'mymod']
>>> dir(mymodule.mymod)
['__builtins__', '__cached__', '__doc__', '__file__', '__loader__', '__name__', '__package__', '__spec__', 'myfun']
>>> print(mymodule.__doc__)
mymodule
我在mymodule文件夹的__init__.py文件里面
>>> print(mymodule.mymod.__doc__)
mymod.py
这是一个测试模块
>>> print(mymodule.mymod.myfun.__doc__)
myfun函数
    用于打印测试
>>> help(mymodule)

>>> help(mymodule.mymod)

>>> help(mymodule.mymod.myfun)

>>> 
\end{Verbatim}





\chapter{文件处理高级知识}
接下来的例子如果涉及到文件的请自己随便创建一个对应文件名的文件,内容随意了。


\section{一行行的操作}
因为文件对象本身是可迭代的,我们简单迭代文件对象就可以对文件的一行行内容进行一些操作。比如:
\begin{tcbpython}
f = open('removeduplicate.py')

for line in f:
    print(line,end='')
\end{tcbpython}
这个代码就将打印这个文件,其中end=''的意思是取消\verb+\n+,因为原来的行里面已经有\verb+\n+了。

然后代码稍作修改就可以在每一行之前加上>>>这个符号了。 
\begin{tcbpython}
f = open('removeduplicate.py')

for line in f:
    print('>>>',line,end='')
\end{tcbpython}

什么?这个输出只是在终端,没有到某个文件里面去,行,加上file参数。然后代码变成如下:
\begin{tcbpython}
import sys

f = open('removeduplicate.py')
pyout=open(sys.argv[1] ,"w")

for line in f:
    print('>>>',line,end='',file=pyout)

pyout.close()
f.close()
\end{tcbpython}
这样我们就制作了一个小python脚本,接受一个文件名然后输出这个文件,这个文件的内容就是之前我们在终端中看到的。

\section{整个文件的列表解析}
python的列表解析(迭代)效率是很高的,我们应该多用列表解析模式。

\subsection{readlines方法}
文件对象有一个readlines方法,能够一次性把整个文件的所有行字符串装入到一个列表中。然后我们再对这个列表进行解析操作就可以直接对整个文件的内容做出一些修改了。不过不推荐使用readlines方法了,这样将整个文件装入内存的方法具有内存爆炸风险,而迭代版本更好一点。

\subsection{文本所有某个单词的替换}
这里举一个例子,将removeduplicate.py文件接受进来,然后进行列表解析,将文本中的newlist全部都替换为list2。

\begin{tcbpython}
import sys

pyout=open(sys.argv[1] ,"w")

print(''.join([line.replace('newlist','list2') 
for line in open('removeduplicate.py')]),file=pyout)

pyout.close()
\end{tcbpython}

我们可以看到这种列表解析风格代码更加具有python风格和更加的简洁同时功能是异常的强大的。

从这里起我们看到如果需要更加复杂的文本处理技巧就需要学习正则表达式和re模块了,请参见re模块这一小节\ref{sec:re模块}。




\part{常用的模块}
\chapter{pickle模块}
\label{sec:pickle模块}
pickle模块可以将某\uwave{一个}复杂的对象永久存入\uwave{一个}文件中,以后再导入这个文件,这样自动将这个复杂的对象导入进来了。

\section{将对象存入文件}
\begin{Verbatim}
import pickle

class Test:
    def __init__(self):
        self.a=0
        self.b=0
        self.c=1
        self.d=1

    def __str__(self):
        return str(self.__dict__)

if __name__ == '__main__':
    test001=Test()
    print(test001)
    testfile=open('data.pkl','wb')
    pickle.dump(test001,testfile)
    testfile.close()
\end{Verbatim}




\section{从文件中取出对象}
值得一提的是从文件中取出对象,原来的类的定义还是必须存在,也就是声明一次在内存中的,否则会出错。
\begin{Verbatim}
import pickle

class Test:
    def __init__(self):
        self.a=0
        self.b=0
        self.c=1
        self.d=1

    def __str__(self):
        return str(self.__dict__)

if __name__ == '__main__':
    testfile=open('data.pkl','rb')
    test001=pickle.load(testfile)
    print(test001)
    testfile.close()
\end{Verbatim}

pickle模块的基本使用就是用dump函数将某个对象存入某个文件中,然后这个文件以后可以用load函数来加载,然后之前的那个对象会自动返回出来。

\begin{Large}
更多内容请参见\href{https://docs.python.org/3/library/pickle.html}{官方文档}。
\end{Large}

\chapter{shelve模块}
shelve模块是基于pickle模块的,也就是只有pickle模块支持的对象它才支持。 之前提及pickle模块只能针对一个对象,如果你有多个对象要处理,可以考虑使用shelve模块,而shelve模块就好像是自动将这些对象用字典的形式包装起来了。除此之外shelve模块的使用更加简便了。

\section{存入多个对象}
我们根据类的操作第三版中定义的类(\ref{sec:类的操作第三版})建立了一个Hero.py文件,就是将那些类的定义复制进去。然后我们新建了几个实例来存入test.db文件中。

\begin{tcbpython}
import shelve
from Hero import Garen

if __name__ == '__main__':
    garen1=Garen()
    garen2=Garen('red')
    garen3=Garen('yellow')
    db=shelve.open('test.db')
    for (key,item) in [('garen1',garen1),('garen2',garen2),('garen3',garen3)]:
        db[key]=item
    db.close()
\end{tcbpython}

我们看到整个过程的代码变得非常的简洁了,然后一个个对象是以字典的形式存入进去的。

\section{读取这些对象}
读取这些对象的代码也很简洁,就是用shelve模块的open函数打开数据库文件,open函数会自动返回一个字典对象,这个字典对象里面的数据就对应着之前存入的键值和对象。

同时通过这个例子我发现,如果自己定义的类,将他们提取出来放入另外一个文件,那么shelve模块读取文件时候是不需要再引入之前的定义。这一点值得我们注意,因为shelve模块内部也采用的是pickle的机制,所以可以猜测之前pickle的那个例子类的定义写在写入文件代码的里面,所以不能载入数据库;而如果将这些类的定义放入一个文件,然后这些类以模块或说模块载入的形式引入,那么读取这些对象就可以以一种更优雅的形式实现。如下所示:
\begin{tcbpython}
import shelve

if __name__ == '__main__':
    db=shelve.open('test.db')
    for key in sorted(db):
        print(db[key])
    db.close()
\end{tcbpython}

我们看到就作为简单的程序或者原型程序的数据库,shelve模块已经很好用而且够用了。

\begin{Large}
更多内容请参见\href{https://docs.python.org/3/library/shelve.html}{官方文档}。
\end{Large}

\subsection{zodb模块}


\chapter{time模块}
time模块提供了一些和时间相关的函数,更加的底层,不过有些函数可能在某些平台并不适用。类似的模块还有datetime模块,datetime是以类的框架来解决一些时间问题的。所以如果只是需要简单的调用一下时间,那么用time模块,如果是大量和时间相关的问题,推荐使用datetime模块。

\section{time函数}
\begin{Verbatim}
>>> import time
>>> time.time()
1404348227.07554
\end{Verbatim}
time函数返回一个数值,这个数值表示从1970年1月1号0时0分0秒到现在的时间过了多少秒。

\section{gmtime函数}
这个函数可以接受一个参数,这个参数是多少秒,然后返回一个特定格式的时间数组\verb+struct_time+。如果不接受参数,那么默认接受的秒数由time函数返回,也就是从那个特定时间到现在过了多少秒,这样这个特定格式的时间数组对应的就是当前时间。

\begin{Verbatim}
>>> time.gmtime()
time.struct_time(tm_year=2014, tm_mon=7, tm_mday=3, tm_hour=0, 
tm_min=53, tm_sec=0, tm_wday=3, tm_yday=184, tm_isdst=0)
>>> time.gmtime(0)
time.struct_time(tm_year=1970, tm_mon=1, tm_mday=1, tm_hour=0, 
tm_min=0, tm_sec=0, tm_wday=3, tm_yday=1, tm_isdst=0)
\end{Verbatim}

\section{localtime函数}
此外类似的还有\textbf{localtime}函数,和gmtime用法和返回完全一模一样,唯一的区别就是返回的是当地的时间。
\begin{Verbatim}
>>> time.strftime('%Y-%m-%d %H:%M:%S',time.localtime())
'2014-07-03 09:19:40'
>>> time.strftime('%Y-%m-%d %H:%M:%S',time.gmtime())
'2014-07-03 01:19:49'
\end{Verbatim}


\section{ctime函数}
\begin{Verbatim}
>>> time.ctime()
'Thu Jul  3 08:54:54 2014'
>>> time.ctime(0)
'Thu Jan  1 07:00:00 1970'
\end{Verbatim}
和gmtime类似,不过返回的是字符串格式的时间。我们看到ctime默认设置的时间是根据localtime函数来的。


\section{strftime函数}
接受那个特定格式的时间数组\verb+struct_time+作为参数,然后返回一定字符串格式的时间。具体例子请参看前面的例子。

其中最常用的格式符有:
\begin{Verbatim}
%Y,多少年;%m,多少月;%d,多少日;
%H,多少小时;%M,多少分;%S,多少秒。
\end{Verbatim}

\%{}X直接输出09:27:19这样的格式,也就是前面的多少小时多少分多少秒可以用一个\%{}X表示即可。

还有一些,比如:\%{}I表示多少小时,不过是[0-12]的形式;\%{}y表示多少年,不过是[00-99]的格式,比如2014年就输出14;\%{}p,本地的AM或PM文字。等等。


\section{sleep函数 }
sleep函数有时需要用到,将程序休眠个几秒的意思。需要接受一个数值参数,单位是秒,可以是零点几秒。但sleep函数只是大概休眠几秒的意思,最好不去用来计时,因为它不大精确。


\begin{Large}
更多内容请参见\href{https://docs.python.org/3/library/time.html}{官方文档}。
\end{Large}

\chapter{sys模块}
sys模块有一些功能很常用,其实在前面我们就看到过一些了。

\section{sys.argv}
在刚开始说明python执行脚本参数传递的问题时就已经讲了sys.argv这个变量。这是一个由字符串组成的列表。
\begin{tcbpython}
import sys

print(sys.argv)
for i in range(len(sys.argv)):
    print(sys.argv[i])
\end{tcbpython}
比如新建上面的一个test.py文件,然后执行:
\begin{Verbatim}
python3 test.py test1 test2
['test.py', 'test1', 'test2']
test.py
test1
test2
\end{Verbatim}
我们可以看到sys.argv[0]就是这个脚本的文件名,然后后面依次是各个参数。

\section{exit函数}
这个我们在编写GUI程序的时候经常看到,在其他脚本程序中也很常用。如果不带参数的话那么直接退出程序,还可以带一个字符串参数,返回错误提示信息,或者带一个数字,这里的详细讨论略过。

\begin{tcbbash}[]
>>> import sys
>>> sys.exit('出错了')
出错了
wanze@wanze-ubuntu:~$ 
\end{tcbbash} 
%$

\section{sys.platform}
返回当前脚本执行的操作系统环境。

Linux 返回字符串值:linux;Windows返回win32;Mac OS X 返回darwin。

\section{sys.path}
一连串字符串列表,是python脚本模块的搜索路径,所以我们自定义的python模块,只需要在sys.path这个列表上新加一个字符串路径即可。

\section{标准输入输出错误输出文件}
sys.stdin,sys.stdout,sys.stderr这三个文件对象对应的就是linux系统所谓的标准输入标准输出和错误输出文件流对象。

\section{sys.version}
sys.version输出当前python的版本信息和编译环境的详细信息。

\mint{bash}+sys.version_info[0]+返回当前python主版本的标识,比如python3就返回数字3。

\section{sys.maxsize}
返回当前计算环境下整数(int)类型的最大值,32位系统是$2**31-1$。
>>> 2**31-1
2147483647
>>> import sys
>>> sys.maxsize
2147483647
>>> 


\begin{Large}
更多内容请参见\href{https://docs.python.org/3/library/sys.html}{官方文档}。
\end{Large}


\chapter{os模块}
\section{getcwd函数}
不管你在终端运行python还是运行某个python脚本,总有一个变量存储着当前工作目录的位置。你可以通过getcwd命令来查看当前工作目录。

\begin{xverbatim}[129]{py}
import os
print(os.getcwd())
\end{xverbatim}
上面是通过\LaTeX 文件运行的python小脚本,当你以python命令来运行某个脚本的时候,你调用python命令的地方就是当前的工作目录\footnote{这里在\LaTeX 文档下的情况有点小复杂,通过我编写的xverbatim.sty我们可以看到当时运行python3命令的当前工作目录就在这个tex文档所在的目录下。}。然后加载的其他模块的各个py文件运行时的当前工作目录和主py文件脚本的当前目录是一样的,都是你运行python命令的地方。

如果是终端调用python就是你终端的当前工作目录所在,你可以用pwd命令来查看。如下所示:
\begin{Verbatim}
=>pwd
/home/wanze
=>python3
>>> import os
>>> print(os.getcwd())
/home/wanze
\end{Verbatim}

\section{chdir函数}
os模块里有一个chdir函数来更改当前工作目录所在地。
\begin{Verbatim}
>>> os.chdir('/home/wanze/pymf')
>>> print(os.getcwd())
/home/wanze/pymf
\end{Verbatim}


\section{environ函数}
os.environ,返回一个字典值,这个字典值里面存储着当前shell的一些变量和值。比如系统中“HOME”所具体的路径名是:
\begin{tcbpython}[]
import os
print(os.environ['HOME'])
\end{tcbpython}
\begin{Verbatim}
/home/wanze
>>> 
\end{Verbatim}

\section{getpid函数}
os.getpid函数,返回当前运行的进程的pid。

\section{stat函数}
返回文件的一些信息。比如st\_{}size是文件的大小,单位是字节。

\begin{tcbpython}[]
import os
import glob

print([os.path.abspath(f) for f in glob.glob('*.py')])

print([f for f in glob.glob('*.py') if os.stat(f).st_size > 400])
\end{tcbpython}
\begin{Verbatim}
['/home/wanze/桌面/test.py', '/home/wanze/桌面/flatten.py']
['flatten.py']
\end{Verbatim}

\subsection{文件大小单位优化}
\begin{tcbpython}[]
import os
import sys

filename = sys.argv[1]

filesize = os.stat(filename).st_size


filename = sys.argv[1]

filesize = os.stat(filename).st_size

for unit in ['字节','KB','MB','GB','TB']:
    if filesize > 1024:
        filesize = filesize/1024
    else:
        break

print(filename + '大小是' +str(int(filesize)) + unit)
\end{tcbpython}
这个python小脚本自动输出合适的单位,具体程序逻辑还是很简单的。

\chapter{os.path模块}
前面提到sys.argv只能返回当前python脚本的文件名,而我们常常需要这个python脚本在系统中的具体位置。前面如os.getcwd等也能获得当前python脚本的所在目录,不过os.path模块的一个有点就是跨平台特性支持很好,也就是一般我们通过其他方式获得的path路径都会用这个模块的函数辅助处理一下。

我们来看下面的例子:
\begin{tcbpython}[]
import os

print(os.path.abspath(__file__))
print(os.path.dirname(os.path.abspath(__file__)))

print(os.path.basename(__file__))
print(os.path.basename(os.environ['HOME']))

\end{tcbpython}
\begin{Verbatim}
/home/wanze/桌面/test.py
/home/wanze/桌面
test.py
wanze
>>> 
\end{Verbatim}

其中\verb+__file__+表示当前文件,然后接下来的一些命令我们后面慢慢讲。

\section{abspath函数}
abspath函数接受一个path路径值然后返回一个正规的普适的路径地址。比如对于当前文件来说\footnote{文件名也是一个路径},就是当前文件的具体路径地址,这个和getcwd的python的当前工作目录是不同的,这里的\verb+__file__+就是命令所在的具体的当前的文件地址。

再看下面的例子演示了空字符串默认当前工作目录,然后也接受绝对路径等。
\begin{Verbatim}
>>> import os
>>> os.path.abspath('')
'/home/wanze'
>>> os.path.abspath('test')
'/home/wanze/test'
>>> os.path.abspath('/test')
'/test'
>>> os.path.abspath('test/')
'/home/wanze/test'
\end{Verbatim}

\section{dirname函数}
dirname函数接受一个路径值然后 返回这个路径除开最后一个元素的前面的路径值。比如上面的例子,路径指向文件,那么dirname函数返回的是除开这个文件名的前面的路径;而如果接受的路径指向目录,那么返回的是除开最后一个文件夹名的前面的路径值。

\section{basename函数}
如上面例子所示,basename函数接受一个路径值然后返回路径的最后一个元素,如果路径指向文件,那么返回的是文件名;如果路径指向目录,那么返回的是最后那个目录的文件夹名。比如下面实现了从绝对路径提取出文件名的功能。
\begin{Verbatim}
>>> import os.path
>>> string = '/home/wanze/test.txt'
>>> fileName,fileExtension = os.path.splitext(os.path.basename(string))
>>> fileName
'test'
\end{Verbatim}

\section{split函数}
将路径path字符串分割,可以视作dirname和basename的组合。
\begin{Verbatim}
>>> os.path.split('/usr/local/bin/test.txt')
('/usr/local/bin', 'test.txt')
>>> os.path.dirname('/usr/local/bin/test.txt')
'/usr/local/bin'
>>> os.path.basename('/usr/local/bin/test.txt')
'test.txt'
\end{Verbatim}



\section{splitext函数}
将某个路径path的后缀分开,这里主要是针对文件名为输入的时候,那么第一个为该文件的名字,输出数组的第二个值是该文件的后缀。这个函数在提取文件名后缀和前面的名字的时候很有用,方便组合出新的文件名。
\begin{Verbatim}
>>> import os
>>> fileName, fileExtension = os.path.splitext('/path/to/somefile.ext')
>>> fileName
'/path/to/somefile'
>>> fileExtension
'.ext'
\end{Verbatim}


\section{join函数}
用于连接多个路径值合并成一个新的路径值,同样相对于简单的字符串拼接,用这个函数处理路径组合具有操作系统普适性和灵活性。
\begin{Verbatim}
>>> os.path.join(os.path.expanduser('~'),'test','lib')
'/home/wanze/test/lib'
\end{Verbatim}

上面join函数多个参数生成的新path在windows下又是不同的输出的。



\section{expanduser函数}
\begin{Verbatim}
>>> import os
>>> os.path.expanduser('~')
'/home/wanze'
>>> os.path.expanduser('~/pymf')
'/home/wanze/pymf'
>>> os.path.join(os.path.expanduser('~'),'pymf','mymodule')
'/home/wanze/pymf/mymodule'
\end{Verbatim}

\verb+~+这个符号可以在这里使用,从而展开为以/home/wanze为基础的绝对路径,兼容大部分系统(在windows下也可以使用。)

同时我们看到join函数可以接受很多不定量的参数,然后将他们组合成为一个新的路径,而且不用你费心是\verb+/+还是\verb+\+,你不需要写这些了,用join函数自然料理好一切。



\section{exists函数}
os.path.exists(path):测试路径或文件等是否存在。如果存在返回True,否则返回False。

\section{isfile和isdir还有islink}
os.path.isfile(path):接受一个字符串路径变量,如果是文件那么返回True,否则返回False(也就是文件不存在或者不是文件是文件夹等情况都会返回False)。

类似的有isdir和islink函数。


\section{samefile函数}
os.path.samefile(path1,path2):如果两个文件或路径相同则返回True\\,否则返回False。


\begin{Large}
更多内容请参见\href{https://docs.python.org/3/library/os.path.html}{官方文档}。
\end{Large}


\chapter{glob模块}
glob模块用法很简单,初步学习就是一个glob函数,接受一个pathname路径值,然后返回这个路径下某些文件名组成的列表。支持\verb+* ? +,意思是任意数量的字符或者任意的一个字符,然后\verb+[?]+明确表示问号。
\begin{Verbatim}
>>> import glob
>>> glob.glob('*.py')
['re_subst.py', 're_sub.py', 'test2.py']
\end{Verbatim}




\chapter{subprocess模块}
我想大家都注意到了现在的计算机都是多任务的,这种多任务的实现机制就是所谓的多个进程同时运行,因为计算机只有一个CPU(现在多核的越来越普及了,它们内部的工作原理我没了解过。)所有计算机一次只能处理一个进程,而这种多进程的实现有点类似你人脑(当然不排除某些极个别现象),你不能一边看电影一边写作业,但是可以写一会作业然后再看一会电影(当然不推荐这么做、),计算机的多进程实现机制也和这个类似,就是一会干这个进程,一会儿做那个进程。

计算机的一个进程里面还可以分为很多个线程,这个较为复杂,就不谈了。比如你编写的一个脚本程序,系统就会给它分配一个进程号之类的,然后cpu有时就会转过头来执行它一下(计算机各个进程之间的切换很快的,所以才会给我们一种多任务的错觉。)而你的脚本程序里面还可以再开出其他的子进程出来, python的subprocess模块主要负责这方面的工作。

\section{call函数}
\begin{tcbpython}[]
import subprocess

# Command with shell expansion
subprocess.call(["echo", "hello world"])
subprocess.call(["echo", "$HOME"])
subprocess.call('echo $HOME',shell=True)
\end{tcbpython}
\begin{Verbatim}
hello world
$HOME
/home/wanze
\end{Verbatim}
%$

其中使用shell=True选项后用法较简单较直观,但网上提及安全性和兼容性可能有问题,他们推荐一般不适用shell=True这个选项。\sidenote{\href{http://stackoverflow.com/questions/3172470/actual-meaning-of-shell-true-in-subprocess}{参考网站}}

如果不使用shell=True这个选项的,比如这里\verb+$HOME+这个系统变量就无法正确翻译过来,如果实在需要home路径,需要使用os.path的expanduser函数。



\section{getoutput函数}
取出某个进程命令的输出,返回的是字符串形式。
\begin{Verbatim}
import subprocess

name=subprocess.getoutput('whoami')
print(name)
\end{Verbatim}



\section{getstatusoutput函数}
某个进程执行的状态。



\section{Popen类}
根据Popen类创建一个进程管理实例,可以进行进程的沟通,暂停,关闭等等操作。前面的函数的实现是基于Popen类的,这是较高级的课题,这里暂时略过。


\begin{Large}
更多内容请参见\href{https://docs.python.org/3/library/subprocess.html}{官方文档}。
\end{Large}


\chapter{collections模块}
\section{namedtuple函数}
collections模块里面的namedtuple函数将会产生一个有名字的数组的类(有名数组),通过这个类可以新建类似的实例。比如:
\begin{xverbatim}[129]{py}
from collections import namedtuple

Point3d=namedtuple('Point3d',['x','y','z'])
p1=Point3d(0,1,2)
print(p1)
print(p1[0],p1.z)
\end{xverbatim}

\begin{Large}
更多内容请参见\href{https://docs.python.org/3/library/collections.html}{官方文档}。
\end{Large}


\section{队列对象}


\chapter{re模块}
\label{sec:re模块}
re模块提供了python对于正则表达式的支持,对于字符串操作,如果之前在介绍字符串类型的一些方法(比如split,replace等等),能够用它们解决问题就用它们,因为更快更简单。实在需要动用正则表达式理念才考虑使用re模块,而且你要克制写很多或者很复杂的(除非某些特殊情况)正则表达式的冲动,因为正则表达式的引入将会使得整个程序都更加难懂和不可捉摸。


\begin{Large}
更多内容请参见\href{https://docs.python.org/3/library/re.html}{官方文档}。
\end{Large}

\section{re模块中的元字符集}
\begin{description}
\item[\emph{.}] 表示一行内的任意字符,如果如果通过re.compile指定\textbf{re.DOTALL},则表示多行内的任意字符,即包括了换行符。此外还可以通过字符串模板在它的前面加上\textbf{(?s)}来获得同样的效果。
\item[\emph{*}] 对之前的字符匹配\uwave{零次}或者多次。
\item[\emph{+}] 对之前的字符匹配\uwave{一次}或者多次。
\item[\emph{?}] 对之前的字符匹配\uwave{零次}或者\uwave{一次}。
\item[\emph{\{m\}}] 对之前的字符匹配(\uwave{exactly})m次。
\item[\emph{\{m,n\}}] 对之前的字符匹配m次到n次,其中n次可能省略,视作默认值是无穷大。
\item[\emph{\^{}}] 表示字符串的开始,如果加上\textbf{re.MULTILINE}选项,则表示行首。此外字符串模板加上\textbf{(?m)}可以获得同样的效果。
\item[\emph{\${}}] 表示字符串的结束,同\^{}类似,如果加上\textbf{re.MULTILINE}选项,则表示行尾。此外字符串模板加上\textbf{(?m)}可以获得同样的效果。
\item[\emph{[]}] [abc]字符组匹配一个字符,这个字符是a或者b或者c。类似的[a-z]匹配所有的小写字母,\verb+[\w]+匹配任意的字母或数字,具体请看下面的特殊字符类。
\item[\emph{|}] 相当于正则表达式内的匹配或逻辑。
\item[\emph{()}] 圆括号包围的部分将会记忆起来,方便后面调用。这个后面在谈及。
\end{description}


\section{re模块中的特殊字符类}
\begin{Verbatim}
\w  任意的字母或数字  [a-zA-Z0-9_]  (meaning word)
\W  匹配任何非字母非数字 [^a-zA-Z0-9_]
\d   [0-9]   (digit) 数字
\D  [^0-9] 非数字
\s   匹配任何空白字符   [ \t\n\r\f\v] 。
\S  匹配任何非空白字符
匹配中文:[\u4e00-\u9fa5]
\end{Verbatim}

其中\^{}在方括号[]里面,只有在最前面,才表示排除型字符组的意思。


\section{转义问题}
正则表达式的转义问题有时会比较纠结。一个简单的原则是以上谈及的有特殊作用的字符有转义问题,如果python中的字符都写成\verb+r''+这种形式,也就是所谓的raw string形式,这样\verb+\n+在里面就可以直接写成\verb+\n+,而\verb+\section+可以简单写为\verb+\\section+即可,也就是\verb+\+字符需要转义一次。

然后字符组的方括号内[]有些字符有时是不需要转义的,这个实在不确定就转义吧,要不就用Kiki测试一下。

\section{re模块的使用}
compile方法生成regular expression object这一条线这里略过了,接下来的讨论全部基于(原始的)字符串模板。

字符串模板前面提及(?m)和(?s)的用法了,然后\textbf{(?i)}表示忽略大小写。

\subsection{匹配和查找}
search,match方法简单地用法就是:
\begin{Verbatim}
re.search(字符串模板, 待匹配字符串)
re.match(pattern, string)
\end{Verbatim}

它们将会返回一个match object或者none,其中match object在逻辑上就是真值的意思。match对字符串的匹配是必须从一开始就精确匹配,这对于正则表达式多少0有点突兀。推荐使用search方法,如果一定要限定行首,或者字符串开始可以用前面讨论的正则表达式各个符号来表达。请看下面的例子。

\begin{tcbpython}[]
import re
string = '''this is test line.
this is the second line.
today is sunday.'''

match = re.search('(?m)^today',string)

if match:
    print('所使用的正则表达式是:',match.re)
    print('所输入的字符串是:',match.string)
    print('匹配的结果是:',match.group(0))
    print('匹配的字符串index',match.span())
else:
    print('return the none value')
\end{tcbpython}

前面说道圆括号的部分将会记忆起来,作为匹配的结果,默认整个正则表达式所匹配的全部是group中的第0个元素,然后从左到右,子group编号依次是1,2,3......。

\begin{Verbatim}
所使用的正则表达式是: re.compile('(?m)^today', re.MULTILINE)
所输入的字符串是: this is test line.
this is the second line.
today is sunday.
匹配的结果是: today
匹配的字符串index (44, 49)
\end{Verbatim}

具体这些信息是为了说明情况,实际最简单的情况可能就需要判断一下是不是真值,字符串模板是不是匹配到了即可。


\subsection{分割操作}
re模块的split函数可以看作字符串的split方法的升级版本,对于所描述的任何正则表达式,匹配成功之后都将成为一个分隔符,从而将原输入字符串分割开来。

下面是我写的zwc小脚本的最核心的部分,用途是统计中英文文档的具体英文单词和中文字符的个数。其中最核心的部分就是用的re的split函数进行正则表达式分割,如果不用那个圆括号的话,那么分隔符是不会包含进去的,这里就是具体匹配的中文字和各个标点符号等等。用了圆括号,那么圆括号匹配的内容也会进去列表。这里就是具体的各个分隔符。

\begin{tcbpython}[]
import re

def zwc(string):
    #中英文常用标点符号
    lst = re.split('([\u4e00-\u9fa5\s,。;])',string)
    #去除 空白
    #去除\s 中英文常用标点符号
    lst = [i for i in lst if not  i in
    [""," ","\n","\t","\r","\f","\v",";",",","。"]]
    print(lst)

if __name__ == '__main__':
    string='''道可道,非常道。名可名,非常名。無名天地之始,有名萬物之母。
    故常無欲,以觀其妙;常有欲,以觀其徼。此兩者同出而異名,
    同謂之玄,玄之又玄,眾妙之門。 '''
    zwc(string)
\end{tcbpython}

字符分割之后后面做了一个小修正,将匹配到的空白字符和中英文标点符号等都删除了,这些是不应该统计入字数的。

具体这个github项目链接在这里:\href{https://github.com/a358003542/zwc}{zwc项目}。


\subsection{替换操作}
基于正则表达式的替换操作非常的有用,其实前面的search方法,再加上具体匹配字符串的索引值,然后修改原字符串,然后再search这样循环操作下去,就是一个替换操作了。re模块有sub方法来专门解决这个问题。

让我们为Linux系统写一个resub命令,这个命令的用途就是将某一个标准输入流或者utf-8文本文件按照你定义的正则表达式规则,依次完成一个\footnote{这里简单起见就是一个,多个情况可以考虑编写另外一个程序来控制之。}正则表达式文本替换工作。这个命令在我们需要对某个utf-8文本文件进行某个你想要的——非简单的精确相同匹配然后替换操作时——特别有用。

为了作为程序的检验,这里提出两个任务:第一个任务是我们在ocr PDF文档之后的输出,经常发现很多标点符号问题,这些需要人手工修改会非常的耗费精力。其中第一个问题如下,"这是一段文字"需要替换成为“这是一段文字”。这个例子之所以特别是因为中文的双引号是分左和右的,这里必须要用正则表达式匹配和替换;第二个任务更加的复杂,那就是从排版角度上讲,如果括号里面的文字都是英文或者数字,那么就使用英文的括号(),如果括号里面有中文或者全是中文\footnote{这里程序的逻辑是都换成中文的全角括号(毕竟中文unicode码具体范围的判断是不太精确的),只有那些纯英文纯数字或者基本英文标点和其他简单符号的再换成英文括号},那么就使用中文的括号()。ocr出来或者甚至人编写的文档都常常难以做到没有瑕疵,第二个任务就是通过resub命令来确保之后的输出文档的括号满足这一要求。

然后程序还需要建立两个选项,一个是自动替换所有,一个是对于每一个替换操作都请求确认——需要打印相关信息。

程序需要经过如下几个阶段:1.明确匹配模板  1.1写出字符串模板 匹配操作 给出匹配的所有情况,最好是行模板匹配模式。  最后明确匹配情况 2.明确匹配的文字的后给出情况 

\chapter{datetime模块}

\chapter{calendar模块}


学编程学到后面就需要学习和了解Linux系统相关知识,从Linux系统的架构可以学到很多软件架构的知识,熟悉了解Linux系统能够让我们做出更多基于Linux的实用的程序。

相关更多内容请参看我编写的指尖上的Ubuntu一书。具体github地址是:\\
\href{https://github.com/a358003542/finger-on-ubuntu}{https://github.com/a358003542/finger-on-ubuntu}

python脚本系统级编程的一大好处就是你经过一些技艺的练习写出来的python系统脚本是可以做到跨平台兼容的。(目前还不清楚具体如何?)



当前工作目录,os.getcwd ,  所谓当前工作目录是指python命令运行的地方,而不是指运行的python脚本所在地。而sys.path的第一个搜索路径是当前脚本所在地。如果你只输入文件名而不输入路径,那么完整的文件名是当前工作目录加上文件名,而import则还是第一搜索这个脚本的所在地。



os.environ  当前shell的环境变量字典。


标准流  sys sys.stdin sys.stdout sys.stderr  标准流对象就是python中的文件对象。

遍历文件树  os.walk   os.listdir

平行计算  ——多进程 管理 os.fork   这个先略过

线程 略过

sys.exit()            
\begin{Verbatim}
echo $status
\end{Verbatim}
%$


 




\chapter{getopt模块}




\part{常用的第三方模块}
\chapter{setuptools模块}
setuptools模块方便python用户快速分发自己写的python模块,特别是对于那些对其他模块有依存关系的模块。

安装就是先安装pip3:
\begin{Verbatim}
sudo apt-get install python3-pip
\end{Verbatim}


然后通过pip3来安装setuptools:
\begin{Verbatim}
sudo pip3 install setuptools
\end{Verbatim}



\chapter{pillow模块}
pillow模块让python具备的初步的图片处理能力。

\section{安装}
推荐使用pip3命令安装pillow模块。

\begin{tcbbash}[]
sudo pip3 install pillow
\end{tcbbash}

此外可能还需要其他宏包来支持一些图片格式,安装官方文档的叙述\footnote{also reference \href{http://askubuntu.com/questions/427358/install-pillow-for-python-3}{this}},下面这些可能需要装上。
\begin{tcbbash}[]
sudo apt-get install python3-dev python3-setuptools
sudo apt-get install libtiff4-dev libjpeg8-dev zlib1g-dev \
    libfreetype6-dev liblcms2-dev libwebp-dev tcl8.5-dev tk8.5-dev
\end{tcbbash}

\subsection{测试安装情况}
官方文档并没有给出测试安装情况的例子,这里给出刚开始那个简单的例子权做测试安装情况的测试。

\begin{tcbpython}[]
from PIL import Image
img = Image.open("test.jpg")
print(img.format, img.size , img.mode)
img.show()
\end{tcbpython}

这个例子的顺利运行需要你找一个jpg图片文件。pillow模块在这里名字叫PIL是因为它的父亲是PIL,pillow fork自它,并setuptools兼容。

我们看到pillow模块的语法还是很清晰的,Image是个类,open方法返回具体的jpg img 对象,这里就简单称作img对象了,img对象有format, size , mode方法,分别是图片的格式(JPEG),图片的尺寸((1920, 1080))和图片的模式(RGB)。然后img对象调用show方法会(通过系统内部工具)显示图片。

open方法就目前按照官方文档的叙述支持如下格式:bmp,eps,gif,im,jpeg,jpeg2000,msp,pcx,png,ppm ,spider,tiff,webp,xbm,xv。还有一些格式要某只支持读要某只支持写这里不说明了。其中写的时候save方法需要明确说明图片的目标格式,而open方法打开的时候图片名字是随意的,pillow会自动检测图片的格式。

如果open方法图片打开失败,将会返回\textbf{IOError}异常。

\section{图片格式转换器}
根据图片格式转换器这个例子简单说明pillow的图片格式转换功能。


\href{https://github.com/a358003542/imgformat}{imgformat小工具}


\chapter{matplotlib模块}
\section{安装}
\begin{tcbbash}[]
sudo apt-get install python3-matplotlib
\end{tcbbash}

\section{ipython下交互绘图}
输入\verb+%pylab+即可。


\begin{tcbpython}[]
import matplotlib.pyplot as plt
\end{tcbpython}
pyplot里面有很多基本的绘图函数,一般都如上引入为"plt"名字。

\section{plt子模块}
\subsection{plot方法}
plt.plot([1,2,3,4])

plot方法,接受一系列数字的列表,绘出线状图\footnote{更确切的描述是在默认画布上加入这些点。}。

如果只接受一个列表数字作为参数,那么认为是y轴上的数值,x轴值取默认值[0,1,2......]

如果接受两个列表数字作为参数,那个第一个认为是x轴上的数值,第二个是y轴上的数值。

\subsection{标签}
plt的xlabel方法,ylabel,控制图形的x轴标签和y轴标签。

\subsection{show方法}
plt的show方法,让图形绘制出来。

\subsection{axis方法}
plt的axis方法控制图形的x轴y轴具体显示的范围。

plt.axis([0,10,0,10])   axis方法接受的列表参数意义是 [xmin,xmax,ymin,ymax]

\subsection{hist方法}



\chapter{numpy模块}
\section{安装}
安装使用类似的语法:
\begin{Verbatim}
sudo apt-get install python3-numpy
\end{Verbatim}

\section{引入规则}
\begin{tcbpython}[]
import numpy as np
\end{tcbpython}

一般numpy引入语句如上所示,这已经是个惯例了,下面这个语句就省略了。

\section{ndarray对象}
\begin{Verbatim}
>>> x = np.array([1,2,3,4,5,6])
>>> print(x)
[1 2 3 4 5 6]
>>> type(x)
<class 'numpy.ndarray'>
\end{Verbatim}

如上所示numpy模块的array方法接受列表之后会将其变成numpy自己的ndarray对象,包括matplotlib模块等在内,它们都使用numpy模块的ndarray对象来表示和运算(类似列表的)数据。

\section{arange函数}
arange(start,end,step)  参数类似range函数。生成numpy的ndarray列表,如下所示:
\begin{Verbatim}
>>> x = np.arange(1,10,0.5)
>>> type(x)
<class 'numpy.ndarray'>
>>> x
array([ 1. ,  1.5,  2. ,  2.5,  3. ,  3.5,  4. ,  4.5,  5. ,  5.5,  6. ,
        6.5,  7. ,  7.5,  8. ,  8.5,  9. ,  9.5])
\end{Verbatim}

\section{histogram函数}


\section{多维ndarray}
\begin{Verbatim}
y=np.array([[1,2,3,4,5],[6,7,8,9,10]])
y,y[0][1],y[0]
\end{Verbatim}


\section{shape属性}
ndarray对象有一个shape属性,表示几行几列。
\begin{Verbatim}
y.shape
\end{Verbatim}

\section{dtype属性}
ndarray对象有一个dtype属性,表示存储相同单元的数据类型。

dtype属性还有小属性itemsize
\begin{Verbatim}
In [1]: import numpy as np
In [2]: a=np.arange(5)
In [3]: a
Out[3]: array([0, 1, 2, 3, 4])
In [4]: a.dtype
Out[4]: dtype('int32')
In [5]: a.dtype.itemsize
Out[5]: 4
\end{Verbatim}




\section{in语句}
in语句测量某元素是不是在这个ndarray对象中。
\begin{Verbatim}
8 in y, 11 in y
\end{Verbatim}

\section{reshape方法}
\begin{Verbatim}
In [8]: ndarray001=np.arange(1,10).reshape(3,3)
In [9]: ndarray001
Out[9]: 
array([[1, 2, 3],
       [4, 5, 6],
       [7, 8, 9]])
\end{Verbatim}




\section{copy方法}



\section{ndarray变成list}



\section{flatten方法}


\section{resize方法}


\section{transpose方法}


\section{eye方法}
创造单位矩阵

\begin{Verbatim}
In [10]: ndarray001=np.eye(3)
In [11]: ndarray001
Out[11]: 
array([[ 1.,  0.,  0.],
       [ 0.,  1.,  0.],
       [ 0.,  0.,  1.]])

\end{Verbatim}


\section{读写文件}
\begin{Verbatim}
In [10]: ndarray001=np.eye(3)
In [12]: np.savetxt("ndarray001.txt",ndarray001)
In [13]: x=np.loadtxt("ndarray001.txt")
In [14]: x
Out[14]: 
array([[ 1.,  0.,  0.],
       [ 0.,  1.,  0.],
       [ 0.,  0.,  1.]])
\end{Verbatim}





\part{tkinter教程}
以后GUI的设计会更加立足简单原生GUI,tkinter这部分最后可能会取代pyqt的位置。

tkinter的界面和pyqt相比确实丑了一点,不过tkinter的优点是快速原型开发,而pyqt则多少有点过于庞大和复杂了。快速开发, 我想这正是python选择tkinter作为其默认GUI模块的原因,如果没有额外的理由,应该使用tkinter来进行你的快速GUI开发。

\chapter{第一步}
\section{安装}
在ubuntu下你可能已经安装了tkinter模块了,不管怎么样运行下面的命令确保tkinter安装上:
\begin{tcbbash}[]
sudo apt-get install python3-tk
\end{tcbbash}

\subsection{测试安装情况}
\begin{tcbpython}[]
from tkinter import *
tk=Tk()
Label(tk,text='Label 的text').pack()
tk.mainloop()
\end{tcbpython}

第一个语句是用Tk类创建了一个母窗体。

第二个语句Label就是一个标签类,其中第一个参数是母窗体,text可选参数控制标签的文本,一般标签的这个text选项都会填上一些内容吧。pack方法是控制这个窗体在母窗体的布局,如果不使用这个方法窗体不会在母窗体中显示。

tk母窗体的mainloop方法是开启GUI的事件驱动。

\chapter{窗体类型}
\section{Label}
标签  Label(母窗体, ...)

text =  标签文字



\section{Button}
按钮 Button(母窗体, ...)

text = 按钮文字

\section{Entry}
单行输入框 Entry(母窗体, ...)

get()  get方法,输入框对象调用get方法之后返回输入框内的字符串。


\section{Frame}
可以简单称之为面板 Frame(母窗体, ...)

width = 面板宽
height = 面板高


\section{Combobox}


\chapter{布局管理}
\section{PACK方法}
side可选参数接受如下参数:\textbf{TOP},\textbf{BOTTOM},\textbf{LEFT},\textbf{RIGHT}。这样控制子窗体在母窗体中的布局位置。\uwave{默认是TOP}。

母窗体下面的子窗体如果 

\section{grid方法}
看名字就知道让子窗体在母窗体服从网格布局的意思。




\part{pyqt4教程}
本部分主要参考资料:\\
1.pyqt4教程,\href{http://blog.cx125.com/books/PyQt4_Tutorial/}{http://blog.cx125.com/books/PyQt4\_{}Tutorial/}

2.Prentice Hall,Rapid GUI Programming with Python and Qt,2007年10月

3.http://pyqt.sourceforge.net/Docs/PyQt4/classes.html

4.http://straightedgelinux.com/blog/python/html/pyqtxt.html



\chapter{刚开始}

\section{安装pyqt4}
ubuntu下安装pyqt4即安装python3-pyqt4即可:
\begin{tcbbash}[]
sudo apt-get install python3-pyqt4
\end{tcbbash}

此外本文后面会谈到用qt designer来辅助设计GUI,你还需要额外安装qt designer软件和pyuic4和pyrcc4命令。(顺便再次提醒下pyrcc4对中文目录目前支持有问题(201410)) 
\begin{tcbbash}[]
sudo apt-get install pyqt4-dev-tools qt4-designer
\end{tcbbash}


检查pyqt4安装情况执行以下脚本即可,显示的是当前安装的pyqt4的版本号:
\begin{tcbpython}[]
from PyQt4.QtCore import QT_VERSION_STR
print(QT_VERSION_STR)
\end{tcbpython}


\section{pyqt4模块简介}
\begin{description}
\item[QtCore] 模块包括了核心的非GUI功能,该模块用来对时间、文件、目录、各种数据类型、流、网址、媒体类型、线程或进程进行处理。
\item[QtGui] 模块包括图形化窗口部件和及相关类。包括如按钮、窗体、状态栏、滑块、位图、颜色、字体等等。
\item[QtHelp] 模块包含了用于创建和查看可查找的文档的类。
\item[QtNetwork] 模块包括网络编程的类。这些类可以用来编写TCP/IP和UDP的客户端和服务器。它们使得网络编程更容易和便捷。
\item[QtOpenGL] 模块使用OpenGL库来渲染3D和2D图形。该模块使得Qt GUI库和OpenGL库无缝集成。
\item[QtScript] 模块包含了使PyQt应用程序使用JavaScript解释器编写脚本的类。
\item[QtSql] 模块提供操作数据库的类。
\item[QtSvg] 模块提供了显示SVG文件内容的类。可缩放矢量图形(SVG)是一种用XML描述二维图形和图形应用的语言。
\item[QtTest] 模块包含了对PyQt应用程序进行单元测试的功能。(PyQt没有实现完全的Qt单元测试框架,相反,它假设使用标准的Python单元测试框架来实现模拟用户和GUI进行交互。)
\item[QtWebKit] 模块实现了基于开源浏览器引擎WebKit的浏览器引擎。
\item[QtXml] 包括处理XML文件的类,该模块提供了SAX和DOM API的接口。
\item[QtXmlPatterns] 模块包含的类实现了对XML和自定义数据模型的XQuery和XPath的支持。
\item[phonon] 模块包含的类实现了跨平台的多媒体框架,可以在PyQt应用程序中使用音频和视频内容。
\item[QtMultimedia] 模块提供了低级的多媒体功能,开发人员通常使用\textbf{phonon}模块。
\item[QtAssistant] 模块包含的类允许集成\textbf{Qt Assistant}到PyQt应用程序中,提供在线帮助。
\item[QtDesigner] 模块包含的类允许使用PyQt扩展\textbf{Qt Designer}。
\item[Qt] 模块综合了上面描述的模块中的类到一个单一的模块中。这样做的好处是你不用担心哪个模块包含哪个特定的类,坏处是加载进了整个Qt框架,从而增加了应用程序的内存占用。
\item[uic] 模块包含的类用来处理.ui文件,该文件由Qt Designer创建,用于描述整个或者部分用户界面。它包含的加载.ui文件和直接渲染以及从.ui文件生成Python代码为以后执行的类。
\end{description}


\chapter{第一个例子}
\section{窗口}
\begin{tcbpython}
import sys
from PyQt4  import QtGui

class MyQWidget(QtGui.QWidget):
    def __init__(self,parent=None):
        QtGui.QWidget.__init__(self,parent)
        self.setGeometry(0, 0, 800, 600)
        #坐标0 0 大小800 600
        self.setWindowTitle('myapp')

myapp = QtGui.QApplication(sys.argv)
mywidget = MyQWidget()
mywidget.show()
sys.exit(myapp.exec_())
\end{tcbpython}

首先导入sys宏包,是为了后面接受sys.argv参数。从PyQt4模块导入QtGui宏包,是为了后面创建QWidget类的实例。

接下来我们定义了MyQWidget类,它继承自QtGui的QWidget类。然后重定义了构造函数,首先继承了QtGui的QWidget类的构造函数,这里将parent的默认参数传递进去了。

然后通过QWidget类定义好的\textbf{setGeometry}方法来调整窗口的左顶点的坐标位置和窗口的大小。

然后通过\textbf{setWindowTitle}方法来设置这个窗口程序的标题,这里就简单设置为myapp了。

任何窗口程序都需要创建一个QApplication类的实例,这里是myapp。然后接下来创建QWidget类的实例mywidget,然后通过调用mywidget的方法\textbf{show}来显示窗体。

最后我们看到系统要退出是调用的myapp实例的\textbf{exec\_}方法。


\section{加上图标}
现在我们在前面第一个程序的基础上稍作修改,来给这个程序加上图标。程序代码如下:
\begin{tcbpython}
import sys
from PyQt4  import QtGui

class MyQWidget(QtGui.QWidget):
    def __init__(self,parent=None):
        QtGui.QWidget.__init__(self,parent)
        self.resize(800,600)
        self.setWindowTitle('myapp')
        self.setWindowIcon(QtGui.QIcon\
        ('icons/myapp.ico'))


myapp = QtGui.QApplication(sys.argv)
mywidget = MyQWidget()
mywidget.show()
sys.exit(myapp.exec_())
\end{tcbpython}



这个程序相对上面的程序就增加了一个\textbf{setWindowIcon}方法,这个方法调用了QtGui.QIcon方法,然后后面跟的就是图标的存放路径,使用相对路径。在运行这个例子的时候,请随便弄个图标文件过来。

这个程序为了简单起见就使用了QWidget类的\textbf{resize}方法来设置窗体的大小。



\section{弹出提示信息}
\begin{tcbpython}
import sys
from PyQt4  import QtGui

class MyQWidget(QtGui.QWidget):
    def __init__(self,parent=None):
        QtGui.QWidget.__init__(self,parent)
        self.resize(800,600)
        self.setWindowTitle('myapp')
        self.setWindowIcon(QtGui.QIcon\
        ('icons/myapp.ico'))
        self.setToolTip('看什么看^_^')
        QtGui.QToolTip.setFont(QtGui.QFont\
        ('微软雅黑', 12))

myapp = QtGui.QApplication(sys.argv)
mywidget = MyQWidget()
mywidget.show()
sys.exit(myapp.exec_())
\end{tcbpython}



上面这段代码和前面的代码的不同就在于MyQWidget类的初始函数新加入了两条命令。其中\textbf{setToolTip}方法设置具体显示的文本内容,然后后面调用QToolTip类的\textbf{setFont}方法来设置字体和字号,我不太清楚这里随便设置系统的字体微软雅黑是不是有效。

这样你的鼠标停放在窗口上一会儿会弹出一小段提示文字。


\section{关闭窗体时询问}
目前程序点击那个叉叉图标关闭程序的时候将会直接退出,这里新加入一个询问机制。

\begin{tcbpython}
import sys
from PyQt4  import QtGui

class MyQWidget(QtGui.QWidget):
    def __init__(self,parent=None):
        QtGui.QWidget.__init__(self,parent)
        self.resize(800,600)
        self.setWindowTitle('myapp')
        self.setWindowIcon(QtGui.QIcon\
        ('icons/myapp.ico'))
        self.setToolTip('看什么看^_^')
        QtGui.QToolTip.setFont(QtGui.QFont\
        ('微软雅黑', 12))

    def closeEvent(self, event):
        reply = QtGui.QMessageBox.question\
        (self, '信息',
            "你确定要退出吗?",
             QtGui.QMessageBox.Yes,
             QtGui.QMessageBox.No)

        if reply == QtGui.QMessageBox.Yes:
            event.accept()
        else:
            event.ignore()

myapp = QtGui.QApplication(sys.argv)
mywidget = MyQWidget()
mywidget.show()
sys.exit(myapp.exec_())
\end{tcbpython}



这段代码和前面代码的不同就是重新定义了\textbf{colseEvent}事件。这段代码的核心就是QtGui类的QMessageBox类的question方法,这个方法将会弹出一个询问窗体。这个方法接受四个参数:第一个参数是这个窗体所属的母体,这里就是self也就是实例mywidget;第二个参数是弹出窗体的标题;第三个参数是一个标准button;第四个参数也是一个标准button,是默认(也就是按enter直接选定的)的button。然后这个方法返回的是那个被点击了的标准button的标识符,所以后面和标准buttonYes比较了,然后执行event的accept方法。

这样这个程序在关闭的时候会弹出一个对话框,询问你是否真的要关闭,具体请读者自己实验一下。

\section{屏幕居中显示窗体}


\begin{tcbpython}
import sys
from PyQt4  import QtGui

class MyQWidget(QtGui.QWidget):
    def __init__(self,parent=None):
        QtGui.QWidget.__init__(self,parent)
        self.resize(800,600)
        self.center()
        self.setWindowTitle('myapp')
        self.setWindowIcon(QtGui.QIcon\
        ('icons/myapp.ico'))
        self.setToolTip('看什么看^_^')
        QtGui.QToolTip.setFont(QtGui.QFont\
        ('微软雅黑', 12))

    def closeEvent(self, event):
        #重新定义colseEvent
        reply = QtGui.QMessageBox.question\
        (self, '信息',
            "你确定要退出吗?",
             QtGui.QMessageBox.Yes,
             QtGui.QMessageBox.No)

        if reply == QtGui.QMessageBox.Yes:
            event.accept()
        else:
            event.ignore()
            
    def center(self):
        screen = QtGui.QDesktopWidget().screenGeometry()
        size =  self.geometry()
        self.move((screen.width()-size.width())/2,\
         (screen.height()-size.height())/2)

myapp = QtGui.QApplication(sys.argv)
mywidget = MyQWidget()
mywidget.show()
sys.exit(myapp.exec_())
\end{tcbpython}



这个例子和前面相比改动是新建了一个center方法,接受一个实例,这里是mywidget。然后对这个实例也就是窗口的具体位置做一些调整。

QDesktopWidget类的\textbf{screenGeometry}方法返回一个量,这个量的width属性就是屏幕的宽度(按照pt像素计,比如1366×768\\,宽度就是1366),这个量的height属性就是屏幕的高度。

然后QWidget类的\textbf{geometry}方法同样返回一个量,这个量的width是这个窗体的宽度,这个量的height属性是这个窗体的高度。

然后调用QWidget类的move方法,这里是对mywidget这个实例作用。我们可以看到move方法的X,Y是从屏幕的坐标原点 (0,0) 开始计算的。第一个参数X表示向右移动了多少宽度,Y表示向下移动了多少高度。

整个函数的作用效果就是将这个窗体居中显示。


\section{QMainWindow类}
QtGui.QMainWindow类提供应用程序主窗口,可以创建一个经典的拥有状态栏、工具栏和菜单栏的应用程序骨架。(之前使用的是QWidget类,现在换成QMainWindow类。)

前面第一个例子都是用的QtGui.QWidget类创建的一个窗体。关于QWidget和QMainWindow这两个类的区别\href{http://stackoverflow.com/questions/3298792/whats-the-difference-between-qmainwindow-and-qwidget-and-qdialog}{参考这个网站}得出的结论是:QWdget类在Qt中是所有可画类的基础(这里的意思可能是窗体的基础吧。) 任何基于QWidget的类都可以作为独立窗体而显示出来而不需要母体(parent)。

QMainWindow类是针对主窗体一般需求而设计的,它预定义了菜单栏状态栏和其他widget(窗口小部件) 。因为它继承自QWidget,所以前面谈及的一些属性修改都适用于它。那么首先我们将之前的代码中的QWidget类换成QMainWindow类。



\begin{tcbpython}
import sys
from PyQt4  import QtGui

class MainWindow(QtGui.QMainWindow):
    def __init__(self,parent=None):
        QtGui.QMainWindow.__init__(self,parent)
        self.resize(800,600)
        self.center()
        self.setWindowTitle('myapp')
        self.setWindowIcon(QtGui.QIcon\
        ('icons/myapp.ico'))
        self.setToolTip('看什么看^_^')
        QtGui.QToolTip.setFont(QtGui.QFont\
        ('微软雅黑', 12))

    def closeEvent(self, event):
        reply = QtGui.QMessageBox.question\
        (self, '信息',
            "你确定要退出吗?",
             QtGui.QMessageBox.Yes,
             QtGui.QMessageBox.No)

        if reply == QtGui.QMessageBox.Yes:
            event.accept()
        else:
            event.ignore()

    def center(self):
        screen = QtGui.QDesktopWidget().screenGeometry()
        size =  self.geometry()
        self.move((screen.width()-size.width())/2,\
         (screen.height()-size.height())/2)

myapp = QtGui.QApplication(sys.argv)
mainwindow = MainWindow()
mainwindow.show()
sys.exit(myapp.exec_())
\end{tcbpython}



现在程序运行情况良好,我们继续加点东西进去。


\section{加上状态栏}

\begin{tcbpython}[]
#!/usr/bin/env python3
#-*- coding: utf-8 -*-
import sys
from PyQt4  import QtGui

class MainWindow(QtGui.QMainWindow):
    def __init__(self,parent=None):
        QtGui.QMainWindow.__init__(self,parent)
        self.resize(800,600)
        self.center()
        self.setWindowTitle('myapp')
        self.setWindowIcon(QtGui.QIcon\
        ('icons/myapp.ico'))
        self.setToolTip('看什么看^_^')
        QtGui.QToolTip.setFont(QtGui.QFont\
        ('微软雅黑', 12))


    def closeEvent(self, event):
        reply = QtGui.QMessageBox.question\
        (self, '信息',
            "你确定要退出吗?",
             QtGui.QMessageBox.Yes,
             QtGui.QMessageBox.No)

        if reply == QtGui.QMessageBox.Yes:
            event.accept()
        else:
            event.ignore()

    def center(self):
        screen = QtGui.QDesktopWidget().screenGeometry()
        size =  self.geometry()
        self.move((screen.width()-size.width())/2,\
         (screen.height()-size.height())/2)

myapp = QtGui.QApplication(sys.argv)
mainwindow = MainWindow()
mainwindow.show()
mainwindow.statusBar().showMessage('程序已就绪...')
sys.exit(myapp.exec_())
\end{tcbpython}



这个程序和前面的区别在于最后倒数第二行,调用mainwindow这个QMainWindow类生成的实例的\textbf{statusBar}方法生成一个QStatusBar对象,然后调用QStatusBar类的\textbf{showMessage}方法来显示一段文字。

如果你希望这段代码在\verb+__init__+方法里面,那么具体实现过程也与上面描述的类似。考虑到状态栏有些程序不一定需要,这里暂时不加到构造函数里面了。


\section{加上菜单栏}

\begin{tcbpython}[]
#!/usr/bin/env python3
#-*- coding: utf-8 -*-
import sys
from PyQt4  import QtGui

class MainWindow(QtGui.QMainWindow):
    def __init__(self,parent=None):
        QtGui.QMainWindow.__init__(self,parent)
        self.resize(800,600)
        self.center()
        self.setWindowTitle('myapp')
        self.setWindowIcon(QtGui.QIcon\
        ('icons/myapp.ico'))
        self.setToolTip('看什么看^_^')
        QtGui.QToolTip.setFont(QtGui.QFont\
        ('微软雅黑', 12))

    #菜单栏
        self.menubar = self.menuBar()
        menu_file=self.menubar.addMenu('文件')
        menu_edit=self.menubar.addMenu('编辑')
        menu_help=self.menubar.addMenu('帮助')

    def closeEvent(self, event):
        reply = QtGui.QMessageBox.question\
        (self, '信息',
            "你确定要退出吗?",
             QtGui.QMessageBox.Yes,
             QtGui.QMessageBox.No)

        if reply == QtGui.QMessageBox.Yes:
            event.accept()
        else:
            event.ignore()

    def center(self):
        screen = QtGui.QDesktopWidget().screenGeometry()
        size =  self.geometry()
        self.move((screen.width()-size.width())/2,\
         (screen.height()-size.height())/2)

myapp = QtGui.QApplication(sys.argv)
mainwindow = MainWindow()
mainwindow.show()
mainwindow.statusBar().showMessage('程序已就绪...')
sys.exit(myapp.exec_())
\end{tcbpython}

和上面讨论加上状态栏类似,这里用QMainWindow类的menuBar方法来获得一个菜单栏对象。然后用这个菜单栏对象的\textbf{addMenu}方法来创建一个新的菜单对象(QMenu类)——方法里面的内容是新建菜单显示的名字。

建议给菜单对象取个名字,后面方便引用。



\section{加上动作}

\begin{tcbpython}[]
#!/usr/bin/env python3
#-*- coding: utf-8 -*-
import sys
from PyQt4  import QtGui

class MainWindow(QtGui.QMainWindow):
    def __init__(self,parent=None):
        QtGui.QMainWindow.__init__(self,parent)
        self.resize(800,600)
        self.center()
        self.setWindowTitle('myapp')
        self.setWindowIcon(QtGui.QIcon\
        ('icons/myapp.ico'))
        self.setToolTip('看什么看^_^')
        QtGui.QToolTip.setFont(QtGui.QFont\
        ('微软雅黑', 12))

    #动作和连接
        act_exit = QtGui.QAction('退出', self)
        act_exit.setStatusTip('退出程序')
        act_exit.triggered.connect(self.close)

        act_about = QtGui.QAction('关于本程序', self)
        act_about.triggered.connect(self.about)

        act_aboutqt = QtGui.QAction('关于Qt', self)
        act_aboutqt.triggered.connect(self.aboutqt)

    #菜单栏
        self.menubar = self.menuBar()
        menu_file=self.menubar.addMenu('文件')
        menu_file.addAction(act_exit)
        menu_edit=self.menubar.addMenu('编辑')
        menu_help=self.menubar.addMenu('帮助')
        menu_help.addAction(act_about)
        menu_help.addAction(act_aboutqt)

#函数
    def about(self):
        QtGui.QMessageBox.about(self,"关于本程序","本程序是一个用于教学的程序。
        \n\nFell free to use it\n(including it's source code).")
    def aboutqt(self):
        QtGui.QMessageBox.aboutQt(self)

    def closeEvent(self, event):
        reply = QtGui.QMessageBox.question\
        (self, '信息',
            "你确定要退出吗?",
             QtGui.QMessageBox.Yes,
             QtGui.QMessageBox.No)

        if reply == QtGui.QMessageBox.Yes:
            event.accept()
        else:
            event.ignore()

    def center(self):
        screen = QtGui.QDesktopWidget().screenGeometry()
        size =  self.geometry()
        self.move((screen.width()-size.width())/2,\
         (screen.height()-size.height())/2)

myapp = QtGui.QApplication(sys.argv)
mainwindow = MainWindow()
mainwindow.show()
mainwindow.statusBar().showMessage('程序已就绪...')
sys.exit(myapp.exec_())
\end{tcbpython}

现在在前面例子的基础上给之前的菜单对象加上动作。比如menu\_{}file就是之前创建的一个菜单对象,现在调用这个对象的\textbf{addAction}方法,将act\_{}exit动作对象加进去。

act\_{}exit动作对象需要在前面定义,通过QtGui类的QAction类的构造函数来创建一个动作对象。构造函数最少需要两个参量:第一个是显示的文本,第二个是这个动作依附的母体(也就是常见的parent变量),通常这里填上self即可。

在这里这个动作对象,就是菜单的下拉选项,如果我们用鼠标点击一下的话,将会触发\textbf{triggered}信号,如果我们connect方法连接到某个槽上(或者某个你定义的函数),那么将会触发这个函数的执行。下面就信号-槽机制详细说明之。

\subsection{信号-槽机制}
GUI程序一般都引入一种事件和信号机制,well,简单来说就是一个循环程序,这个循环程序等到某个时刻程序会自动做某些事情比如刷新程序界面啊,或者扫描键盘鼠标之类的,等用户点击鼠标或者按了键盘之后,它会接受这个信号然后做出相应的反应。

所以你一定猜到了,close函数可能就是要退出这个循环程序。我们调用主程序的\verb+exec_+方法,就是开启这个循环程序。

\begin{tcbpython}
myapp.exec_()
\end{tcbpython}

pyqt4的旧信号-槽连接语句我在这里忽略了,网上到处都是。下面就新的语句说明之。

\begin{tcbpython}
act_exit.triggered.connect(self.close)
\end{tcbpython}

我们看到新的信号-槽机制语句变得更精简更易懂了。整个过程就是如我前面所述,某个对象发出了某个信号,然后用connect将这个信号和某个槽(或者你定义的某个函数)连接起来即形成了一个反射弧了。

这里的槽就是self主窗口实例的close方法,这个是主窗口自带的函数。

然后我们看到aboutqt和about函数。具体读者如果不懂请翻阅QMessageBox类的静态方法about和aboutqt。



\chapter{简单的编辑器}
简单编辑器这个例子还是在前面第一个例子的基础上进一步加上了一些功能,使之更加像一个我们平时所见的那种编辑器。这里为了节省篇幅代码我就不列出来了,具体的代码在文件夹[pyqt4-works]的[简单的编辑器]里面。

别看这个例子似乎有点复杂了,但其实它相对于前面的第一个例子只是加上了pyqt4内置的文本编辑器窗件,然后做了一些优化,仅此而已。当然相对于成熟的实际使用的那种文本编辑器而言还有很大的改进优化空间。

请读者看到\verb+#编辑器+的下面的代码,这里利用pyqt4的QTextEdit类来创建了一个文本编辑器对象,然后新建了一个变量,fileName,这个存储着当前程序打开的文件名。然后QMainWindow类的布局默认中间是什么CentralWidget,这里将其设置为self.editor也就是我们创建的文本编辑器对象即可。

做到这里我们发现一个文本编辑窗口已经出来了,如果你愿意,这就算是一个最最简单的文本编辑器了,只是一些最最基本的功能没加上使得这个程序毫无用处。所谓的基本的功能包括新建文件,打开文件,保存文件,另存为文件,判断文件是否已保存(退出时),剪切复制粘贴等。

这里例子是最最精简的版本,推荐读者自己阅读代码然后有不懂的查阅pyqt4的各个类的说明,比如\href{http://pyqt.sourceforge.net/Docs/PyQt4/classes.html}{这个官方的类的查阅网站}。

有时间我会在这里继续补上具体代码更详细的说明。






\chapter{pyqt4实例}
这是本文关于pyqt4介绍的重点内容,通过一个个从简单到复杂逐步深入的实例来学习pyqt4,我认为这是最有效率的。具体更加详细的讨论在各个子项目中有时间我会慢慢加上。

\section{按钮打印你好}
\begin{Verbatim}
import sys
from PyQt4  import QtGui

class Mybutton(QtGui.QPushButton):
    def __init__(self,parent=None):
        QtGui.QPushButton.__init__(self,parent)
        self.resize(150, 90)
        self.center()
        self.setWindowTitle('你好')
        self.clicked.connect(self.hello)

    def center(self):
        screen = QtGui.QDesktopWidget().screenGeometry()
        size =  self.geometry()
        self.move((screen.width()-size.width())/2,\
         (screen.height()-size.height())/2)

    def hello(self):
        print('你好')

myapp = QtGui.QApplication(sys.argv)
mywidget = Mybutton()
mywidget.show()

myapp.exec_()
sys.exit()
\end{Verbatim}
这个代码最值得一提的就是pyqt4最新的信号-槽机制。就是某个对象的某个信号调用connect方法连接到某个槽上,所谓槽实际上就是定义的函数,只是这个函数没加上()圆括号,只是一个函数符号,所以称之为槽。


\section{文件对话框}
\begin{Verbatim}
import sys
from PyQt4  import QtGui

class Mybutton(QtGui.QPushButton):
    def __init__(self,parent=None):
        QtGui.QPushButton.__init__(self,"打开文件",parent)
        self.resize(150, 90)
        self.center()
        self.setWindowTitle('打开文件')
        self.clicked.connect(self.openfile)

    def center(self):
        screen = QtGui.QDesktopWidget().screenGeometry()
        size =  self.geometry()
        self.move((screen.width()-size.width())/2,\
         (screen.height()-size.height())/2)


    def openfile(self):
        filename=QtGui.QFileDialog.getOpenFileName(self,"打开文件...",".","")
        print('文件'+str(filename)+'已选择')

myapp = QtGui.QApplication(sys.argv)
mywidget = Mybutton()
mywidget.show()

myapp.exec_()
sys.exit()
\end{Verbatim}
这个例子和上面例子的区别就在于将按钮的动作改成了openfile,然后用QFileDialog的静态方法getOpenFileName来获取打开文件的对话框。


\section{更多的按钮更多的对话框}
这个例子我们在前面两个例子的基础上,加上更多的按钮并对应更多的对话框(文件选择对话框,颜色选择对话框,字体选择对话框,一般信息对话框,一般选择对话框(关闭机制))。同时引入一种简单的布局。

\begin{Verbatim}
import sys
from PyQt4  import QtGui

class Mywidget(QtGui.QWidget):
    def __init__(self,parent=None):
        QtGui.QWidget.__init__(self,parent)
        self.center()
        self.setWindowTitle('myapp')

        button1=QtGui.QPushButton("文件")
        button2=QtGui.QPushButton("颜色")
        button3=QtGui.QPushButton("字体")
        button4=QtGui.QPushButton("关于")
        button5=QtGui.QPushButton("关于Qt")
        mainlayout=QtGui.QHBoxLayout()
        mainlayout.addWidget(button1)
        mainlayout.addWidget(button2)
        mainlayout.addWidget(button3)
        mainlayout.addWidget(button4)
        mainlayout.addWidget(button5)
        self.setLayout(mainlayout)
        button1.clicked.connect(self.openfile)
        button2.clicked.connect(self.choosecolor)
        button3.clicked.connect(self.choosefont)
        button4.clicked.connect(self.about)
        button5.clicked.connect(self.aboutqt)

    def center(self):
        screen = QtGui.QDesktopWidget().screenGeometry()
        size =  self.geometry()
        self.move((screen.width()-size.width())/2,\
         (screen.height()-size.height())/2)

    def openfile(self):
        filename=QtGui.QFileDialog.getOpenFileName(self,"打开文件...",".","")
        print('文件'+str(filename)+'已选择')

    def choosecolor(self):
        colorname=QtGui.QColorDialog.getColor()
        print('颜色'+str(colorname)+'已选择')

    def choosefont(self):
        fontname=QtGui.QFontDialog.getFont()
        print('字体'+str(fontname)+'已选择')

    def about(self):
        QtGui.QMessageBox.about(self,"关于本程序","本程序用于测试按钮和对话框。")
    def aboutqt(self):
        QtGui.QMessageBox.aboutQt(self)

    def closeEvent(self, event):
        #重新定义colseEvent
        reply = QtGui.QMessageBox.question\
        (self, '信息',
            "你确定要退出吗?",
             QtGui.QMessageBox.Yes,
             QtGui.QMessageBox.No)

        if reply == QtGui.QMessageBox.Yes:
            event.accept()
        else:
            event.ignore()

myapp = QtGui.QApplication(sys.argv)
mywidget = Mywidget()
mywidget.show()
myapp.exec_()
sys.exit()
\end{Verbatim}



\section{funnyclock例子}
请参看github项目:\href{https://github.com/a358003542/funnyclock}{一个有趣的时钟例子}


\section{timer例子}
请参看github项目:\href{https://github.com/a358003542/timer}{一个计时和倒计时程序}


\section{myretext例子}
请参看github项目:\href{https://github.com/a358003542/myretext}{简单模拟了retext软件},实现了即时编写markdown和即时显示的功能。



\chapter{pyqt4实例进阶}



\part{kivy教程}
\section{安装}
具体安装请参看官方文档,Ubuntu下我的安装如下:
\begin{tcbbash}[]
sudo add-apt-repository ppa:thopiekar/pygame
sudo add-apt-repository ppa:kivy-team/kivy
sudo apt-get update
sudo apt-get install python3-kivy
sudo apt-get install kivy-examples
\end{tcbbash}

\subsection{测试安装情况}
第一个例子作为测试安装情况:
\begin{tcbpython}[]
import kivy
#kivy.require('1.8.0')

from kivy.app import App
from kivy.uix.label import Label
class MyApp(App):
    def build(self):
        return Label(text='hello')


if __name__ == '__main__':
    MyApp().run()
\end{tcbpython}






\part{网络编程}


\part{其他}
\chapter{暂时还不知道分类的东东}

\section{matplotlib加入中文}
\href{http://blog.sciencenet.cn/blog-43412-343002.html }{参考网站}

\section{matplotlib棒状图上加上说明文字}
\href{http://stackoverflow.com/questions/7423445/how-can-i-display-text-over-columns-in-a-bar-chart-in-matplotlib}{参考网站}



\section{从excel中读取数据}

xlrd模块

\section{原处修改对象的函数没有返回值}
这个问题我就遇到过,这个原则一定要牢记在心!

\begin{Verbatim}
>>> lst=[1,2,3]
>>> lst=lst.append(4)
>>> print(lst)
None
\end{Verbatim}

\part{其他python实例程序}




















\part{附录}
\chapter{sql数据库技术}
\begin{flushright}
\begin{notecard}[red!30]{12em}
 一个程序没有数据库哪有什么用处啊。
\end{notecard}
\end{flushright}

更多内容请参看指尖上的Ubuntu一书其中的sql技术相关章节。

\chapter{pyqt4类参考}
详细的pyqt4类的参考请参看\href{http://pyqt.sourceforge.net/Docs/PyQt4/classes.html}{这个网页}。下面只是尽可能的将我涉及到的一些pyqt4的类的一些信息尽可能的写上去,某些可以临时做为参考,不全也不可能做到齐全。。

\section{QHBoxLayout类}
在QtGui模块中,继承自QBoxLayout,QBoxLayout继承自QLayout,QLayout继承自QObject和QLayoutItem。

简单布局,让元素横向排列。

使用方法如下:
\begin{Verbatim}
mainlayout=QtGui.QHBoxLayout()
mainlayout.addWidget(button1)
mainlayout.addWidget(button2)
self.setLayout(mainlayout)
\end{Verbatim}
首先建立一个布局对象,有QHBoxLayout类生成,然后在这个布局实例里面用\textbf{addWidget}方法来加入元素,然后将这个布局用主窗口的\textbf{setLayout}方法设置进去。

\section{QVBoxLayout类}
在QtGui模块中,继承自QBoxLayout,QBoxLayout继承自QLayout,QLayout继承自QObject和QLayoutItem。

和QHBoxLayout类似,简单布局,竖向排列。


\section{QWidget类}
在QtGui模块中,继承自QObject和QPaintDevice。

\subsection{构造函数}
\begin{Verbatim}
QWidget.__init__(self, QWidget parent = None, Qt.WindowFlags flags = 0)
\end{Verbatim}

默认parent是None,如果是其他widget,那么这个widget就变成那个(参量)widget的子窗口了,如果那个(参量)widget删除了,这个widget也就是删除了。

\subsection{激活窗口}
\begin{Verbatim}
QWidget.activateWindow(self)
\end{Verbatim}
activateWindow方法用于激活这个窗口,这个窗口是可见的并且可以键盘输入。点击窗口的标题栏一个窗口就上前显现了用的就是这个函数。

\subsection{调整窗口大小}
\begin{Verbatim}
QWidget.setGeometry(self, QRect)
QWidget.setGeometry(self, int ax, int ay, int aw, int ah)
QWidget.resize(self, QSize)
QWidget.resize(self, int w, int h)
\end{Verbatim}

\subsection{设置图标}
\begin{Verbatim}
QWidget.setWindowIcon(self, QIcon icon)
QWidget.setWindowIconText(self, QString)
\end{Verbatim}
就是设置程序下面显示的图标和图标文字。(QString对象我测试了直接用python3的字符串对象也是可以的。)


\subsection{设置窗口标题}
\begin{Verbatim}
QWidget.setWindowTitle(self, QString)
\end{Verbatim}


\subsection{show方法}
\begin{Verbatim}
QWidget.show(self)
\end{Verbatim}
显示某个窗口和它的子窗口,相当于让这个窗口可见,和\textbf{setVisible(True)}等价。

\subsection{设置提示词}
\begin{Verbatim}
QWidget.setToolTip(self, QString)
\end{Verbatim}

\subsection{closeEvent方法}
\begin{Verbatim}
QWidget.closeEvent (self, QCloseEvent)
\end{Verbatim}
一般在窗口关闭时调用,你可以重定义这个方法来改变窗口在面对关闭事件时的反应。



\section{QIcon类}
在QtGui模块中。

这个类提供了可缩放图标的解决方案。

\subsection{构造函数}
\begin{Verbatim}
__init__(self)
__init__(self, QPixmap pixmap)
__init__(self, QIcon other)
__init__(self, QString fileName)
__init__(self, QIconEngine engine)
__init__(self, QIconEngineV2 engine)
__init__(self, QVariant variant)
\end{Verbatim}

最常用的是第四种形式,也就是用相对路径(要注意操作系统的不同可能带来的问题)引用某个图标文件。


\section{QPushButton类}
来自QtGui模块,继承自QAbstractButton类,QAbstractButton类继承自QWidget类。

QPushButton也就是GUI设计中最常见的按钮。

\subsection{构造函数}
\begin{Verbatim}
__init__(self, QWidget parent = None)
__init__(self, QString text, QWidget parent = None)
__init__(self, QIcon icon, QString text, QWidget parent = None)
\end{Verbatim}
我们看到按钮显示的文字和图标在这里是可以设置的。


\section{QFileDialog}
来自QtGui模块,继承自QDialog,QDialog继承自QWidget。

弹出一个文件或目录选择对话框。

简单的使用常通过静态方法来实现:
\subsection{静态方法}
更多信息请参见官方类文档。

\subsubsection{getExistingDirectory方法}
返回用户选定的已存在的目录。这里parent是对话框依附的父窗口,caption是弹出窗口的标题,directory是窗口弹出时默认打开的位置,options是?

\subsubsection{getOpenFileName方法}
返回用户选定的已存在文件。比如:
\begin{Verbatim}
filename=QtGui.QFileDialog.getOpenFileName(self,"打开文件...",".","")
\end{Verbatim}
这里filename是字符串类型,就是你选择的文件的名字(包含绝对引用地址),后面"."是当前目录的意思,在后面过滤器空字符串表示所有的文件都显示。


\section{QColorDialog类}
来自QtGui模块,继承自QDialog,QDialog继承自QWidget。

弹出一个选择颜色的窗口。

\subsection{静态方法}
\subsubsection{getColor方法}
简单的设置不给参数也是可以的。

\section{QFontDialog类}
来自QtGui模块,继承自QDialog,QDialog继承自QWidget。

弹出一个选择字体的窗口。
\subsection{静态方法}
\subsubsection{getFont方法}
简单的设置不给参数也是可以的。


\section{QMessageBox类}
来自QtGui模块,继承自QDialog,QDialog继承自QWidget。

弹出一个对话框用于用户做出一些选择或者给出一些提示信息。

\subsection{静态方法}
\subsubsection{about方法}
需要三个参数,第一个参数是窗口的依附父窗口,第二个参数是窗口的标题,第三个参数是窗口的文字信息。

\subsubsection{aboutQt方法}
最少只需要一个参数,即弹出窗口的依附父窗口。


\section{QMainWindow类}



\section{QDialog类}

%这里空一行

\end{common-format}
\end{document}
% !Mode:: "TeX:UTF-8"%確保文檔utf-8編碼
%新加入的命令如下:addchtoc addsectoc reduline showendnotes hlabel
%新加入的环境如下:common-format  fig linefig xverbatim

\documentclass[12pt,oneside]{book}
\newlength{\textpt}
\setlength{\textpt}{12pt}
\newif\ifphone
\phonefalse


\usepackage{myconfig}
\usepackage{mytitle}




\begin{document}
\frontmatter

\titlea{pyqt4指南}
\author{万泽}
\authorinfo{作者:}
\editor{德山书生}
\email{a358003542@gmail.com}
\editorinfo{编者:德山书生,湖南常德人氏。}
\version{0.1}
\titleLA

\addchtoc{前言}
\chapter*{前言}
\begin{common-format}
前置知识:python语言基础知识。从变量各个操作对象到程序结构直到类。


本文参考资料:\\
1.pyqt4教程,\href{http://blog.cx125.com/books/PyQt4_Tutorial/}{http://blog.cx125.com/books/PyQt4\_{}Tutorial/}

2.Prentice Hall,Rapid GUI Programming with Python and Qt,2007年10月

3.


%这里空一行。

\end{common-format}


\addchtoc{目录}
\setcounter{tocdepth}{2}
\tableofcontents

\begin{common-format}
\mainmatter

\chapter{刚开始}

\section{安装pyqt4}
ubuntu下安装pyqt4即安装python3-pyqt4即可:\\
\verb+sudo apt-get install python3-pyqt4+

检查pyqt4安装情况执行以下脚本即可,显示的是当前安装的pyqt4的版本号:
\begin{Verbatim}
from PyQt4.QtCore import QT_VERSION_STR
print(QT_VERSION_STR)
\end{Verbatim}

\section{pyqt4模块简介}
\begin{description}
\item[QtCore] 模块包括了核心的非GUI功能,该模块用来对时间、文件、目录、各种数据类型、流、网址、媒体类型、线程或进程进行处理。
\item[QtGui] 模块包括图形化窗口部件和及相关类。包括如按钮、窗体、状态栏、滑块、位图、颜色、字体等等。
\item[QtHelp] 模块包含了用于创建和查看可查找的文档的类。
\item[QtNetwork] 模块包括网络编程的类。这些类可以用来编写TCP/IP和UDP的客户端和服务器。它们使得网络编程更容易和便捷。
\item[QtOpenGL] 模块使用OpenGL库来渲染3D和2D图形。该模块使得Qt GUI库和OpenGL库无缝集成。
\item[QtScript] 模块包含了使PyQt应用程序使用JavaScript解释器编写脚本的类。
\item[QtSql] 模块提供操作数据库的类。
\item[QtSvg] 模块提供了显示SVG文件内容的类。可缩放矢量图形(SVG)是一种用XML描述二维图形和图形应用的语言。
\item[QtTest] 模块包含了对PyQt应用程序进行单元测试的功能。(PyQt没有实现完全的Qt单元测试框架,相反,它假设使用标准的Python单元测试框架来实现模拟用户和GUI进行交互。)
\item[QtWebKit] 模块实现了基于开源浏览器引擎WebKit的浏览器引擎。
\item[QtXml] 包括处理XML文件的类,该模块提供了SAX和DOM API的接口。
\item[QtXmlPatterns] 模块包含的类实现了对XML和自定义数据模型的XQuery和XPath的支持。
\item[phonon] 模块包含的类实现了跨平台的多媒体框架,可以在PyQt应用程序中使用音频和视频内容。
\item[QtMultimedia] 模块提供了低级的多媒体功能,开发人员通常使用\textbf{phonon}模块。
\item[QtAssistant] 模块包含的类允许集成\textbf{Qt Assistant}到PyQt应用程序中,提供在线帮助。
\item[QtDesigner] 模块包含的类允许使用PyQt扩展\textbf{Qt Designer}。
\item[Qt] 模块综合了上面描述的模块中的类到一个单一的模块中。这样做的好处是你不用担心哪个模块包含哪个特定的类,坏处是加载进了整个Qt框架,从而增加了应用程序的内存占用。
\item[uic] 模块包含的类用来处理.ui文件,该文件由Qt Designer创建,用于描述整个或者部分用户界面。它包含的加载.ui文件和直接渲染以及从.ui文件生成Python代码为以后执行的类。
\end{description}


\chapter{第一个例子}
\section{窗口}
\begin{Verbatim}
import sys
from PyQt4  import QtGui

class MyQWidget(QtGui.QWidget):
    def __init__(self,parent=None):
        QtGui.QWidget.__init__(self,parent)
        self.setGeometry(0, 0, 800, 600)
        #坐标0 0 大小800 600
        self.setWindowTitle('myapp')

myapp = QtGui.QApplication(sys.argv)
mywidget = MyQWidget()
mywidget.show()
sys.exit(myapp.exec_())
\end{Verbatim}

首先导入sys宏包,是为了后面接受sys.argv参数\footnote{这里sys.argv就是这个py文件的文件名组成的列表:['窗口.py']}。从PyQt4模块导入QtGui宏包,是为了后面创建QWidget类的实例。

接下来我们定义了MyQWidget类,它继承自QtGui的QWidget类。然后重定义了构造函数,首先继承了QtGui的QWidget类的构造函数,这里将parent的默认参数传递进去了。

然后通过QWidget类定义好的\textbf{setGeometry}方法来调整窗口的左顶点的坐标位置和窗口的大小。

然后通过\textbf{setWindowTitle}方法来设置这个窗口程序的标题,这里就简单设置为myapp了。

任何窗口程序都需要创建一个QApplication类的实例,这里是myapp。然后接下来创建QWidget类的实例mywidget,然后通过调用mywidget的方法\textbf{show}来显示窗体。

最后我们看到系统要退出是调用的myapp实例的\textbf{exec\_}方法。


\section{加上图标}
现在我们在前面第一个程序的基础上稍作修改,来给这个程序加上图标。程序代码如下:
\begin{Verbatim}
import sys
from PyQt4  import QtGui

class MyQWidget(QtGui.QWidget):
    def __init__(self,parent=None):
        QtGui.QWidget.__init__(self,parent)
        self.resize(800,600)
        self.setWindowTitle('myapp')
        self.setWindowIcon(QtGui.QIcon\
        ('icons/myapp.ico'))


myapp = QtGui.QApplication(sys.argv)
mywidget = MyQWidget()
mywidget.show()
sys.exit(myapp.exec_())
\end{Verbatim}

这个程序相对上面的程序就增加了一个\textbf{setWindowIcon}方法,这个方法调用了QtGui.QIcon方法,然后后面跟的就是图标的存放路径,使用相对路径。在运行这个例子的时候,请随便弄个图标文件过来。

这个程序为了简单起见就使用了QWidget类的\textbf{resize}方法来设置窗体的大小。



\section{弹出提示信息}
\begin{Verbatim}
import sys
from PyQt4  import QtGui

class MyQWidget(QtGui.QWidget):
    def __init__(self,parent=None):
        QtGui.QWidget.__init__(self,parent)
        self.resize(800,600)
        self.setWindowTitle('myapp')
        self.setWindowIcon(QtGui.QIcon\
        ('icons/myapp.ico'))
        self.setToolTip('看什么看^_^')
        QtGui.QToolTip.setFont(QtGui.QFont\
        ('微软雅黑', 12))

myapp = QtGui.QApplication(sys.argv)
mywidget = MyQWidget()
mywidget.show()
sys.exit(myapp.exec_())
\end{Verbatim}

上面这段代码和前面的代码的不同就在于MyQWidget类的初始函数新加入了两条命令。其中\textbf{setToolTip}方法设置具体显示的文本内容,然后后面调用QToolTip类的\textbf{setFont}方法来设置字体和字号,我不太清楚这里随便设置系统的字体微软雅黑是不是有效。

这样你的鼠标停放在窗口上一会儿会弹出一小段提示文字。


\section{关闭窗体时询问}
目前程序点击那个叉叉图标关闭程序的时候将会直接退出,这里新加入一个询问机制。
\begin{Verbatim}
import sys
from PyQt4  import QtGui

class MyQWidget(QtGui.QWidget):
    def __init__(self,parent=None):
        QtGui.QWidget.__init__(self,parent)
        self.resize(800,600)
        self.setWindowTitle('myapp')
        self.setWindowIcon(QtGui.QIcon\
        ('icons/myapp.ico'))
        self.setToolTip('看什么看^_^')
        QtGui.QToolTip.setFont(QtGui.QFont\
        ('微软雅黑', 12))

    def closeEvent(self, event):
        reply = QtGui.QMessageBox.question\
        (self, '信息',
            "你确定要退出吗?",
             QtGui.QMessageBox.Yes,
             QtGui.QMessageBox.No)

        if reply == QtGui.QMessageBox.Yes:
            event.accept()
        else:
            event.ignore()

myapp = QtGui.QApplication(sys.argv)
mywidget = MyQWidget()
mywidget.show()
sys.exit(myapp.exec_())
\end{Verbatim}

这段代码和前面代码的不同就是重新定义了\textbf{colseEvent}事件。这段代码的核心就是QtGui类的QMessageBox类的question方法,这个方法将会弹出一个询问窗体。这个方法接受四个参数:第一个参数是这个窗体所属的母体,这里就是self也就是实例mywidget;第二个参数是弹出窗体的标题;第三个参数是一个标准button;第四个参数也是一个标准button,是默认(也就是按enter直接选定的)的button。然后这个方法返回的是那个被点击了的标准button的标识符,所以后面和标准buttonYes比较了,然后执行event的accept方法。

这样这个程序在关闭的时候会弹出一个对话框,询问你是否真的要关闭,具体请读者自己实验一下。

\section{屏幕居中显示窗体}
\begin{Verbatim}
import sys
from PyQt4  import QtGui

class MyQWidget(QtGui.QWidget):
    def __init__(self,parent=None):
        QtGui.QWidget.__init__(self,parent)
        self.resize(800,600)
        self.center()
        self.setWindowTitle('myapp')
        self.setWindowIcon(QtGui.QIcon\
        ('icons/myapp.ico'))
        self.setToolTip('看什么看^_^')
        QtGui.QToolTip.setFont(QtGui.QFont\
        ('微软雅黑', 12))

    def closeEvent(self, event):
        #重新定义colseEvent
        reply = QtGui.QMessageBox.question\
        (self, '信息',
            "你确定要退出吗?",
             QtGui.QMessageBox.Yes,
             QtGui.QMessageBox.No)

        if reply == QtGui.QMessageBox.Yes:
            event.accept()
        else:
            event.ignore()
            
    def center(self):
        screen = QtGui.QDesktopWidget().screenGeometry()
        size =  self.geometry()
        self.move((screen.width()-size.width())/2,\
         (screen.height()-size.height())/2)

myapp = QtGui.QApplication(sys.argv)
mywidget = MyQWidget()
mywidget.show()
sys.exit(myapp.exec_())
\end{Verbatim}
这个例子和前面相比改动是新建了一个center方法,接受一个实例,这里是mywidget。然后对这个实例也就是窗口的具体位置做一些调整。

QDesktopWidget类的\textbf{screenGeometry}方法返回一个量,这个量的width属性就是屏幕的宽度(按照pt像素计,比如1366×768\\,宽度就是1366),这个量的height属性就是屏幕的高度。

然后QWidget类的\textbf{geometry}方法同样返回一个量,这个量的width是这个窗体的宽度,这个量的height属性是这个窗体的高度。

然后调用QWidget类的move方法,这里是对mywidget这个实例作用。我们可以看到move方法的X,Y是从屏幕的坐标原点 (0,0) 开始计算的。第一个参数X表示向右移动了多少宽度,Y表示向下移动了多少高度。

整个函数的作用效果就是将这个窗体居中显示。


\section{QMainWindow类}
QtGui.QMainWindow类提供应用程序主窗口,可以创建一个经典的拥有状态栏、工具栏和菜单栏的应用程序骨架。(之前使用的是QWidget类,现在换成QMainWindow类。)

前面第一个例子都是用的QtGui.QWidget类创建的一个窗体。关于QWidget和QMainWindow这两个类的区别\href{http://stackoverflow.com/questions/3298792/whats-the-difference-between-qmainwindow-and-qwidget-and-qdialog}{参考这个网站}得出的结论是:QWdget类在Qt中是所有可画类的基础(这里的意思可能是窗体的基础吧。) 任何基于QWidget的类都可以作为独立窗体而显示出来而不需要母体(parent)。

QMainWindow类是针对主窗体一般需求而设计的,它预定义了菜单栏状态栏和其他widget(窗口小部件) 。因为它继承自QWidget,所以前面谈及的一些属性修改都适用于它。那么首先我们将之前的代码中的QWidget类换成QMainWindow类。


\begin{Verbatim}
import sys
from PyQt4  import QtGui

class MainWindow(QtGui.QMainWindow):
    def __init__(self,parent=None):
        QtGui.QMainWindow.__init__(self,parent)
        self.resize(800,600)
        self.center()
        self.setWindowTitle('myapp')
        self.setWindowIcon(QtGui.QIcon\
        ('icons/myapp.ico'))
        self.setToolTip('看什么看^_^')
        QtGui.QToolTip.setFont(QtGui.QFont\
        ('微软雅黑', 12))

    def closeEvent(self, event):
        reply = QtGui.QMessageBox.question\
        (self, '信息',
            "你确定要退出吗?",
             QtGui.QMessageBox.Yes,
             QtGui.QMessageBox.No)

        if reply == QtGui.QMessageBox.Yes:
            event.accept()
        else:
            event.ignore()

    def center(self):
        screen = QtGui.QDesktopWidget().screenGeometry()
        size =  self.geometry()
        self.move((screen.width()-size.width())/2,\
         (screen.height()-size.height())/2)

myapp = QtGui.QApplication(sys.argv)
mainwindow = MainWindow()
mainwindow.show()
sys.exit(myapp.exec_())
\end{Verbatim}

现在程序运行情况良好,我们继续加点东西进去。


\section{加上状态栏}
在\verb+__init__+方法下加入语句:\\
\verb+self.statusBar().showMessage('这是状态栏')+

这样就显示了状态栏信息。

这里用QMainWindow类的statusBar方法获得状态栏,然后用状态栏的showMessage方法插入状态栏信息。


\section{加上菜单栏}
用QMainWindow类的menuBar方法来获得一个菜单栏。然后用这个菜单栏对象的addMenu方法来创建一个新的菜单对象——方法里面的内容是新建菜单显示的名字。

\subsection{加上动作}
某个菜单对象使用addAction方法来加上某个动作。\\
\verb+file.addAction(exit)+


也就是所谓的Action对象,通过QtGui类的QAction子类创建动作对象。创建过程的三个参数是图标,文本和母体。

\verb+exit.setStatusTip('退出程序')+\\
setStatusTip方法设置状态栏提示信息。


\subsection{事件和信号}
下面就pyqt4中的事件和信号机制详细说明之。

将事件和信号联系起来用connect方法,该方法接受三个参数:第一个是对于谁,第二个是做了什么,第三个下面做什么。

\begin{Verbatim}
self.connect(exit, QtCore.SIGNAL('triggered()'),
 QtCore.SLOT('close()'))
\end{Verbatim}

这里的意思是对于self上面的exit对象,接受了信号triggered,然后执行操作close()。


\section{加上工具栏}





%这里空一行

\end{common-format}
\end{document}




% !Mode:: "TeX:UTF-8"%確保文檔utf-8編碼
%新加入的命令如下:addchtoc addsectoc reduline showendnotes hlabel
%新加入的环境如下:common-format  fig linefig xverbatim

\documentclass[12pt,oneside]{book}
\newlength{\textpt}
\setlength{\textpt}{12pt}
\newif\ifphone
\phonefalse


\usepackage{myconfig}
\usepackage{mytitle}




\begin{document}
\frontmatter

\titlea{书籍}
\titleb{使用\LaTeX 排版}
\titlec{一种良好的风格}
\author{作者}
\authorinfo{作者:}
\editor{德山书生}
\email{a358003542@gmail.com}
\editorinfo{编者:德山书生,湖南常德人氏。}
\version{0.1}
\titleLC

\addchtoc{前言}
\chapter*{前言}
\begin{common-format}
开头说的话

%这里空一行。

\end{common-format}


\addchtoc{目录}
\setcounter{tocdepth}{2}
\tableofcontents

\begin{common-format}
\mainmatter

\section{安装}
安装使用类似的语法:
\begin{Verbatim}
sudo apt-get install python3-numpy
\end{Verbatim}

\begin{Verbatim}
import numpy as np
\end{Verbatim}
后面的例子都假定使用了这个语句。


\section{ndarray对象}
\begin{Verbatim}
import numpy as np
x = np.array([1,2,3,4,5,6])
print(x)
type(x)
\end{Verbatim}


\section{ndarray元素的引用}
语法和list对象中元素的引用类似。


\section{多维ndarray}
\begin{Verbatim}
y=np.array([[1,2,3,4,5],[6,7,8,9,10]])
y,y[0][1],y[0]
\end{Verbatim}


\section{shape属性}
ndarray对象有一个shape属性,表示几行几列。
\begin{Verbatim}
y.shape
\end{Verbatim}

\section{dtype属性}
ndarray对象有一个dtype属性,表示存储相同单元的数据类型。

dtype属性还有小属性itemsize
\begin{Verbatim}
In [1]: import numpy as np
In [2]: a=np.arange(5)
In [3]: a
Out[3]: array([0, 1, 2, 3, 4])
In [4]: a.dtype
Out[4]: dtype('int32')
In [5]: a.dtype.itemsize
Out[5]: 4
\end{Verbatim}




\section{in语句}
in语句测量某元素是不是在这个ndarray对象中。
\begin{Verbatim}
8 in y, 11 in y
\end{Verbatim}

\section{reshape方法}
\begin{Verbatim}
In [8]: ndarray001=np.arange(1,10).reshape(3,3)
In [9]: ndarray001
Out[9]: 
array([[1, 2, 3],
       [4, 5, 6],
       [7, 8, 9]])
\end{Verbatim}




\section{copy方法}



\section{ndarray变成list}


\section{类似range的arange函数}

\section{flatten方法}


\section{resize方法}


\section{transpose方法}


\section{eye方法}
创造单位矩阵

\begin{Verbatim}
In [10]: ndarray001=np.eye(3)
In [11]: ndarray001
Out[11]: 
array([[ 1.,  0.,  0.],
       [ 0.,  1.,  0.],
       [ 0.,  0.,  1.]])

\end{Verbatim}


\section{读写文件}
\begin{Verbatim}
In [10]: ndarray001=np.eye(3)
In [12]: np.savetxt("ndarray001.txt",ndarray001)
In [13]: x=np.loadtxt("ndarray001.txt")
In [14]: x
Out[14]: 
array([[ 1.,  0.,  0.],
       [ 0.,  1.,  0.],
       [ 0.,  0.,  1.]])
\end{Verbatim}




\chapter{参考资料}
1.An introduction to Numpy and Scipy  2012 M. Scott Shell

2.NumPy Beginner's Guide 2013 Ivan Idris

%这里空一行

\end{common-format}
\end{document}



